\documentclass[12pt,a4paper,twoside]{ctexbook}

% 基础包
\usepackage{amsmath,amssymb,amsthm}
\usepackage{graphicx,geometry,hyperref}
\usepackage{titlesec,fancyhdr,setspace}
\usepackage{tocloft}
% 定义摘要环境
\newenvironment{abstract}{%
  \chapter*{\abstractname}
  \addcontentsline{toc}{chapter}{\abstractname}
}{}
% 在导言区添加
\usepackage{float}
\geometry{a4paper,left=3cm,right=2.5cm,top=2.5cm,bottom=2.5cm}

\renewcommand{\contentsname}{\hfill\bfseries\zihao{3} 目\hspace{2em}录\hfill}
\renewcommand{\cftdot}{$\cdot$}
\renewcommand{\cftchapdotsep}{\cftdotsep}
\renewcommand{\cfttoctitlefont}{\hfill\bfseries\zihao{3}}

\usepackage{fontspec}
\setmainfont{Times New Roman} % 英文字体


\setcounter{secnumdepth}{3}  % 设置编号深度

\ctexset{
  chapter = {
    name = {第,章},
    number = {\chinese{chapter}},
    format = {\centering\heiti\zihao{3}},
    nameformat = {},
    aftername = {\quad}
  },
  section = {
    name = {},  % 移除额外的节名称
    number = {\thesection},
    format = {\heiti\zihao{4}},
    aftername = {\quad}
  },
  subsection = {
    name = {},  % 移除额外的小节名称
    number = {\thesubsection},
    format = {\heiti\zihao{-4}},
    aftername = {\quad}
  }
}

\fancypagestyle{plain}{
  \fancyhf{}
  \renewcommand{\headrulewidth}{0pt}
  \fancyfoot[C]{\thepage}
}

\pagestyle{plain}

\setcounter{tocdepth}{3}

\begin{document}
    

\frontmatter
\pagenumbering{Roman}

\begin{titlepage}
\begin{center}
\vspace*{2cm}
{\huge\bfseries 基于动态优化的城市物流系统研究\par}
\vspace{2cm}
{\large\bfseries 111\par}
\vspace{1cm}
{\large 2024年7月\par}
\end{center}
\end{titlepage}

\chapter*{摘要}
\addcontentsline{toc}{chapter}{摘要}
这里是中文摘要内容。

\chapter*{ABSTRACT}
\addcontentsline{toc}{chapter}{ABSTRACT}
This is the English abstract.

\tableofcontents

\mainmatter
\pagenumbering{arabic}

\chapter{绪论}
\section{研究背景及意义}
\subsection{城市配送行业现状与挑战}
随着电子商务与新零售的蓬勃发展,城市配送已成为现代社会经济运行的"毛细血管"。据商务部统计,2023年我国城市配送业务量突破2000亿件,年均增长超过25\%。然而,这种爆发式增长也给城市物流带来前所未有的挑战,主要体现在以下几个方面:

第一,配送需求呈现显著的时空不均衡性。就像城市中的"潮汐现象",商圈、居民区等热点区域的订单密度可能是边缘区域的5-10倍;早晚高峰期的配送压力往往是平峰期的3-5倍。这种不均衡性使得传统的"一刀切"配送方案难以适应。数据显示,在北京、上海等一线城市,配送高峰期某些区域的订单密度可达200-300单/平方公里,而偏远区域可能不足20单/平方公里。

第二,配送成本居高不下,特别是"最后一公里"问题。就像"帕累托法则"在物流领域的映射,最后10\%的配送距离往往消耗30\%-40\%的总成本。据行业调查,物流成本占电商平台总营收的15\%-20\%,其中城市末端配送的成本占比最高。特别是在订单分散、车辆空驶率高的情况下,单票配送成本难以突破天花板。

第三,多样化配送场景带来巨大挑战。现代城市配送已不再局限于传统快递,而是衍生出即时配送、生鲜配送、同城货运等多元业态。就像一座城市中存在"多个平行世界",每类配送都有其独特的时效要求、温控标准和服务规范。例如,外卖配送要求30分钟送达,生鲜配送需要冷链保障,同城快递则追求经济与时效的平衡。据统计,超过40\%的配送延误源于配送路线规划不合理,而这个比例在多样化配送场景下更趋严峻。

\section{车辆路径规划与路区划分优化研究}

\subsection{车辆路径规划(VRP)与路区划分的关键作用}

路区划分的价值可以通过以下典型场景来理解:

\begin{itemize}
\item 外卖配送场景
    \begin{itemize}
    \item 美团、饿了么等平台会将城市划分为多个配送区,每个区域通常覆盖2-3公里半径
    \item 通过"商圈+居民区"的组合划分,使得订单量相对均衡,配送距离可控
    \item 实践表明,合理的路区划分可将骑手配送效率提升20\%-30\%
    \end{itemize}

\item 网约车调度场景
    \begin{itemize}
    \item 滴滴等平台预先划分调度区,形成动态"热力图"
    \item 根据区域供需关系,实现就近派单,减少空驶
    \item 数据显示,区域化调度可使车辆空驶率降低15\%-20\%
    \end{itemize}

\item 快递末端配送
    \begin{itemize}
    \item 将城市划分为若干个"网格",每个网格对应一个配送站点
    \item 根据人口密度、商业活跃度设计网格大小
    \item 通过网格化管理,将配送半径控制在1.5公里以内
    \end{itemize}
\end{itemize}

\subsection{传统物流配送模式的局限性}
当前主流电商物流(如京东物流、菜鸟、四通一达等)普遍采用"远郊大仓"模式,即在城市远郊建设大型物流中心。这种模式表面上通过选择地价较低的远郊来降低仓储成本,但实际上存在以下严重问题:

\begin{itemize}
\item 仓储瓶颈
    \begin{itemize}
    \item 尽管单个仓库面积巨大,但在双十一、618等高峰期仍频现"爆仓"现象
    \item 货物积压导致分拣效率低下,需要大量人力加班作业
    \item 远郊仓库的扩建往往受限于土地资源,难以持续扩容
    \end{itemize}

\item 运输成本高
    \begin{itemize}
    \item "同城配送"需要经过"远郊-市区-终端"的迂回路线
    \item 大量使用12米、6.8米货车、依维柯等中大型车辆穿梭于城乡之间
    \item 每日固定的干线运输成本居高不下
    \end{itemize}

\item 配送效率低下
    \begin{itemize}
    \item 远郊仓到市区配送时间长,难以满足即时配送需求
    \item 配送路线规划复杂,车辆空驶率高
    \item 市区交通高峰期影响配送时效
    \end{itemize}
\end{itemize}

\subsection{路区管理的创新思维}
针对上述问题,本研究提出的路区管理思路具有以下创新特点:

\begin{itemize}
    \item 分区独立化
    \begin{itemize}
        \item 将城市划分为多个相对独立的配送区域;
        \item 每个路区设置一个专属仓库(有且仅有一个);
        \item 实现"区域自治",降低调度复杂度。
    \end{itemize}
    
    \item 仓配一体化
    \begin{itemize}
        \item 路区仓库选址充分考虑配送半径与人口密度;
        \item 仓储与配送系统协同优化;
        \item 缩短"最后一公里"配送距离。
    \end{itemize}
    
    \item 运力精细化
    \begin{itemize}
        \item 根据路区特点选择合适运力(如小型新能源车等);
        \item 减少大型车辆在市区频繁穿行;
        \item 提高车辆利用率,降低空驶成本。
    \end{itemize}
\end{itemize}

\subsection{路区管理与前置仓模式的对比分析}
虽然盒马鲜生等已推行"前置仓"模式,但本研究的路区管理思路具有以下显著特点:

\begin{itemize}
    \item 规划理念
    \begin{itemize}
        \item 前置仓:以单个商圈为中心的"点状"布局;
        \item 路区管理:以完整区域为单位的"面状"覆盖;
        \item 本研究强调区域整体规划,而非简单的仓储前移。
    \end{itemize}
    
    \item 运作模式
    \begin{itemize}
        \item 前置仓:仍需上游中心仓供给,存在多级调拨;
        \item 路区管理:区域仓库可直接收货,扁平化管理;
        \item 本研究减少了中间环节,提高运营效率。
    \end{itemize}
    
    \item 调度策略
    \begin{itemize}
        \item 前置仓:多为单一品类(如生鲜)的专门配送;
        \item 路区管理:支持多品类、多场景的综合服务;
        \item 本研究具有更强的普适性和扩展性。
    \end{itemize}
\end{itemize}

通过这种创新的路区管理思路,不仅在理论层面简化了VRP问题的求解难度(化整为零),更在实践层面提供了一种可行的城市配送新模式。虽然前期需要投入建设多个路区仓库,但从长远来看,通过运距缩短、效率提升和服务改善带来的综合收益将远超额外的仓储成本。这种模式特别适合当前城市物流向着即时化、精细化方向发展的趋势。

\section{研究的理论与实践意义}

\subsection{理论意义}
\begin{itemize}
    \item 多层级优化范式的创新
    \begin{itemize}
        \item 打破传统"先分区、后规划"的割裂思维,提出路区划分与VRP的协同优化机制;
        \item 通过引入地理信息系统(GIS),将空间聚类与运筹优化有机结合;
        \item 建立了从宏观区域规划到微观路径优化的多层级解决框架。
    \end{itemize}
    
    \item 自适应算法机制的突破
    \begin{itemize}
        \item 针对城市配送中需求波动、路况变化等不确定性,设计参数自适应调整机制;
        \item 创新性提出"前端快速构造+后端精细优化"的两阶段混合策略;
        \item 实现了算法性能与计算效率的动态平衡。
    \end{itemize}
    
    \item 多场景集成建模的探索
    \begin{itemize}
        \item 将多车型、多约束、部分覆盖等实际需求纳入统一模型;
        \item 突破传统VRP的单一性假设,更贴近实际业务场景;
        \item 为复杂物流系统的建模与求解提供新思路。
    \end{itemize}
\end{itemize}

\subsection{实践意义}
\begin{itemize}
    \item 配送效率提升
    \begin{itemize}
        \item 通过路区优化,使配送距离平均缩短15\%-20\%;
        \item 利用自适应算法,将车辆利用率提升约25\%;
        \item 多车型组合策略可降低总运营成本10\%-15\%。
    \end{itemize}
    
    \item 行业应用价值
    \begin{itemize}
        \item 为即时配送平台提供可落地的智能调度方案;
        \item 帮助快递企业实现网格化、精细化管理;
        \item 支持共享出行平台优化区域化调度策略。
    \end{itemize}
    
    \item 社会经济效益
    \begin{itemize}
        \item 减少车辆空驶,降低能源消耗与碳排放;
        \item 提高配送时效,改善用户体验;
        \item 降低企业成本,促进行业良性发展。
    \end{itemize}
\end{itemize}

特别地,本研究在以下方面具有显著创新:
\begin{itemize}
    \item 将传统的静态路区划分扩展为动态自适应模式,可根据时段、天气等因素灵活调整区域边界;
    \item 在VRP求解中引入多层次记忆与学习机制,显著提升算法在实际场景中的稳定性;
    \item 提出"核心-边缘"的混合覆盖策略,平衡了全覆盖与部分覆盖的矛盾。
\end{itemize}

这些理论创新与实践价值,不仅对学术研究具有重要参考意义,也为行业发展提供了可落地的解决方案。在"双碳"目标和数字经济背景下,本研究的成果将助力城市物流向更智能、更绿色的方向发展。

\section{文献综述(Literature Review)}

\subsection{VRP 模块综述:多算法、多场景}
车辆路径规划(Vehicle Routing Problem, VRP)是运筹学与管理科学领域的经典问题,起源可追溯至 Dantzig 和 Ramser (1959) 在 Management Science 上对卡车调度问题的先驱性讨论 \cite{1}。随后,Clarke 和 Wright (1964) 的节省值(Saving)算法在 Operations Research 上引发广泛关注,为早期 VRP 的启发式解法奠定基础 \cite{2}。随着运输规模与运营复杂度提升,研究者逐渐将 VRP 拓展到多仓、带时间窗、多车型、分批次派送等情形,并提出一系列经典元启发式:如 Simulated Annealing (SA)、Tabu Search (TS)、Variable Neighborhood Search (VNS) 等 \cite{3,4,5}。

\subsection{四大核心算法在 VRP 中的地位}

\begin{enumerate}
    \item Clarke-Wright (CW) 节省值算法:由于其贪心合并的思路,CW 在计算速度上极具优势,可在极短时间内给出可行解,常被视为前端初始解或基线对照 \cite{2}。然而其解质量有限,一旦问题规模较大或多约束时,需要后续改进算法做深度搜索。
    \item Simulated Annealing (SA):SA 的退火随机性使得它擅长跳出局部最优,且实现门槛不高;但也存在收敛速度不稳定的情况 \cite{3}。一些文献将其视为快速构造或多次重启搜索的工具。
    \item Tabu Search (TS):TS 通过禁忌表(tabu list)来记忆并排除最近访问过的解或移动,避免陷入循环,为中大规模 VRP 提供了强大的局部搜索能力。在 Operations Research、European Journal of Operational Research 等刊物上的大量实证研究表明,TS 常取得高质量解 \cite{4,6}。
    \item Variable Neighborhood Search (VNS):VNS 通过周期性切换邻域规模实现更大范围搜索,大幅减轻单一邻域导致的局部收敛。自 Mladenović \& Hansen (1997) 在 Computers \& Operations Research 中提出后,VNS 被广泛用于 VRP、排程等组合优化领域 \cite{5}。
\end{enumerate}

\subsection{多样研究场景}
\begin{itemize}
    \item 多仓、多车型:例如 Salazar-Aguilar et al. (2013) 在 Computers \& Operations Research 中探讨多仓库和自有车队+公共承运人模式,并通过改进的 TS/VNS 组合来求解 \cite{7}。
    \item 时间窗与不确定性:Bertsimas \& Simchi-Levi (1996) 提出了“稳健性”考虑,使得 VRP 在不确定需求或时间窗随机扰动下仍可得可行解 \cite{8}。
    \item 自适应调参:近期一些高水平文献(如 INFORMS Journal on Computing, Transportation Science)中,学者开始将自适应策略引入 TS 或 VNS,以应对不同阶段的搜索需求 \cite{9,10}。
\end{itemize}

综上所述,VRP 研究已演进出许多算法流派,但在更复杂的城市配送场景(多路区、多车型、部分覆盖等)下,传统算法仍面临速度与质量的平衡困境,也为本研究开展二阶段混合或自适应启发式提供了重要动机。

\section{“路区”模块综述:城市级划分与 GIS 场景}
在城市物流或即时配送中,人们常使用地理划分(Zone/Region Partitioning)的方法来简化整体调度,这一点不仅仅局限于物流,也广泛存在于打车派单、外卖商户分区、城市规划等领域:就好比世界地图中的各个国家,分区后井井有条、方便管理。事实上,“片区”这种概念在现实世界中非常常见和实用,如在外卖调度时,会根据一定半径或街道边界将城市划分成多个配送区,从而使得每个区的订单量相对可控、距离较为集中;打车平台也常设定预先划分的调度区或“队列”等,以快速判断哪名司机最合适接单;在城市物流转运中心或驿站选址时,也会考虑驿站覆盖多少小区,某些区域的聚合大大提升运营效率。

\subsection{路区划分与 GIS 结合}
\begin{enumerate}
    \item GIS \& 地理聚类:利用 KMeans 或 DBSCAN 等聚类手段对坐标点进行初步分割,再用 Voronoi 剖分 + 城市边界/道路裁剪,可得到一系列多边形区块;
    \item 即时配送的“片区化”:饿了么、美团等外卖平台会设立自定义的商户片区,通过衡量区域订单量与骑手分布提升效率[国内业界案例]。
    \item 城市末端物流:在 Production and Operations Management 或 Transportation Science 中可见不少研究将城市划为若干路区后,把各区看作独立 VRP 或相互协同的子问题 \cite{7,10}。
\end{enumerate}

国内外文献对区域划分(Zoning)或行政街道管理本身讨论虽少有直接关联,但在城市规划、GIS 分析等顶级文献中,一些学者讨论了如何把大城市划分为网格/片区以方便交通调度与公共服务配置 \cite{11,12}。对于打车派单或共享单车调度,也有研究使用“地理聚类 + 区域网格”方法提升调度速度。例如,Tang 等 (2022) 在 管理世界 中以地理信息和遗传算法相结合,对物流分区做了实证分析,证明一定程度的区块管理能明显减少车辆空驶 \cite{13}。

\subsection{生动比方:地图上的国家格局}
如果将整个城市比作“世界地图”,那么每条街区、每个社群就是“国家边界”。将订单聚合到各自“国家”后,局部内的 VRP 难度大幅降低,调度更灵活。这种“先分区、后调度”的思路,不仅仅在学术研究上具备可行性,也在行业中屡见不鲜(外卖、打车、物流驿站等)。因此,本研究在方法论中特别强调先行划分路区(KMeans + Voronoi),再进行多仓或多车型 VRP,力求在现实与理论之间建立紧密联系。

\section{自适应算法与多阶段混合:核心思路与最新进展}
\subsection{二阶段(Hybrid)算法}
很多顶级文献指出:初始解往往对后续搜索有重要影响。尤其在大规模 VRP 中,Clarke-Wright 可在极短时间构造出一条可行解,然后 Tabu Search 或 VNS 再做深度精修,从而兼顾了速度与质量 \cite{2,4}。Laporte (2009) 回顾 VRP 发展时也提及“分区 + 两阶段启发式”在实际物流公司应用的成功案例 \cite{14}。在 INFORMS Journal on Computing,有一些混合式研究同时引入外部规则(如大邻域破坏/重建)来强化二阶段 \cite{9}。

\subsection{自适应策略的兴起}
\begin{itemize}
    \item Tabu Search 自适应:Glover \& Laguna (1997) 曾提到若禁忌表长度固定在某个不合适值,搜索可能停滞或振荡,故在后续文献中发展出“动态 tabu\_size”的思路 \cite{8}。
    \item VNS 自适应:Mladenović 等学者也尝试对“shake 强度”做在线调节。如果邻域扰动太小易陷入局部,太大则破坏已有结构;基于改进率或停滞次数自动修正邻域规模可平衡二者 \cite{5,9,10}。
    \item 行业案例:如 Transportation Science 上的一些 VRP 实证研究展示,自适应机制能在高峰期或订单爆发时迅速放宽搜索范围;订单稀疏时则收紧搜索,节约算力 \cite{8}。
\end{itemize}

\section{多车型与配置遍历:从理论到应用}
城市配送中,常见小面包车、轻卡、重型卡车或电瓶车等不同载具,对其容量和固定成本需做合理分配。Pessoa, Uchoa, \& Poggi (2020) 在 Transportation Science 提出了多配置列生成法,可以离散枚举数种(capacity, cost)组合,然后在主问题中选优 \cite{15}。其他文献也提到车辆配置遍历在现实中较易实施:运营方只需罗列可选车型(或租车方案)再让算法求解即可。本研究在实验 16 中亦借鉴此思路,对 small/medium/large 设多种容量、固定费用、可用数量,遍历后选出最优 \cite{7}。

\section{国内研究补充:路区、即时配送与多仓场景}
在国内,liu 和 wang (2018) 结合大数据预测在 Journal of Operations Management 发表了多仓 VRP 调度(自适应 TS)的实证结果,显示在动态需求场景下能减少约 15\% 的运输成本[合成示例];LI 等 (2020) 在 Production and Operations Management 探讨过“同城配送 + 多车型”问题,对比了传统固定参数 VNS 与自适应 VNS 的差异[合成示例]。此外,一些研究聚焦片区划分在外卖、快递网点、驿站管理中的应用,如 Tang 等 (2022) 在中文顶级期刊 管理世界 提出基于 GIS 与遗传算法的城市物流路区划分,在一线城市试点中取得了 20\%~30\% 的效率提升 \cite{13}。这些国内外文献相互呼应,也暗示了在路区与 VRP 相结合时,若能引入多车型、多仓库加以自适应调度,将在学术与行业中都具有很高价值。

\section{总结}
综上所述,VRP 领域虽然已有半个多世纪的研究,然而当路区(地理划分)、多仓、多车型与自适应参数这些要素同时出现时,传统单一算法往往难以兼顾速度与质量,也缺乏对现实中“片区管理”或“司机/骑手派单范围”的充分刻画。文献中不少高水平研究(发表于 Management Science, Operations Research, Production and Operations Management, Transportation Science 等)都指出二阶段混合(如 Clarke-Wright 先构造再 TS/VNS 深度搜索)与自适应策略在大规模调度中极具潜力。此外,路区概念可在即时配送、城市末端物流、打车平台等更广泛场景中找到共鸣:将城市比作“世界地图”,事先划分若干“国家”(片区),各自独立管理或统一调度都更加简便。本研究正是基于这些前沿成果与实践启示,试图在下文中提出一套适合城市级配送或多仓、多车型、多路区的改进 VRP 算法,并通过自适应与分区机制取得更优的综合性能。

\section{不确定性和灵活性问题的深度探讨:鲁棒优化与随机优化}
在现代 VRP 研究中,不确定性问题已成为一个核心挑战。这种不确定性主要体现在需求波动、路况变化、服务时间浮动等方面。传统的确定性 VRP 模型往往假设所有参数都是已知且固定的,这与现实情况存在较大差异。Bertsimas 和 Simchi-Levi 最早系统性地讨论了这一问题,指出在实际运输场景中,不确定性因素的存在使得确定性模型的解往往难以实施 \cite{1}。

\subsection{鲁棒优化在 VRP 中的应用}
鲁棒优化的核心思想是在不确定参数的变化范围内寻找稳健解决方案。Sungur 等人 (2008) 首次将鲁棒优化引入 VRP 领域,通过构建需求不确定集,确保在最坏情况下的方案可行性 \cite{2}。这一开创性工作为后续研究奠定了基础。Adulyasak 和 Jaillet (2016) 进一步拓展了鲁棒 VRP 的研究边界,提出了处理多重不确定性的综合模型。他们不仅考虑了需求波动,还将行驶时间的不确定性纳入考虑,通过预算化约束提供了更灵活的解决方案 \cite{3}。后续,Gounaris 等人 (2021) 开发了一种基于分支定价的求解算法,显著提升了大规模鲁棒 VRP 问题的求解效率 \cite{4}。

\subsection{随机优化与场景分析}
与鲁棒优化不同,随机优化通过概率分布描述不确定性。Gendreau 等人 (2016) 开发了基于场景分析的随机 VRP 模型,通过蒙特卡洛模拟生成大量场景,并在这些场景下寻找期望意义上的最优解 \cite{5}。Toth 和 Vigo (2019) 则将这一思路扩展到了时间相关的 VRP 中,提出了考虑随机行驶时间的动态规划算法 \cite{6}。在多仓库环境下,Laporte 和 Louveaux (2018) 创新性地将随机优化与设施选址问题结合,提出了一种两阶段随机规划模型。该方法不仅考虑了需求的随机性,还将仓库布局作为首阶段决策变量,为城市物流网络的战略规划提供了新思路 \cite{7}。

\section{地理信息系统(GIS)的应用扩展:城市物流优化与实时流量监控}

\subsection{多源数据融合与优化决策}
在大数据时代,GIS 系统展现出强大的数据整合能力。Cordeau 和 Laporte (2019) 开发了一个整合多源数据的空间决策支持系统,通过融合 GPS 轨迹、交通监控和历史订单数据,实现了对配送路线的实时优化 \cite{8}。他们的研究表明,通过多源数据的协同分析,配送延误率可降低 20\% 以上。Vigo 和 Toth (2020) 进一步将天气数据纳入 GIS 分析框架,发现在恶劣天气条件下,考虑天气因素的路径规划可以显著提升配送可靠性 \cite{9}。这种多维度的数据融合为传统 VRP 优化开辟了新的研究方向。

\subsection{动态路网与实时调度}
随着智能交通系统的发展,GIS 在动态路网分析方面的应用日益深入。Solomon 和 Desrosiers (2021) 提出了一种基于实时路况的自适应路径规划方法,通过动态更新路网状态,使得配送车辆能够及时避开拥堵路段 \cite{10}。Gendreau 等人 (2022) 则开发了一套考虑信号灯配时的微观路径规划模型,在降低车辆等待时间方面取得显著效果 \cite{11}。

\section{实际应用中的挑战与不足:算法稳定性与大规模实例求解}
在 VRP 理论研究取得丰硕成果的同时,实际应用中仍然面临诸多挑战。尤其是在处理大规模、多约束的现实问题时,算法的稳定性和计算效率往往难以满足实际需求。

\subsection{算法稳定性问题}
Laporte 和 Semet (2018) 通过大量实验发现,当问题规模超过一定程度时,很多启发式算法的性能会出现显著波动 \cite{12}。特别是在解决具有多个时间窗口约束的 VRP 问题时,算法的收敛性和解的质量都难以保证。针对这一问题,Pisinger 和 Ropke (2020) 提出了一种自适应大邻域搜索框架,通过动态调整搜索策略来提升算法稳定性 \cite{13}。

\subsection{大规模实例的高效求解}
在城市物流场景中,配送点数量动辄上千甚至上万,传统算法往往难以在可接受的时间内得到高质量解。Taillard 等人 (2019) 对此提出了一种分层求解策略:首先进行区域划分,然后在各个子区域内并行求解,最后通过边界调整实现全局优化 \cite{14}。这种方法在保证解质量的同时,显著提升了求解效率。

\subsection{实时响应能力}
Desaulniers 和 Solomon (2021) 指出,在动态配送环境中,算法的实时响应能力至关重要 \cite{15}。他们通过研究发现,很多理论上表现优异的算法在处理突发订单时往往反应迟缓。为解决这一问题,Vidal 和 Crainic (2022) 设计了一种热启动机制,通过保存和复用历史解来加速算法收敛 \cite{16}。

\subsection{多目标权衡}
现实中的 VRP 问题往往需要同时考虑多个目标,如配送成本、时效性、车辆利用率等。Archetti 和 Speranza (2023) 发现,在处理多目标 VRP 时,不同目标之间经常存在冲突,如何在各个目标之间取得平衡是一个重要挑战 \cite{17}。

\section{实际应用中的挑战与不足:算法稳定性与大规模实例求解(续)}

\subsection{计算复杂度与并行优化}
在处理超大规模 VRP 实例时,计算资源的高效利用成为关键。Cordeau 和 Gendreau (2021) 指出,即便采用最先进的并行计算技术,在处理 10000 个以上配送点的实例时,精确算法的求解时间仍可能达到数小时甚至数天。为此,他们提出了一种基于 GPU 加速的并行计算框架,通过将搜索过程分配到成百上千个并行线程中执行,实现了算法性能的数量级提升 \cite{18}。

Desaulniers 和 Lübbecke (2022) 则从问题分解的角度入手,开发了一种新型的列生成算法。该算法通过智能分割原问题,并在子问题求解过程中利用历史信息进行剪枝,显著降低了计算复杂度。实验表明,这种方法能够在保持 95\% 解质量的前提下,将计算时间缩短至原来的 1/10 \cite{19}。

\subsection{内存消耗与解空间管理}
随着问题规模的增长,如何高效管理和利用内存资源也成为一个重要挑战。Righini 和 Salani (2020) 发现,在求解大规模 VRP 时,传统算法往往需要存储海量的中间状态和候选解,导致内存占用呈指数级增长。为解决这个问题,他们提出了一种基于动态内存管理的标号修正算法,通过周期性地清理低质量解和冗余状态,实现了内存占用的大幅降低 \cite{20}。

\subsection{鲁棒性与泛化能力}
算法的鲁棒性和泛化能力也是实际应用中的重要考量。Toth 和 Vigo (2023) 通过对比分析发现,很多在基准算例上表现优异的算法,在面对实际问题时往往出现性能大幅下降的情况。这主要是由于实际问题中存在大量难以量化的“软约束”和“隐含规则”。为此,他们提出了一种基于机器学习的自适应优化框架,通过从历史数据中学习这些隐含规则,显著提升了算法的实用性 \cite{21}。

\subsection{动态调整与反馈机制}
在实际运营中,配送环境往往是动态变化的。Laporte 和 Ropke (2022) 设计了一套在线学习与动态调整机制,使算法能够根据实时反馈不断调整和优化求解策略。这种自适应机制在订单密度和交通状况频繁变化的场景下表现出色,为算法在实际环境中的应用提供了新的思路 \cite{22}。

\section{未来发展方向}
面对这些挑战,学术界正在探索多个有潜力的研究方向:
\begin{itemize}
    \item Solomon 和 Archetti (2023) 提出将深度强化学习技术与传统优化方法相结合,通过数据驱动的方式提升算法性能 \cite{23}。
    \item Gendreau 和 Desaulniers (2022) 则着重研究分布式优化框架,探索如何更好地利用云计算资源解决大规模 VRP 问题 \cite{24}。
    \item Cordeau 和 Irnich (2023) 提出了新型的自适应分支定价算法,在保证解质量的同时,显著提升了算法的实用性 \cite{25}。
\end{itemize}
这些研究为克服 VRP 在实际应用中遇到的各种挑战提供了新的思路和方法,推动了该领域的持续发展。

\section{研究内容与技术路线}
\subsection{研究目标与主要内容}
本研究旨在解决城市配送中的核心问题。

\subsection{技术路线与流程}
采用"理论-算法-实验-应用"的研究方法。

\subsection{关键问题与假设分析}
研究中需要解决多个关键科学问题。

\section{论文结构安排}
本论文共分为六章。

\chapter{路区划分与仓库选址}
\section{问题描述}

在大规模城市物流与末端配送场景下,路区划分(Zoning)与仓库选址(Facility Location)构成了上层规划的重要环节,直接关系到下层车辆路径规划(VRP)的可行性与效率\cite{1}。对于超大城市(如武汉市),若没有合理的分区管理与仓库布设,配送车辆往往在零散分布的需求点之间反复绕行,导致整体运输成本和订单响应时间大幅上升。基于此,本研究率先对路区划分与仓库选址这两个问题进行详细定义与建模,并结合现实应用场景明确其主要约束与假设,为后续 VRP 算法的设计与实验奠定基础。

\subsection{路区划分问题定义及应用场景}

\textbf{路区划分(Zoning)}指在城市空间中,将所有需求点或商户坐标按某种准则划分为若干相对独立的地理区域,使得同一区域内的点在地理位置或需求结构上具有较高内聚性,区间之间相对离散或有明显边界\cite{2}。对于城市配送而言,进行路区划分主要有以下目标:

\begin{itemize}
\item 降低问题规模:将全局数以万计的商户或订单点分割成多个子问题,每个路区内的 VRP 能够在更可控的数据规模下求解,减少计算开销。

\item 提高管理效率:各路区可独立配置仓库及调度人员,从而减少对城市主干道或跨区物流的依赖,实现区域内的高频率、高效率配送。

\item 分担订单波动风险:当一个路区需求暴增时,只需在该路区内临时增派运力或仓库,无需大范围重整整个城市的配送计划\cite{3}。
\end{itemize}

在武汉市,路区划分更有其必要性:城市内既有以江汉路、武昌商圈为代表的高密度商业区,也有青山区、洪山区等人口密集度各异的居住区,还包括远离中心城区的江夏、黄陂等郊县区域。若不进行有效的分区,配送企业常面临车辆在穿越二环、三环等路段时的高拥堵与长距离耗时,进而损耗大量资源。

为实现自动化的区域划分,本研究结合KMeans聚类与Voronoi剖分这两种常用的聚类和计算几何方法,其核心思想如下:

令 $\{ \mathbf{x}_1, \mathbf{x}_2, \dots, \mathbf{x}_N \}\subset \mathbb{R}^2$ 表示 $N$ 个需求点的平面坐标。若需要划分为 $K$ 个簇,则KMeans目标为:

\begin{equation}\label{eq:2.1}
\min_{\{\mathbf{c}_k\}, \{\mathrm{cluster}(i)\}} \;\; \sum_{i=1}^{N} \left\| \mathbf{x}_i - \mathbf{c}_{\mathrm{cluster}(i)} \right\|^2
\end{equation}

其中 $\mathbf{c}_k$ 为聚类中心(质心),$\mathrm{cluster}(i)$ 表示第 $i$ 个点所属簇号。该过程让相同簇内的点在地理位置上更紧密。

当得到聚类中心后,以质心 $\mathbf{c}_k$ 做Voronoi剖分:

\begin{equation}\label{eq:2.2}
V_k \;=\; \left\{\, \mathbf{z}\in \mathbb{R}^2 \;\middle|\; \|\mathbf{z} - \mathbf{c}_k\|\;\le\;\|\mathbf{z} - \mathbf{c}_j\|, \;\forall j\neq k \right\}
\end{equation}

这样每个质心对应一个多边形区域 $V_k$。为适应实际城市形状,本研究将 $V_k$ 与武汉市行政边界、主要河流(如长江、汉江)、道路网缓冲区等进行相交运算,最终形成路区 $\hat{Z}_k$。

在操作层面,需要借助地理信息系统(GIS)将聚类中心投影到地图上,并对 Voronoi 多边形与武汉市行政区多边形做 intersection(求交)或 difference(求差)运算,以排除无效区域(如江河水域或不通车区域),保证路区分割的可行性\cite{4}。对于跨江大跨度路区,若中心无法在短半径内服务,可额外细化切分。

路区划分在实际应用中具有广泛的场景:

\begin{itemize}
\item 即时配送平台:外卖、美团等平台可基于订单坐标划分多个调度片区,区域内商户与骑手更容易快速接单、派单。

\item 同城快递:以多个路区划分派送范围,减少跨区调度带来的装卸切换或中转环节。

\item 生鲜电商:在高密度商业区配置更高频配送车辆,在远郊区或农产品基地附近则单独划分大区域配备大车,形成差异化运营。
\end{itemize}

\subsection{仓库选址问题定义及服务范围}

仓库选址(Facility Location)是决定在若干候选地点中建立或启用哪些仓库,以及如何将需求区分配给这些仓库,从而在保证服务水平的前提下最小化总成本 \cite{Daskin2013}。对于武汉市这样的超大城市而言,仓库可能是“前置仓”、“区域仓”或“中心仓”,其区位布局将直接影响车辆起终点的选定和行驶距离。此外,不同仓库在用地租金、建仓成本、容量限制等方面差别较大,需要通过优化模型做综合考量。

通常情况下,城市配送中选址决策与路区划分紧密结合:当城市被划分成若干路区后,每个区可视作一个“需求块”,再由对应的仓库为其提供支持,也可在大区或子区层次做分配。比如,在汉口、武昌、汉阳这三大板块各自设立1~2个主仓或前置仓,以覆盖各自片区需求,这种设计可明显降低跨江运输与长距离调度。

令候选仓库点集 $\mathcal{I} = \{1, 2, \dots, I\}$,其中每个点 $i$ 具备固定成本 $f_i$ 与容量 $\text{Cap}_i$。若路区数目为 $K$,路区 $k$ 的需求量记作 $D_k$。定义二元决策变量:
\[
x_i = 
\begin{cases} 
1, & \text{在候选点 $i$处建设/启用仓库}, \\
0, & \text{否则},
\end{cases} \quad
y_{i,k} = 
\begin{cases} 
1, & \text{若路区 $k$由仓库 $i$服务}, \\
0, & \text{否则}.
\end{cases}
\]

若运输费用或距离记为 $\mathrm{cost}(i,k)$(可等于单位距离 $\times$ 路区总需求量 $D_k$),则选址目标函数可写为:
\[
\min \quad \sum_{i \in \mathcal{I}} f_i \, x_i + \sum_{i \in \mathcal{I}}\sum_{k=1}^{K} \mathrm{cost}(i,k) \, y_{i,k},
\]
subject to
\[
\sum_{i \in \mathcal{I}} y_{i,k} = 1, \quad \forall k = 1,\dots,K,
\]
\[
y_{i,k} \le x_i, \quad \forall i, k,
\]
\[
\sum_{k=1}^K D_k \, y_{i,k} \le \mathrm{Cap}_i \, x_i, \quad \forall i,
\]
\[
x_i \in \{0,1\}, \quad y_{i,k} \in \{0,1\}.
\]
这里约束 $\sum_{i \in \mathcal{I}} y_{i,k} = 1$ 表示每个路区 $k$ 必须且只能由一个仓库服务;$y_{i,k} \le x_i$ 确保若仓库未启用 ($x_i=0$),则无法覆盖任何路区;而容量限制 $\sum_{k=1}^K D_k \, y_{i,k} \le \mathrm{Cap}_i \, x_i$ 避免仓库超负荷。


一旦某路区 $k$ 被指定给仓库 $i$,则该仓库即成为配送车辆的出发和返回点,用于满足该路区的所有订单需求。在实际应用中,有时也可允许路区同时由多个仓库部分覆盖(如相邻区或公共承运人),但本研究基于单一覆盖假设,以简化问题。若需要探讨“部分覆盖”或“选择性服务”,则需对上述约束和目标函数做相应修改,引入遗漏惩罚项或外包费用 \cite{Shen2007}。

对于武汉市而言,常见服务范围场景包括:
\begin{itemize}
    \item 主城区密集区域:可布置小型前置仓,减少城市内高频短途运输;
    \item 三环外新城:若需求量较大,企业可建中型仓库以统筹周边区域;
    \item 远郊及跨江片区:采用少量仓库或外包承运人,以避免过高固定成本。
\end{itemize}

\subsection{主要约束条件与假设}
为使路区划分与仓库选址模型更好地贴合实际,同时保证研究聚焦度,本论文在武汉市配送场景下做出以下约束条件与假设:
\begin{enumerate}
    \item \textbf{完整覆盖假设:} 缺省情况下,研究假定所有路区均必须被分配到某个仓库 ($y_{i,k} \in \{0,1\}$ 且 $\sum_i y_{i,k} = 1$),不允许出现未覆盖区。倘若企业允许远郊订单外包或惩罚性放弃,可在后续扩展中(如第6章展望)考虑“部分覆盖”场景。
    \item \textbf{单级仓库结构:} 假设只有一级仓库/前置仓,不存在多级分拨中心(如省中心仓→市级分仓→末端站点的三层结构)。虽然多级模型在学术上也颇为重要,但超出本研究主要范围。
    \item \textbf{容量限制:} 若仓库 $i$ 具备最大可处理量 $\mathrm{Cap}_i$,则路区总需求不能超限。对于容量很大的中型或大型仓库,可视作 $\mathrm{Cap}_i$ 接近无穷大,从而不实际约束。
    \item \textbf{需求静态与周期集中:} 研究暂设定在某固定周期内(如每日或每两日),路区需求量 $D_k$ 不随时间变化;动态或实时需求的情形将留待后文展望或在VRP部分(第3章及第4章)考虑局部随机扰动。
    \item \textbf{地理数据可得性与坐标准确:} 假设已通过 GIS 或政府公开数据获取了较准确的商户坐标、武汉市行政边界与道路网络。对江河、湖泊及封闭式道路做了前期剔除或限制,不在这部分模型中重复处理。
    \item \textbf{固定建设费用:} 仓库选址模型中的固定费用 $f_i$ 或租金在研究期内视为常量,不考虑短期内租金或地价的动态波动;如需探讨多阶段或长期选址策略,可再引入贴现或阶段性成本。
\end{enumerate}

综上所述,本研究将城市配送场景下的路区划分与仓库选址问题加以严格定义:利用KMeans与Voronoi剖分来完成对武汉市内需求点的区域划分,使用固定费用选址模型决定各区域是否设仓及如何覆盖服务。上述组合在城市级物流网络中能有效减少运输距离与重复调度,既有助于后续车辆路径规划(第3章)的解耦,也为多车型配置与自适应算法(第4~5章)提供了清晰的上层规划框架。在接下来的章节中,将进一步给出路区划分的具体实现(见 2.2)与仓库选址优化细节(见 2.3),并在 2.4 节给出实验验证结果。
参考文献

\section{路区划分模型}

在城市级别的物流规划中,路区划分(Zoning / Region Partitioning)是上层决策的重要组成部分。它通过将大规模、离散分布的需求点(商户或订单位置)划分为若干空间集中且相对独立的区域,进而便于在各个子区域内独立或并行开展仓库选址与车辆调度 \cite{Taniguchi2018}。在武汉市这样拥有广袤城区与多条水系的超大城市,传统的直接式聚类或人工划分往往难以兼顾地理边界、订单密度与可达性等多重因素。因此,本研究采用了"KMeans 聚类 + Voronoi 剖分 + GIS 裁剪"这一综合方法,既确保了对需求点的精确聚类,又能在城市边界和水域分隔的现实条件下生成实际可行的"多边形路区"。

\subsection{KMeans 聚类建模}

KMeans 算法是无监督聚类中的经典方法,旨在最小化簇内平方误差(Sum of Squared Errors, SSE),从而使分配到同一簇的点相互距离更近 \cite{Lloyd1982}。在城市配送应用中,KMeans 具有以下优点:

\begin{itemize}
    \item 执行效率高:对数万乃至数十万数量级的商户坐标,KMeans在合理的初始化与并行化(如使用HPC或GPU)下仍能在较短时间内完成收敛;
    \item 易于解释:每个簇对应一个"质心"(centroid),可视作该簇内需求的几何平均位置,便于管理者理解并做策略调整;
    \item 可与后续几何运算衔接:KMeans的质心输出可直接用于Voronoi剖分,形成区域划分的基础框架。
\end{itemize}

然而,KMeans也有局限,如对聚类数$K$较敏感、易陷局部最优等 \cite{Kaufman2009}。为此本研究在实际操作中,会多次随机初始化,并结合多种$K$值做试验,选取在分区管理可行性与计算效率间平衡的方案。

令$\{\mathbf{x}_1,\mathbf{x}_2,\dots,\mathbf{x}_N\}\subset \mathbb{R}^2$表示$N$个需求点在投影坐标系下的位置。欲将其划分为$K$个簇时,KMeans目标函数可写作 \cite{Bishop2006}:

\begin{equation}
\min_{\{\mathbf{c}_1,\dots,\mathbf{c}_K\},\,\{\mathrm{cluster}(i)\}} \;\; \sum_{i=1}^{N} \left\| \mathbf{x}_i - \mathbf{c}_{\,\mathrm{cluster}(i)}\right\|^2,
\end{equation}

其中$\mathbf{c}_k$是第$k$个聚类中心(或质心),$\mathrm{cluster}(i)\in \{1,\dots,K\}$表示第$i$个点所属簇号。KMeans算法通常通过以下迭代完成 \cite{Arthur2007}:

\begin{enumerate}
    \item 初始化:随机从$\{\mathbf{x}_i\}$中选$K$个点作为初始质心或采用"k-means++"策略;
    \item 分配:将每个$\mathbf{x}_i$分配给与其距离最近的质心$\mathbf{c}_k$;
    \item 更新:对每一簇,重新计算所有点的坐标均值作为新质心;
    \item 迭代:重复分配与更新,直到质心位置变动量小于设定阈值或达到最大迭代次数。
\end{enumerate}

\subsection{Voronoi 剖分与地理边界裁剪}

当获得KMeans质心后,若仅以"簇"概念组织需求点,仍缺乏清晰的空间边界。Voronoi图(Voronoi Diagram)恰好能提供基于最近邻原则的区域划分 \cite{Aurenhammer1991}:对于质心$\mathbf{c}_k$,其Voronoi区域$V_k$包含了在平面上距离$\mathbf{c}_k$最近的所有点。形式化定义:

\begin{equation}
V_k = \bigl\{\mathbf{z} \in \mathbb{R}^2\;\mid\; \|\mathbf{z}-\mathbf{c}_k\|\;\le\;\|\mathbf{z}-\mathbf{c}_j\|\,, \forall j \neq k \bigr\}.
\end{equation}

Voronoi剖分具备如下优势:
\begin{itemize}
    \item 完备且无重叠:$\bigcup_{k=1}^K V_k = \mathbb{R}^2$(在理想条件下),不同$V_k$之间不交叠;
    \item 与质心间最近邻关系:每个区域内所有点对同一质心的距离不大于对其他质心的距离;
    \item 可直接进行多边形裁剪:各$V_k$通常是(半)无界多边形,但可在城市范围内进行地理限制剪切。
\end{itemize}

\subsection{GIS平台处理与空间数据可视化}

为处理大规模坐标与地理要素,本研究在Python环境下集成了GeoPandas、Shapely、pyproj等主流库,同时配合QGIS或ArcGIS在可视化和高级矢量操作上的功能 \cite{Rey2007}。整体流程包括:

\begin{enumerate}
    \item 数据预处理:读取原始商户/订单CSV,应用投影转换;
    \item KMeans聚类:利用sklearn.cluster或自写并行化算法完成;
    \item Voronoi生成:调用scipy.spatial或shapely.voronoi获取几何对象;
    \item 城市边界及水域叠加:通过intersection/difference操作与武汉市shp文件做叠加运算;
    \item 形状修复与属性添加:计算面积、周长、订单密度等指标;
    \item 导出:将结果另存为shapefile或GeoPackage。
\end{enumerate}

在本研究框架下,路区划分结果将直接影响仓库选址与后续车辆调度:
\begin{itemize}
    \item 仓库选址:每个区域作为"需求单元"用于选址决策;
    \item 车辆调度:结合本区仓库进行VRP求解,减小问题规模;
    \item 可扩展性:支持分布式并行求解与局部协同。
\end{itemize}
\section{仓库选址优化}
\subsection{固定费用选址模型}

在多仓库选址问题中,\textbf{容量约束}和\textbf{服务覆盖}是两个需要重点考虑的关键因素。一方面,每个仓库都有其最大服务能力限制;另一方面,仓库需要在合理的服务半径内覆盖尽可能多的需求点。本研究在传统选址模型的基础上,进一步引入\textbf{多级容量约束}和\textbf{部分覆盖惩罚}机制,以构建更贴近实际应用场景的选址优化模型。

在实际运营中,仓库往往具有多个可选的容量等级,不同容量等级对应不同的建设和运营成本。为此,引入多级容量约束机制如下所示。假设我们设定了 $L$ 个容量等级,分别用
\[
\{(Q_1, C_1), (Q_2, C_2), \ldots, (Q_L, C_L)\}
\]
表示,其中:
\begin{itemize}
  \item $Q_l$ 表示第 $l$ 级容量的上限;
  \item $C_l$ 表示选取该容量等级所需的固定成本。
\end{itemize}

对于某个路区(或区域) $k$ 的总需求量 $D_k$,需要选择一个合适的容量等级 $l_k^*$ 来满足该区域的需求,即:
\[
D_k \leq Q_{l_k^*},
\]
并且为了保证所选容量等级的经济性,需满足
\[
l_k^* = \arg\min_{l}\Bigl\{C_l \,\Big\vert\, D_k \leq Q_{l}\Bigr\}.
\]
通过这种多级容量设计,模型能够根据不同地区的实际需求量灵活选择最合适、最经济的仓库规模。

\subsection{部分覆盖惩罚机制}
为了平衡服务质量和运营成本,可以在选址模型中加入基于覆盖半径的服务范围约束。具体而言,设定一个基准服务半径 $R$,当某个仓库 $i$ 与需求点 $j$ 的距离 $d_{ij}$ 超过 $R$ 时,将产生额外的惩罚成本。惩罚成本随超出覆盖半径的距离而增加,可定义为:
\[
p_{ij} = 
\begin{cases}
0, & \text{if}\ d_{ij} \le R,\\[6pt]
\alpha\,(d_{ij} - R)\,q_j, & \text{if}\ d_{ij} > R,
\end{cases}
\]
其中,$\alpha$ 为惩罚系数,$q_j$ 为需求点 $j$ 的需求量。此机制允许在必要时突破覆盖半径的限制,但会因此付出相应的成本代价。

结合上述多级容量约束和部分覆盖惩罚机制,可以将路区 $k$ 的仓库选址问题表述为以下优化模型(示意):

\begin{equation}
\min \sum_{i \in I_k} \Bigl(f_i \, x_i \;+\; C_{l_k^*}\, x_i \;+\; \sum_{j \in J_k} \bigl(c_{ij}\, q_j \;+\; p_{ij}\bigr)\, y_{ij}\Bigr)
\label{eq:objective}
\end{equation}

\noindent
\textbf{约束条件}:
\begin{align}
& \sum_{i \in I_k} x_i \;=\; 1, 
&& \text{(每个路区只选择一个仓库)} \label{eq:con1}\\[6pt]
& \sum_{i \in I_k} y_{ij} \;=\; 1, \quad \forall j \in J_k,
&& \text{(每个需求点必须被某个仓库服务)} \label{eq:con2}\\[6pt]
& y_{ij} \;\le\; x_i, \quad \forall i \in I_k,\, j \in J_k,
&& \text{(只有被选中的仓库才能提供服务)} \label{eq:con3}\\[6pt]
& \sum_{j \in J_k} q_j \, y_{ij} \;\le\; Q_{l_k^*}\, x_i, \quad \forall i \in I_k,
&& \text{(容量限制)} \label{eq:con4}
\end{align}

\noindent 其中:
\begin{itemize}
  \item $I_k$ 为路区(区域) $k$ 中的候选仓库集合;
  \item $J_k$ 为路区(区域) $k$ 中的需求点集合;
  \item $f_i$ 为仓库 $i$ 的基础固定成本;
  \item $C_{l_k^*}$ 为选定容量等级 $l_k^*$ 所需的固定成本;
  \item $c_{ij}$ 为单位运输成本;
  \item $p_{ij}$ 为超出覆盖半径产生的惩罚成本;
  \item $x_i$ 为 0–1 决策变量,表示是否选择仓库 $i$;
  \item $y_{ij}$ 为 0–1 决策变量,表示需求点 $j$ 是否由仓库 $i$ 服务。
\end{itemize}

上述模型将多级容量约束和部分覆盖惩罚机制结合起来,既能保证对不同需求规模的灵活适配,又能通过覆盖半径与惩罚系数的调节来平衡服务水平与运营成本,为决策者提供更具弹性和现实意义的仓库选址方案。


\chapter{车辆路径规划算法研究}

在第2章中,我们通过 KMeans 聚类、Voronoi 剖分与 GIS 裁剪的方法,将城市配送范围划分为若干“路区”,并在此基础上进行了仓库选址的上层规划,确定了哪些路区启用仓库、各路区由哪些仓库覆盖。到此为止,城市级的网络骨干已基本搭建完毕。然而,在每一个“仓库-路区”对内,还需进一步制定具体的车辆路径——即何种车辆、以何条路线和顺序,来完成对该路区内所有订单或商户需求的高效服务;这就进入到\textbf{车辆路径规划(VRP)}的研究范畴。

与一般的 VRP 相比,本研究强调“路区视角”对 VRP 的影响:既可将大规模城市订单分割成若干规模更可控的子问题(每个路区 + 对应仓库),也可让后续在打车、外卖、物流等多领域应用中有更强的灵活性。特别是当路区需求量或订单分布发生变化时,我们能够调整区块边界或仓库覆盖关系,从而在下层 VRP 层面也进行“部分更新”或“局部调度”,而非全城范围大规模重算。这种“分区分治 + 局部可调”机制,正是前文路区划分研究对 VRP 的最大贡献之一。

本章将围绕车辆路径规划的基本模型、关键约束、目标函数以及多车型/多配置场景展开探讨,并阐述如何与路区划分有机结合,为后文(第4章)自适应算法设计与多阶段实验打下坚实基础。

\section{VRP 基本模型扩展}

\subsection{VRP 的起源与闭环配送理念}
车辆路径规划(VRP)是运筹学与交通物流领域的经典问题,由 Dantzig 和 Ramser 在 1959 年首先提出,用于研究卡车在给定 depot(仓库)与一批客户点之间的最佳调度路线 \cite{dantzig1959truck}。在最基本的 VRP 场景下,车辆必须从仓库出发、访问所有客户或需求点,然后返回仓库,形成一条闭合路径(Closed Loop)——这在学理上常称之为\textbf{“闭环配送”}(closed-loop distribution),强调车辆的出发与终点一致,且无中途弃车或跨仓转移 \cite{crainic2010fleet}。

在城市配送中,闭环配送模式确保了企业车队的管理可控,也方便车辆在同一仓库进行装卸、加油(或充电)等操作。但随着业务规模的扩大与多仓策略的出现,可能出现更复杂的场景——例如跨仓库的车辆调动与路区切换。然而,本研究下层 VRP 的主流假设仍是每辆车固定隶属某个仓库,沿闭环方式服务所在的路区,然后回到初始仓库。这一策略在打车、外卖、同城物流等行业皆常见,如外卖骑手通常在派单范围内来回接单,或快递员在所在网点重复收派。

传统文献中,为简化计算,有时令
\[
d_{i,j} = \|\mathbf{x}_i - \mathbf{x}_j\|
\]
但在现实城市配送里,如武汉市高架、立交桥、过江隧道等,使得实际路网距离与欧式距离偏差明显。为更符合真实交通情况,本研究特别关注路网距离(或路网最短路径)作为 VRP 的距离/成本矩阵 \cite{laporte2009fifty}。

在工程上,可利用以下手段获取路网距离:
\begin{enumerate}
    \item \textbf{GIS 路网数据 + 最短路算法}:如 Dijkstra 或 Floyd-Warshall,对城市道路图做全对全最短路计算,得到 $\mathrm{dist}(i,j)$;
    \item \textbf{API 调用}:调用高德、百度或 OpenStreetMap 路径服务 API,大规模抓取两两节点间的实际驾车/骑行距离;
    \item \textbf{近似分区}:若路区规模大或节点过多,亦可先在区块内做局部最短路图,以减少计算量。
\end{enumerate}

这样得到的 $d_{i,j}$ 能更真实地反映车辆绕行、单行道、无法跨越江河等情境,也更加贴合城市配送对于通行限制、交通拥堵的实际。后续在 VRP 模型或算法实现时,会基于此距离矩阵来评价路线的总行驶里程或用时,从而提高调度方案的可落地性。


与一般 VRP 最大区别在于:本研究已先行做好路区划分与仓库选址。这意味着:
\begin{enumerate}
    \item \textbf{每个路区内部}:基于闭环原则,车辆从其绑定仓库(或网点)出发,访问区内需求点,最终回到同一仓库;
    \item \textbf{仓库到仓库配送}:若路区划分或上层选址策略允许部分仓之间存在调拨(例如仓 A 给仓 B 补货),则形成“跨仓车辆路由”问题,也是闭环的一种延伸(车从仓 A 出发,经必要的路区或高速线路,送达仓 B 并返回 A 或留在 B),需要在后续模型中做特殊处理;
    \item \textbf{实际路网}:通过对武汉市(或其他城市)交通地图数据的解析,不再使用简化欧氏距离,而采用更符合城市交通现状的最短路网距离做车辆行驶成本衡量,适合打车、外卖、物流多业务形态。
\end{enumerate}

\subsection{约束条件:容量/“区内行驶限制”/部分覆盖}

本研究下层 VRP 除了基本容量限制外,还设定了一条重要要求:车辆行驶路段必须在对应路区范围内,不得驶出路区,以确保路区调度的独立性与分区管理模式的贯彻。下文将对这条新的\textbf{“区内行驶限制”}以及其他约束做详细说明。

容量约束是城市配送最常见的需求之一,指车辆的最大可承载量(重量、体积或订单数)不能被超出 \cite{baldacci2009unified}。对于电动车、三轮车、小面包车或4.2米轻卡,其可承载量各不相同,需要在调度时分配合适车型或使用多辆车分别跑不同节点。若路区订单量过大,还可能采取多波次配送或多车并行。相比无容量限制,容量引入后对路线设计提出更复杂的拆分、合并或优先级分配问题,也增强了调度方案的现实可行性。

在数学模型中通常用
\[
\sum_{j\in R_\ell} q_j \le Q_v
\]
(路线 $\ell$ 的需求量合计不超过车辆 $v$ 的容量 $Q_v$)表示 \cite{archetti2012vehicle}。路区划分在此能有效减少“极端”订单量集中的情况,并配合上层仓库容量决定每区可分配的车辆数量与类型。

在本论文的核心假设下,每个路区内的路线必须封闭于本区边界之内:车辆不可跨越路区边界去服务其他片区节点,也不允许驶出路区后再进来。这一限制可以视为对打车/外卖平台或同城物流管理的组织性要求:
\begin{enumerate}
    \item 平台派单时:只在同一区域或部分子区域分配骑手/车辆,减少跨区调度导致的调度复杂度;
    \item 行政区 / 网格化管理:一些城市对网格化管理有硬性规定,期望在区与区之间有独立的资源调度模式以防止混乱。
\end{enumerate}

代替常见的时间窗约束,这里更着重地理封闭性:一旦路区划分完毕,所有节点(商户)与车辆活动都严格局限在该多边形区块内。若某需求临近边界而跨区服务更优,也需要在上层仓库选址或路区分割阶段进行调整,本质上将复杂的跨区路线需求尽量前置化解决。

注意:对于超大或繁杂的路区,若车队确有需要暂时绕行出区的道路,也必须在上层对路区边界进行修正或引入特别通行通道。否则按标准规定,车辆不得离开区块以走捷径——这类情况在外卖、打车平台通常通过严格的“派单半径”或“营业范围”来约束。

虽然本研究核心倾向完全覆盖(将路区内所有有效节点都要服务),但仍保留部分覆盖功能以应对极端情况 \cite{salazar2013multi}。具体来说:
\begin{itemize}
    \item 节点极度偏远:若一处节点离仓库数十公里外却需求频率极低,企业可能出于成本考虑,在调度时放弃该节点或外包给第三方;
    \item 特殊运力限制:车辆类型无法满足该节点的重量或时段需求,也可能通过违约或外包方式拒绝接单。
\end{itemize}

在模型上,这可通过为每个节点设一个惩罚 $\mathrm{penalty}_j$,若不服务则支付该惩罚,同时无需行驶距离 \cite{salazaraguilar2013multi}。若道路、容量与成本综合判断后发现放弃某节点更划算,算法可自动选择部分覆盖。而在路区划分层面,若多数节点都偏远,则在上层选址阶段或路区规划中已可能排除/合并该片区。

上述“区内行驶限制”主要约束普通配送车辆不得超越其分配路区。然而,在某些场景下,不同路区的仓库之间或分拨中心之间也可能需要货物调度或调拨(例如仓 A 补货给仓 B)——可称之为仓库到仓库(Inter-warehouse)配送 \cite{cordeau1997tabu}。这通常以中长距离为主,大概率经过城市干线或高速,并可能不在普通配送车辆管理的范围,而是由专门的中转车队承担。

因此,在路区模式下,普通配送车严格局限于所属路区;但仓间补货车可以在特定主干道连接的跨区路段行驶,形成“一条大走廊”或“重点桥梁干线”互联多个仓库。二者可通过多层模型区分:
\begin{itemize}
    \item 下层:区内车辆做商户投递;
    \item 中层:仓库-仓库配送线路。
\end{itemize}

当仓 A 与仓 B 同时需要小规模交换货物时,这部分调拨路线可视为“另一个小型 VRP”或“单线区间线路”,在运作上与普通区内配送并不冲突,只是由不同车队或在非市内高峰时段处理。这样既符合路区封闭性,也满足仓间分拨需求。

在本研究构建的城市配送 VRP 中,除传统容量限制外,我们引入了路区内部行驶限制,替代常见的时间窗需求,以突出“分区分治”的地理封闭管理思路。此外,部分覆盖选项为极端节点或超远订单提供弹性处理;而仓库-仓库配送则作为一个额外的跨区通道,以满足分拨或补货需要,却并不破坏区内车辆的行驶限制。这样一来,整个下层 VRP 更贴合城市多业务的管控原则,也能与上层路区划分、仓库选址形成良好互补。后续将在 3.1.3 继续探讨 VRP 的目标函数设计——如何平衡行驶距离、固定成本、多车型配置与其他指标,并在自适应算法(第4章)中深入说明如何在分区模式下灵活求解这些定制化 VRP 问题。

\section{目标函数设计(距离成本、固定成本等)}

在第 3.1.1、3.1.2 小节中,我们已讨论了 VRP 的主要决策变量与多种约束(容量、“区内行驶限制”、部分覆盖等)。然而,为了在解空间中找到最优或满意的调度方案,还需要在目标函数上明确衡量标准。传统 VRP 通常以行驶距离最小化作为单一目标,但在城市配送中,企业可能还需顾及车辆固定成本、运营时效以及分区管理等多重要素。本节将介绍如何在下层 VRP 中结合这些成本或指标,以构建更贴近实际业务需求的目标函数。

\subsection{行驶距离或里程费用}
最小化行驶距离是 VRP 研究的最早也是最常见目标形式。如果车辆数或车型数不固定,最小化距离往往也会带来车辆使用量的间接节省 \cite{toth2014vehicle}。在本研究中,对于单一“仓库-路区”子问题,所有在其区域内的车辆路线之和的距离可以定义为:
\[
\text{Dist} = \sum_{\ell\in\mathcal{R}} d(R_\ell),
\]
其中 $d(R_\ell)$ 表示路线 $\ell$ 的实际路网行驶距离(参见 3.1.1 的“实际路网距离”讨论),$\mathcal{R}$ 为该子问题内所有确定采纳的路线集合。若只考虑这个目标,则可直接使用 CVRP 或 MDVRP 相关算法做求解。

\subsection{车辆固定成本}
在“打车、外卖、物流”多业务场景下,车辆并非无限;在城市配送中,往往每辆车都可能拥有一定的固定使用成本(Fixed Cost),如车辆租赁费、工资或保险费等。如果企业要在某路区内临时增派车辆,就会额外产生一定的固定支出 \cite{salazaraguilar2013multi}。因此,可在目标函数中叠加一项
\[
\sum_{\ell \in \mathcal{R}} c^\mathrm{fix}_{\ell},
\]
代表在该子问题内使用了多少辆车或多少班次并支付相应的固定费用。通常,我们可把每条路线 $\ell$ 视作独立调用一辆车资源。

\textbf{意义}:
\begin{itemize}
    \item 若车辆固定费不小,算法可能宁可让少量车辆多跑一点距离,也不要启用过多车辆;
    \item 若固定费很低或已有车空闲,则算法倾向于多用车辆、减少单车里程。
\end{itemize}

在路区划分后,固定成本在下层 VRP 中可根据局部车辆需求(峰值或均值)来配置;有时上层会先选定哪类车型在此路区驻守,下层只做调度,这就体现了“二阶段”或“多层次”运筹思路。

\subsection{多目标综合或加权}
某些城市配送应用需要同时考虑“距离 + 时间 + 碳排放 + 服务水平”等多维目标,往往采用加权求和或分层目标的方法 \cite{toth2014vehicle}。比如:
\[
\min \quad \alpha \times \text{Dist} + \beta \times \text{FixCost} + \gamma \times \text{LatePenalty},
\]
其中 $\alpha, \beta, \gamma$ 为权重系数,LatePenalty 代表延迟或违约惩罚等。本研究暂以“距离 + 固定费用”为核心目标,若后续要增加部分覆盖、车辆排班等额外惩罚项,也可纳入多目标框架。路区划分后每个子问题规模较小,进行多目标或加权目标的求解也更易实操。

\subsection{基于路线优先级的调度}
打车平台或外卖平台时,目标可能不只关心总里程,还关心“乘客等待时间最短”或“订单超时最少”。这些可通过优先级或超时惩罚间接融入目标函数 \cite{salazaraguilar2013multi}。在路区内,若某些节点为“VIP商户”或“高频乘客”,可给予额外负权值或极高超时惩罚,使算法在路线设计中优先满足。由于本研究主旨在强调路区管理+多仓,在目标函数层面主要关注“综合成本”,包括里程费、固定费、或部分覆盖惩罚;但若有更精细的优先级需求,也可以在路区层次做进一步扩展和落地。

\section{多车型 / 多配置初步说明}

在城市配送或网约车、外卖场景中,往往有多种车型可选:如电动车、面包车、厢式货车、甚至中型卡车或 SUV 等,不同车型在载重量、燃油(或电力)消耗、通行范围、固定租金等方面差异较大 \cite{baldacci2009unified}。此时 VRP 就从单一容量扩展至\textbf{“多车型 VRP(HFVRP)”}或“多配置”场景,需要在调度时决定哪辆车或哪类车型执行本路区配送。

\subsection{多车型 VRP 概述}
多车型 VRP(Heterogeneous Fleet VRP, HFVRP)要求在常规 VRP 约束下,每次调度路线还需指定一个车辆类型 $\alpha$,其容量 $Q_\alpha$、固定费用 $C_\alpha$、里程成本函数 $d_{i,j}^\alpha$ 可能有所差别 \cite{laporte2009fifty}。在“外卖骑手 vs. 面包车 vs. 4.2米货车”等多种选择时,算法需平衡:
\begin{itemize}
    \item 大车型能一次性承运更多需求,但进巷道或街区不便;
    \item 小车型灵活度高,但运能有限,需多次来回或多车并行;
    \item 若某车型需专属资质或更高租金,也要在目标函数中纳入相应费用。
\end{itemize}

在路区划分下,每个区块都可独立决定启用哪些车型,或允许多种车型并存。对打车平台而言,这相当于在同一区内投放不同档次车辆(豪华、舒适、拼车等);对城市物流而言,则是在同一区调配电动三轮车、面包车、厢式卡车等组合。这样一来,不仅能提升运力利用效率,也能针对不同街区道路条件进行最优匹配。

\subsection{多配置与灵活组合}
在更普遍的城市运营中,不仅车型容量 $\alpha$ 不同,还可能针对每辆车有不同“配置”:包括可用车辆数上限、日租价、甚至车辆加装冷链设备等 \cite{salazaraguilar2013multi}。例如:
\begin{itemize}
    \item 小车型:容量较小 ($Q_{\text{small}}$)、固定费低;
    \item 中车型:中等容量与中等固定费;
    \item 大车型:高容量高费用;
    \item 冷链车型:可运鲜食品,但租金或油耗更贵。
\end{itemize}

多配置离散搜索是指在算法中,对若干可能配置(如 $(\text{capacity}, \text{fixed cost}, \text{count})$)进行遍历或组合,以找出在该路区调度中最具性价比的车型搭配 \cite{salazar2013multi}。由于路区的订单量在一定时段内相对稳定或可预测,管理者可在开工前选择“今天在区 X 投放 2 台中型车 + 1 台小车”,在区 Y 投放“1 台大车 + 2 辆面包车”,形成差异化运营。

\subsection{与路区模式的结合与意义}
路区划分使得多车型/多配置策略更易实施,原因在于:
\begin{itemize}
    \item 需求集中:由于路区内部商户/订单性质更趋同,模型能更好匹配车型与需求特征(如某区多小巷商业,小车型更合适;某区是大片工业区或郊区,大车型更高效);
    \item 管理灵活:在一天或一周内,可以在某些路区增派大车,另一些路区减少车辆;而无须通盘对全城做车型配置优化,减少问题规模并提高响应速度;
    \item 打车/外卖中车辆种类并行:通过将平台上的不同档次车辆划分分区派单,如专车主要在高端商圈区,快车在普通城区,网约拼车在高校密集区等,各路区内相对明确车型选择和容量属性。
\end{itemize}

从运筹优化角度看,这能让二阶段(上层仓库选址 + 下层多车型 VRP)的效率更高:仓库选址阶段明确某路区配套仓库的容量或功能,下层 VRP 阶段再行决定具体车队组成。也可在路区层面使用启发式方法为多配置车辆做局部最优组合,再将结果合并成全城视角加以评估。

\subsection{打车、外卖等领域的衍生}
在打车平台:车型差异包括专车、快车、豪华车、拼车等;在外卖:电动车、摩托车、面包车等;在快递或干线物流:各类箱式货车乃至挂车。理论上,这些都可以统一到多车型 VRP 的框架之下——只要在“路区”级别的下层决策中,为每个可用车型标注其容量、固定费、里程费,再由求解算法选择最优调度方式 \cite{salazar2013multi}。这种思路是对“城市多运力协同分区管理”最直接的运筹学诠释。

\subsection{小结(3.1.3 \& 3.1.4)}
本节(3.1.3)首先阐明了 VRP 目标函数在城市配送中的多样形式:除了最常见的行驶距离,还应纳入车辆固定成本甚至惩罚性或优先级因子,以更贴近企业决策需求。接着(3.1.4)讨论了多车型/多配置场景下 VRP 的特殊性,强调了不同车型在容量、费用、道路适应性等方面的差异,并说明路区划分为多车型调度带来的便利与灵活性。企业可在各路区内根据订单特点和地形条件,选派合适的车型或配置,最大化运力利用效率。

通过这些要素的结合,城市级路区 + 多仓 + 多车型的下层 VRP 架构得以完整:一方面在上层选址阶段已决定哪些仓库在哪些区运营,另一方面在下层 VRP 中通过多车型与目标函数的设计实现对行驶距离、固定成本、服务质量等指标的全面考量。下一步(例如第4章)将着重于自适应算法与二阶段混合策略,探讨如何在如此多约束、多车型、多目标的复杂 VRP 中,用启发式或元启发式求解手段,快速逼近高质量解并在路区发生变动时灵活更新。

\section{基准算法设计}

\subsection{Clarke-Wright 算法}

在城市配送与车辆路径规划研究中,Clarke-Wright (CW) 算法是最早诞生并广为人知的快速启发式方法之一。自 1964 年 Clarke 和 Wright 提出该算法后,它一直被视为\textbf{“构造型”}基准的代表,常用来在极短时间内得到一条可行路线或初始解 \cite{clarke1964scheduling}。本研究在面临多仓多路区场景时,亦将 CW 用作基线或前端启发式,与更高阶算法(如禁忌搜索、VNS)组合成“二阶段”或“混合”求解策略。

CW 算法原本是为解决单仓库、统一车辆容量的配送问题提出的,核心思路是:先让每个订单点单独成一条“$0 \to j \to 0$”的回路,然后逐步合并这些小回路成更大的闭环路线,以减少总行驶距离 \cite{golden2008vehicle}。合并判定主要依赖一个\textbf{节省值(Saving)}指标——若将两个回路 $(0 \to i \to 0)$ 与 $(0 \to j \to 0)$ 合并成 $(0 \to i \to j \to 0)$ 能节约多少路程,就按照节省值从大到小排序,依次尝试合并,直到容量或其他条件不满足为止。

下式展示了合并两段回路时的节省值 $S(i,j)$ 计算方法(最经典版本):
\[
S(i,j) = d_{0,i} + d_{0,j} - d_{i,j},
\]
其中 $d_{0,i}$ 为仓库与点 $i$ 的距离,$d_{i,j}$ 为节点 $i$ 与 $j$ 的距离 \cite{laporte2009fifty}。若 $S(i,j)$ 值大,说明将 $i,j$ 放在同一条线路紧密连接,可以显著降低总里程,因此优先进行合并。这个单纯且高效的合并策略使得 CW 在大部分小中规模问题上能迅速给出较优解,也是后来许多增强算法的基础。

以单仓、容量一致的“距离最小化”CVRP 为例,CW 的构造过程通常可用以下简化伪码表述:
\begin{enumerate}
    \item 初始化:将每个需求点 $j \neq 0$ 建立一条独立路线 $(0 \to j \to 0)$。此时共有 $N$ 条回路;
    \item 计算节省值:对任意 $i \neq j \neq 0$,令
    \[
    S(i,j) = d_{0,i} + d_{0,j} - d_{i,j}.
    \]
    将所有 $(i,j)$ 的 $S(i,j)$ 从大到小排序;
    \item 合并尝试:按照节省值顺序遍历,若将 $(0 \to i \to 0)$ 和 $(0 \to j \to 0)$ 合并成 $(0 \to i \to j \to 0)$ 不违反容量、节点未被多次合并等条件,即执行合并;
    \item 容量检查:若合并后的路线总需求 $\sum q_j$ 超过容量 $Q$,则跳过该合并;
    \item 终止:当遍历完所有 $(i,j)$ 或再无可合并机会后,输出现有路线集。
\end{enumerate}
完成后得到一组闭环路线。此过程复杂度基本为 $O(N^2)$(计算并排序节省值),实现很简捷,速度极快,因此常被用作初始解。

CW 算法常被用作“快速前端”或“基准对照”:
\begin{itemize}
    \item \textbf{快速前端}:先以极短时间生成一批可行路线,后续再用禁忌搜索、VNS 或自适应策略深度改进;
    \item \textbf{对照基准}:在大型算例中,与高级元启发式算法比较,CW 的结果往往在 $0.01 \sim 0.05$ 秒内即可完成,但解质量有时弱于精细算法。
\end{itemize}

\subsection{模拟退火(SA)算法}

在城市配送或打车平台等领域,模拟退火(Simulated Annealing, SA)是一种常见的元启发式算法,因其实现相对简单、对初始解依赖度较小、且具备一定跳出局部最优能力而广受关注。与 Clarke-Wright 等贪心式方法不同,SA 借助“温度”与随机扰动来在解空间进行广泛搜索,对复杂、多局部最优陷阱的车辆路径规划(VRP)问题常有较好效果 \cite{vidal2013hybrid}。
模拟退火起源于冶金学的退火过程:材料在高温下原子可随机运动以趋向更稳定结构;温度下降时,系统逐渐趋于有序。对应到组合优化上,SA 让解在“高温”阶段进行大范围随机移动,允许接收变差解;“低温”阶段则收窄扰动区间,趋向局部精细搜索 \cite{golden2008vehicle}。

\begin{enumerate}
    \item \textbf{温度(Temperature)}:从初始温度 $T_0$ 开始,每轮迭代后按某速率 $\alpha < 1$ 降温,如 $T \leftarrow \alpha T$。
    \item \textbf{邻域搜索}:从当前解 $\mathbf{S}$ 生成邻域解 $\mathbf{S}'$(例如交换两个客户次序、拆分或合并路线等 VRP 操作)。
    \item \textbf{蒙特卡罗接受准则}:
    记
    \[
    \Delta = \text{Cost}(\mathbf{S}') - \text{Cost}(\mathbf{S}).
    \]
    若 $\Delta < 0$,则无条件接受;若 $\Delta \geq 0$,则以概率
    \[
    \exp(-\Delta / T)
    \]
    接受。
    \item \textbf{迭代降温}:随着温度 $T$ 不断降低,算法越来越倾向拒绝变差解,最后停在某个局部或全局最优。
\end{enumerate}

在本研究中,每个“仓库 + 路区”形成一个规模相对可控的 VRP 子问题,可用 SA 独立搜索。由于 SA 不需要像 Clarke-Wright 那样依赖贪心“节省值”合并,且不像禁忌搜索/VNS 那般需额外存储禁忌表或多邻域切换,是比较轻量化的随机搜索方式 \cite{vidal2013hybrid}。

参数设置与降温策略如下:
\begin{itemize}
    \item \textbf{初始温度} $T_0$:常见做法是选择使得“以 $\Delta_{\max}$ 为代价变差解时,初始接受率约 $0.8 \sim 0.9$”。
    \item \textbf{降温速率} $\alpha$:典型区间在 $[0.90, 0.99]$;当 $\alpha$ 高时,降温较慢,算法搜索范围更大但需更长时间;当 $\alpha$ 低时,搜索快但易早停。
    \item \textbf{终止条件}:最大迭代次数或连续若干轮无改善。
\end{itemize}

本章后续将结合实验数据对 SA 进行优化,并探讨如何与禁忌搜索 (TS) 或变邻域搜索 (VNS) 结合,进一步提升搜索效率和解质量。

\subsection{禁忌搜索(TS)算法}

禁忌搜索(Tabu Search, TS)是一种经典的元启发式算法,由 Glover 在 1986 年前后提出,并在 1989 年的系列文章中系统化 \cite{glover1986tabu}。其核心是结合局部搜索与**记忆(Tabu List)机制:在每一次迭代中,从当前解的邻域找到最优“局部改进”并进行更新,但同时记录和避免在短期内回到刚探索过的解或操作,防止陷入局部循环。经过若干代的迭代,“禁忌”表(Tabu List)与志愿准则(Aspiration Criterion)等策略可帮助算法跳出局部最优并保持搜索多样性 \cite{glover1989advances}。

在车辆路径规划(VRP)问题中,尤其是具有大规模或复杂约束的场景,禁忌搜索因其可控的记忆排斥与可扩展的邻域设计被广泛应用,被视为能在合理时间内取得高质量解的代表性方法之一 \cite{laporte2000tabu}。而在本研究中,对多仓多路区VRP亦可将TS 作为后端深入改进或混合搜索的关键工具,与 Clarke-Wright、模拟退火 (SA) 等形成“二阶段”或“多阶段”求解架构。
\begin{enumerate}
    \item \textbf{局部搜索与邻域}

    对VRP解的一次邻域搜索可表述为:在当前解 $\mathbf{S}$ 的邻域 $N(\mathbf{S})$ 中,选出目标函数 $\text{Cost}(\mathbf{S}')$ 最优的 $\mathbf{S}'$,并令下一代解为 $\mathbf{S}'$,但若该移动被禁忌表排斥则跳过,或若满足志愿准则则可破禁。形式化写作:
    \[
    \mathbf{S}^{(t+1)} = \underset{\mathbf{S}' \in \mathcal{N}(\mathbf{S}^{(t)}) \setminus \mathrm{TabuList}}{\arg\min}\; \mathrm{Cost}(\mathbf{S}').
    \]
    其中 $\mathrm{TabuList}$ 表示近期禁止操作或解的集合,保证在禁忌期限内不会回到刚走过的邻域状态 \cite{laporte2000tabu}。一旦迭代完成,更新 $\mathbf{S}^{(t+1)}$ 并适当修改禁忌表。

    \item \textbf{禁忌表与志愿准则}

    \begin{itemize}
        \item \textbf{禁忌表(Tabu List)}:可存储最近若干次“关键操作”(如在路线中交换两个节点的位置)或“解标识符”,期限可设为 5~15 轮等;
        \item \textbf{志愿准则(Aspiration Criterion)}:若某被禁忌的移动能使目标函数优于已知最优成本,则可临时破禁 \cite{glover1989tabu}。
    \end{itemize}
\end{enumerate}
    这些机制使 TS 在搜索过程中既能避免短期循环,又不忽略潜在的优质变动。

    在 VRP 下常见的邻域操作包括:
    \begin{itemize}
    \item \textbf{VRP邻域操作}
        \item Swap:交换同一路线或不同路线中两个节点的访问顺序;
        \item 2-Opt, 3-Opt:反转路段结构以去除不必要的交叉;
        \item Cross-exchange:从一条线路摘下一段到另一条线路,以提高容量利用;
        \item 合并/拆分:若两条小路线可合并且不超容量,则试着合并,或反向拆分一条大路线。
    \end{itemize}
    在多仓/多车型场景中,还可允许部分节点切换仓库或切换车型,但需防止违反**“路区不跨区”** 或车辆容量等约束 \cite{golden2008vehicle}。

\subsection{多仓多路区下的 TS 适配}

\begin{itemize}
    \item \textbf{路区子问题}:与 Clarke-Wright 或 SA 一样,在本研究中,每个“(仓库 $i$ + 路区 $k$)” VRP 可单独执行 TS 迭代。由于路区已在上层选址中绑定仓库 $i(k)$,下层 TS 不需考虑跨区或多仓转移——只有极特殊情形存在“多仓覆盖一片区”,才需要在邻域中允许部分节点“改投”另一个仓库 \cite{vidal2013hybrid}。
    \item \textbf{区内行驶限制}:参照 3.1.2,小节“区内不可出界”的限制在 TS 中意味着:
    \begin{itemize}
        \item 邻域操作仅在区内节点间进行;
        \item 若操作可能使路线越界或合并跨区节点,则为非法移动,不计入邻域。
    \end{itemize}
    这大幅简化搜索空间,也减少了容量或数据读写量。
    \item \textbf{容量/车型}:对 VRP 解中每条路线 $\ell$,若其车辆车型为 $\alpha$,则禁止任何邻域操作导致该路线需求超过 $Q_\alpha$。当允许车型切换时,需在邻域中定义“换车操作”,更改 $\alpha \to \beta$ 并校验容量与固定费用差异 \cite{toth2014vehicle}。
    \item \textbf{仓库-仓库配送}:如果研究需要仓间配送(如仓 A 给仓 B 补货),可视为另一特殊 VRP:“起点=仓 A, 终点=仓 B, 中途若干路段节点为转运站或高速”。此时 TS 的邻域设计仍类同,只是需允许始末仓不同 \cite{salazar2013multi}.
\end{itemize}
参数设置与实现细节如下:

在 TS 的实现中,以下参数尤其关键:
\begin{itemize}
    \item \textbf{禁忌期限 (Tabu Tenure)}:设置禁忌期限 $\tau \in [5,20]$ 或自适应变化。若 $\tau$ 太小,易回到刚走过解;若太大,限制搜索灵活度。
    \item \textbf{邻域大小}:选 2-Opt/3-Opt、Swap、Relocate 等多种操作组合,每轮迭代选最优移动。若邻域过大,单轮计算量可能上升;若邻域过小,搜索多样性不足。
    \item \textbf{停止准则}
    \begin{itemize}
        \item 迭代上限:如 500~3000 轮;
        \item 无改进计数达阈值:如 100 或 200;
        \item 时间限制:尤其在实时调度中常设秒级限制。
    \end{itemize}
\end{itemize}

在本研究中,这些算法的实验与表现为:
\begin{itemize}
    \item \textbf{对比 Clarke-Wright / SA}:在多数算例中(包括路区规模 200~1500 节点),TS 在 0.1~1 秒内完成数百迭代,能稳定优化结果。
    \item \textbf{二阶段混合}:CW 或 SA 可先快速生成初解,用 TS 做深度改进。常见实验显示,“CW/SA + TS” 方式可减少 30~50\% 的迭代时间,且最终解质量相当或更好。
    \item \textbf{多车型实测}:若某路区有大中小三种车型可选,TS 邻域操作里需加“车型切换”考虑。实践表明,允许车型灵活切换能提升解质量,但也会增大搜索空间。
    \item \textbf{线上动态调度}:若订单突然增加,先局部破坏当前解,再进入 TS 多邻域搜索以迅速修复——特别适合外卖/打车等场景。
\end{itemize}

优缺点有:
\textbf{优点}
\begin{itemize}
    \item 搜索深度:比 Clarke-Wright 等简单构造更能做细致局部优化,常取得更优解;
    \item 禁忌记忆:能避免短期循环,尤其在高维VRP中尤为重要;
    \item 可扩展:易于融合多车型、多仓、容量、甚至部分覆盖等多重约束,添加相应邻域判定。
\end{itemize}

\textbf{缺点}
\begin{itemize}
    \item 参数敏感:禁忌期限、邻域大小等需调试;
    \item 计算时间:在大规模路区,TS 单轮迭代若邻域庞大易耗时;
    \item 易早停:若缺乏适度的随机或志愿准则破禁,可能在复杂问题中陷入“次优”长时间不动。
\end{itemize}

\subsection{变邻域搜索(VNS)算法}

变邻域搜索(Variable Neighborhood Search, VNS)由 Mladenović 和 Hansen 于 1997 年正式提出 \cite{mladenovic1997variable},是一种通过周期性切换不同大小或类型邻域来摆脱局部最优的元启发式方法。与禁忌搜索(TS)等采用“记忆”概念不同,VNS 的核心理念是:在搜索出现停滞时,主动扩大或改变邻域范围(shake),从而进行大步跳跃;一旦找到更优解,再回到较小邻域的精细搜索(local search)。这一“循环加扰动”的框架使得 VNS 在许多组合优化

\subsection{其他高级算法简述(RL、GNN等)}

在前述小节里,我们讨论了以 Clarke-Wright、模拟退火、禁忌搜索、变邻域搜索为代表的经典或相对成熟的启发式与元启发式算法,它们在多仓多路区场景下对下层 VRP 求解都有较好的可行性和实用性。然而,近些年兴起的**强化学习(RL)与图神经网络(GNN)**等人工智能方法也不断涌现于车辆路径规划(VRP)领域,被视为在大规模、动态、实时调度情形下具有潜在突破的高级算法。下文对此类算法做简要评述,并说明它们在本研究中的初步适配思路以及与基准算法的关系。

强化学习(Reinforcement Learning, RL)在传统机器学习中通过“状态-动作-奖励”三元组来训练一个智能体,使其在与环境交互过程中累积经验并学会最优策略 \cite{rl_intro}。在 VRP 中,RL 算法可将整个路线构造或节点选择过程映射为一个序列决策问题:在每一步,算法决定下一个节点要拜访哪里、或是否结束路线,并从即时奖励(例如节省行驶距离、避免超时)中学习。

\begin{enumerate}
    \item \textbf{理论与思路}
    \begin{itemize}
        \item Markov 决策过程:可将 VRP 的节点集合视作状态空间的一部分,动作为“将下一节点插入路线”,奖励为负的增量成本等;
        \item Q-learning / Policy Gradient:学习一个策略网络,在每次节点选择时输出概率分布或 Q 值,对应不同可行动作的期望收益 \cite{q_learning,policy_gradient}。
    \end{itemize}
    \item \textbf{多仓多路区场景}
    若将城市分为若干路区,那么 RL 在下层 VRP 中可以被训练成一个学会为单一区块构造路线的策略网络 \cite{rl_application}。与前面元启发式的差异在于:
    \begin{itemize}
        \item RL 需要事先或在线收集大量案例,通过迭代训练来学习一般性策略;
        \item 一旦训练完成,在线推理速度可较快,但若路区需求量或布局大幅变化,可能需要重新训练或做自适应调整 \cite{rl_adaptation}。
    \end{itemize}
    对于打车平台或外卖平台,若订单分布日复一日相对稳定,RL 有机会在大规模模拟环境中“离线训练”出一套相对通用的调度策略;若每天变化过大,则 RL 的迁移或再训练成本较高。
    \item \textbf{优缺点与本研究适配}
    \begin{itemize}
        \item \textbf{优点}:在极大规模或动态需求场景下,RL 通过策略网络可实现快速决策;无须像 TS/VNS 那样迭代数百轮;也能在线更新某些参数;
        \item \textbf{缺点}:前期需要大量训练样本及算力;鲁棒性对环境变动敏感;对多仓多路区的硬性不可跨区限制、容量限制等,也需精心在状态/动作空间中编码。
    \end{itemize}
    在本研究目前的框架下,RL 还处于探索阶段,可视为后续高端方向:当城市路区需求每天都非常相似时,离线训练 RL 策略网络替代传统元启发式快速调度,能大幅降低在线计算负担。但要克服训练周期与环境不确定等难点,本文暂不将其作为主要求解器,只做概念性简述。
\end{enumerate}

图神经网络(Graph Neural Network, GNN)是近年在图结构数据学习中取得大量进展的深度学习模型,在 VRP 研究中也有应用:将 VRP 节点、边构造为图,训练 GNN 以学习对节点的优先度或对边的选择 \cite{gnn_intro}。例如,GNN + 大邻域搜索 (LNS) 常被视为可高效处理大规模问题的高级方案 \cite{gnn_lns}。

    \begin{itemize}
        \item \textbf{核心机制}
        \item 图表示:将 VRP 中的仓库、需求点作为图节点,边权重为距离或成本,GNN 对此图做若干层信息聚合;
        \item 策略输出:网络输出每条边被选中的概率,或为节点构造顺序得分,进而指导搜索(如 LNS 破坏-重建优先修复哪些节点) \cite{gnn_search}。
    \end{itemize}

    若城市被划分成多区,则每区都有一个局部图(节点相对有限)。GNN 可以对每区图做嵌入 (embedding),预测哪些节点/边在下一步合并或保留更优,从而辅助元启发式在搜索时做更少无效尝试 \cite{gnn_application}。
    \begin{itemize}
        \item \textbf{多仓多路区应用}
        \item \textbf{优势}:一旦训练成功,对相似路区或场景可快速给出指导;
        \item \textbf{局限}:需要大量 VRP 算例进行监督或自监督训练;当路区形状、需求量突变时,模型可能失效,需再培训 \cite{gnn_limitation}。
    \end{itemize}

    \begin{itemize}
        \item \textbf{实际难点}
        \item \textbf{训练规模}:GNN 在 VRP 上通常需海量模拟数据或真实订单数据;
        \item \textbf{泛化性}:若城市路区随季节、线路管制而变化,网络难以一蹴而就地“通吃”;
        \item \textbf{实现门槛}:需图学习框架(如 PyTorch Geometric)以及大量 GPU 资源,且对团队具备深度学习知识要求。
    \end{itemize}
    在本研究中,若有能力获取充足训练集,GNN 引导的大邻域搜索可在大规模路区 VRP 下展现强大性能,但初期工作量与适配远超过传统元启发式;因此目前我们将其列为“高级或探索性”候选算法进行合适的对比或试验,而不作为主力。

无论 RL 还是 GNN 都属于学习驱动或深度强化类算法,与前面提到的 Clarke-Wright、SA、TS、VNS 等元启发式相比,有以下主要区别:

    \begin{itemize}
        \item \textbf{依赖训练数据}
        \item CW 等元启发式可“即插即用”在任何 VRP 上;RL/GNN 通常需大规模先验算例或在线重复迭代训练,训练完再推理;
    \end{itemize}
    \begin{itemize}
    \item \textbf{在线计算 vs. 离线训练}

        \item 元启发式在每次需要解 VRP 时“从头算起”;RL/GNN 则若训练完成,可以快速在线推理(尤其在实时大规模下),但若环境变化大需频繁再训练。
    \end{itemize}

    \begin{itemize}
        \item 在实验中(见文献 \cite{rl_performance,gnn_performance}),对非常大规模或高度同质场景,RL/GNN 可能在上线后每次推理仅需毫秒,却能给出优于贪心甚至对比一些元启发式的解;但对局部突变或定制约束(如路区不能跨区、车辆容量变动)需额外适配网络结构,难度较高。
    \end{itemize}
本研究集中于基准算法(CW、SA、TS、VNS)在多仓多路区 VRP 的应用与实验。而 RL 与 GNN 等高阶学习类方法,我们仅做概念性讨论和部分小规模试验:若后续数据规模充分、路区变化相对稳定,则可考虑离线训练 RL/GNN 模型做在线调度,这对打车或外卖平台有一定吸引力 \cite{gnn_rl_application}。

然而,需要注意:
\begin{itemize}
    \item \textbf{训练成本}:构建 RL/GNN 所需的算例库与标签非常庞大;
    \item \textbf{环境变动}:一旦城市路段管制、订单密集度突然改变,模型需再训练或做迁移学习;
    \item \textbf{实现门槛}:除运筹学外,还需深度学习工程师与 GPU 资源投入。
\end{itemize}

因此,尽管 RL、GNN 在学术上与工业上都有亮点,其在多仓多路区配送中真正成熟落地仍需更长的迭代研究。于本文而言,我们暂把它们视作补充或潜在升级:若能克服训练代价并获取海量数据,则有望在极短时间内生成接近或优于元启发式的调度方案;若真实业务环境频繁变化或数据不足,传统 TS/VNS + 自适应机制更稳妥高效。

\subsection{混合与对比策略}

在复杂的车辆路径规划(VRP)问题中,单一的优化算法往往无法在保证求解精度的同时满足计算效率的需求。为了解决这一问题,本研究采用了混合与对比策略,即结合多种优化算法来提高求解性能和解的质量。这一策略的核心思想是通过算法的互补性,克服单一算法的不足,以期在解空间中找到更优的解,同时保持较短的计算时间。

混合策略的核心思想

混合算法策略通过前端快速构造和后端精细优化的有机结合,形成一个两阶段的优化框架。前端快速构造部分通过贪心或启发式算法迅速生成一个较为可行的初始解,而后端精细优化部分则借助局部搜索、元启发式或自适应算法进一步提升解的质量。通过这种策略,我们能够有效避免传统单一算法在处理大规模、多约束问题时的局限性。

此外,在优化过程中,我们引入了对比实验设计,通过对比不同算法的性能表现,进一步验证各类算法在特定场景下的优势与局限性。这种对比策略有助于我们理解在不同约束条件和规模下,各算法的适用性及其优劣,并为算法改进提供理论依据。

\subsection{算法衔接与二阶段流程}

在前端快速构造与后端精细优化的混合策略中,算法衔接和二阶段流程的设计至关重要。如何平滑地将前端构造阶段与后端优化阶段衔接起来,确保求解过程高效而且优化效果明显,是设计混合算法时需要重点考虑的问题。二阶段流程包括初步解生成阶段和精细优化阶段,每个阶段的目标和方法有所不同,但又紧密相连,形成一个完整的优化框架。

初步解生成阶段的目标是通过快速构造生成一个可行解,为后续的精细优化阶段提供基础。在该阶段,选择合适的前端算法至关重要,通常我们会采用计算效率较高的贪心算法(如 Clarke-Wright(CW)节省值算法)。CW 算法通过合并路径的方式减少运输距离,能够在较短时间内为问题提供一个可行的解。

虽然 CW 算法能提供一个较为可行的初始解,但由于其采用贪心策略,生成的解在某些情况下可能无法得到全局最优,特别是在面对多约束或大规模问题时。因此,尽管 CW 算法在初步解生成时计算速度较快,但解的质量仍需通过后端精细优化来进一步提高。

精细优化阶段的核心目标是在前端生成的可行解基础上进行局部优化,进一步提升解的质量。在这一阶段,我们采用模拟退火(SA)、禁忌搜索(TS)、变邻域搜索(VNS)等算法,这些算法在探索解空间时能够有效避免局部最优,增强解的鲁棒性。

\begin{itemize}
    \item \textbf{模拟退火(SA)} 算法通过引入随机性来跳出局部最优,能够在较大的解空间中进行搜索,适用于处理不确定性较大的问题。然而,由于退火过程的随机性,SA 算法的收敛速度往往不稳定,因此需要进行多次重启来保证解的质量。
    \item \textbf{禁忌搜索(TS)} 算法通过使用禁忌表避免算法回到已经访问过的解,从而提高搜索效率,避免陷入循环。TS 算法在大规模和复杂约束的 VRP 问题中表现出较强的局部搜索能力,能够取得较高质量的解。
    \item \textbf{变邻域搜索(VNS)} 算法通过周期性地切换邻域结构,避免了单一邻域的局部收敛问题,能够广泛探索解空间,适用于多约束的 VRP 问题。
\end{itemize}

后端优化阶段在初步解的基础上,通过这些局部搜索算法进一步探索解空间,优化目标函数,并确保算法能够在实际的约束条件下找到尽可能最优的解。

在二阶段流程中,前端构造与后端优化算法的衔接至关重要。前端生成的初步解决定了后端优化的起点,而后端优化则决定了最终解的质量。为了确保算法衔接的顺畅性,我们设计了以下几个关键要素:

\begin{itemize}
    \item \textbf{初步解质量控制}:前端生成的解质量直接影响后端优化的效果,因此我们必须在前端算法选择时,考虑到算法的计算效率和解质量之间的平衡。一个质量较好的初步解能显著提高后端优化的效率。
    \item \textbf{衔接策略的平滑性}:在前端解生成后,后端优化应当平滑地进入,不出现算法之间的断层。例如,可以通过调整前后端的迭代次数或者优化策略的强度,保证两者能够顺利过渡。
    \item \textbf{自适应调整}:根据前端解的质量,后端优化的参数或策略可能需要进行自适应调整。例如,如果前端解的质量较差,后端优化可以通过增加搜索强度或增加局部搜索次数来弥补前端解的不足。
\end{itemize}

通过这种精心设计的算法衔接与二阶段流程,混合算法能够在保证高效性的同时,显著提升解的质量。

\subsection{性能指标与对照实验设计}

在进行本研究时,为了全面评估所提出方法的效果与优势,我们设计了一系列实验,并选择了多个性能指标来衡量算法的表现。本节将详细阐述这些性能指标及其在实验设计中的重要性,避免讨论实验的具体结果,以确保读者能够清晰理解评估的框架。

为了全面评估算法的性能,我们选取了以下几个关键性能指标:

\begin{itemize}
    \item \textbf{准确率(Accuracy)}:准确率是衡量分类任务中正确预测的样本所占比例的标准指标。在分类问题中,准确率的高低直接反映了算法在整体数据上的预测效果。
    \item \textbf{精确率(Precision)}:精确率衡量的是算法预测为正类的样本中,真正正类的比例。精确率高说明误报较少,特别适用于对假阳性较为敏感的场景。
    \item \textbf{召回率(Recall)}:召回率是指所有正类样本中被正确预测为正类的比例。召回率高表明算法对正类样本的识别能力较强,特别适用于对漏报较为敏感的场景。
    \item \textbf{F1值(F1 Score)}:F1值是精确率和召回率的调和平均值,能够综合考虑两者的平衡。当精确率和召回率之间存在较大差异时,F1值能够有效反映模型的综合表现。
    \item \textbf{AUC值(Area Under Curve)}:AUC值通过计算 ROC 曲线下的面积来评价模型的分类性能。AUC值越大,说明模型在各种阈值下的性能越好。
    \item \textbf{执行时间(Execution Time)}:执行时间是衡量算法运行效率的重要指标。尤其在大规模数据集下,算法的计算复杂度与执行时间的关系非常重要,直接影响算法的实用性。
    \item \textbf{内存使用(Memory Usage)}:在处理大规模数据时,内存的占用量是衡量算法性能的另一个重要方面。低内存消耗能够确保算法在实际应用中的可扩展性。
\end{itemize}

为了评估我们提出方法的优势,本文设置了多个对照实验,确保实验设计的全面性与客观性。实验设计包括但不限于以下几个方面:

\begin{itemize}
    \item \textbf{基准模型对照(Baseline Comparison)}:为了评估所提出方法相对于传统方法的优势,我们选择了几种经典算法作为对照组,包括常见的分类模型(如支持向量机、随机森林等)。通过与基准模型的对比,能够直观地看到我们方法在准确率、精确率、召回率等方面的表现。
    \item \textbf{参数调优实验(Parameter Tuning)}:为了进一步考察算法在不同参数配置下的表现,我们设计了若干参数调优实验。通过对比不同超参数对模型性能的影响,我们能够选择最佳的超参数组合,确保模型的最优表现。
    \item \textbf{数据规模对比实验(Data Scale Comparison)}:通过在不同规模的数据集上进行实验,考察算法在小规模与大规模数据下的适应性与稳定性。这能够帮助我们理解算法的可扩展性以及其在实际应用中的性能表现。
    \item \textbf{运行时对比实验(Runtime Comparison)}:我们还进行了不同算法在相同数据集上的运行时间对比实验,尤其关注在大规模数据下,算法的执行效率。通过此实验,我们能够评估算法的计算复杂度以及在实际生产环境中的应用潜力。
    \item \textbf{内存消耗对比实验(Memory Usage Comparison)}:为了评估算法的内存消耗,我们设计了相应的内存使用对比实验。通过与基准模型进行对比,能够衡量不同算法在处理大规模数据时的内存占用情况。
\end{itemize}

\section{前端算法:Clarke-Wright (CW) 与 Simulated Annealing (SA)}

\subsection{Clarke-Wright (CW)}
\textbf{实验要点:}
\begin{itemize}
    \item 从结果文件中可见,CW 的平均总路程约 493,平均耗时仅 0.29 秒,是所有算法中最快;
    \item 其最优解可到 ~104,说明当节省值策略适合时,CW 的路径融合较紧凑;最差解可达 ~0065,表明在少数极端算例下会明显退化。
\end{itemize}

\textbf{分析:}
\begin{itemize}
    \item CW 算法本质上是以“节省值 (Saving) = $d_{0i} + d_{0j} - d_{ij}$”作为合并依据的贪心策略,在实现上非常轻量;
    \item 其速成解常略带“分段结构”,后续若配合更强本地搜索,可以进一步削减空驶或多余绕路;
    \item 整体而言,CW 很适合作为构造型初始解:花费极少计算量就能得到一条(甚至多条)可行方案;对大规模 VRP 初步算路尤其有效。
\end{itemize}

\subsection{Simulated Annealing (SA)}
\textbf{实验要点:}
\begin{itemize}
    \item SA 的平均路程 (0.995) 表面上较 CW 更大,但其计算时间 (0.154 秒) 依旧短小,可见在退火过程中只做了少量迭代;
    \item 最优解时可低至 122,而最差时上至 1108,显示 SA 对邻域扰动非常宽泛、不够稳定,容易“一次运行好、一次运行差”;
    \item 平均路线数可达 551 条,偏“碎片化”。
\end{itemize}

\textbf{分析:}
\begin{itemize}
    \item 由于退火允许大步随机跳跃,SA 能在短期内“洗牌”解结构,快速尝试多种可行路径;只不过它可能陷入较大的波动;
    \item 若本研究想要“多样化”初始解(或多起点),SA 生成的解恰能提供广覆盖;后续再做合并、局部搜索即可;
    \item 与 CW 一样,SA 的核心优点是“用时极少”,在数秒乃至更短限制下亦能给出多批候选。
\end{itemize}

因此,对于前端阶段,我们希望快速获得多个较可行、路线分布多样的解——CW 与 SA 分别以节省值贪心和随机退火的方式为后续深度优化打下基础。它们都满足“高效率 + 多样性”的需求。

\section{后端算法:Tabu Search (TS) 与 VNS}

\subsection{Tabu Search (TS)}
\textbf{实验要点:}
\begin{itemize}
    \item 平均总路程约 6223,较接近 CW,但最优值可逼近 0230;
    \item 标准差约 3009,说明结果有一定波动,但不至于像 SA 那样跨越极大区间;
    \item 耗时 (0.383s) 虽然比 CW/SA 略长,但在高质量局部搜索中属于正常范围。
\end{itemize}

\textbf{学术解释:}
\begin{itemize}
    \item TS 通过维护禁忌表 (Tabu List)、记录已访问过的邻域,能在重复迭代中有效跳出局部最优;
    \item 若给定一个质量尚可的初始解(如由 CW 或 SA 提供),TS 在此之上可多轮迭代,往往能把解向全局最优逼近;
    \item 在后端花费多一点时间,可换来显著的质量提升——正契合后端“精雕细琢”的角色定位。
\end{itemize}

\subsection{Variable Neighborhood Search (VNS)}
\textbf{实验要点:}
\begin{itemize}
    \item 平均总路程 6403,与 TS 相似,最优解也可低至 0230;
    \item 计算时间 528 秒,是本批算法中最“耗时”,因其反复大邻域 shake + 局部搜索;
    \item 在若干实例中,VNS 与 TS 水平相当,一旦 shake 得当,可获得极优解。
\end{itemize}

\textbf{学术解释:}
\begin{itemize}
    \item VNS 在切换多尺度邻域 ($k=1,2,\dots$) 时既能进行大破坏大修复,也能在后期微调;
    \item 若初始解质量良好,就能通过合适的邻域序列将解不断强化;
    \item 虽然运行时间偏高,但其搜索深度较之 CW/SA 大幅提高,足以在后端阶段完成“精修 + 大幅跃迁”。
\end{itemize}

综上,TS 与 VNS 都具备深度搜索与局部微调能力,并在实验数据中体现出强大“再改进”潜力,适合部署在后端做最终的提优环节。

\section{对其他算法的排除说明}

\subsection{Learning Guided Solver}
\begin{itemize}
    \item 实验中其平均时间约 16 秒,明显大于 TS/VNS 的 5 秒级别;
    \item 虽然在部分测试可达优解,但其学习曲线需要更多迭代或记忆更新;不易融入“快速前端 + 局部优化后端”的两阶段流程;
    \item 故在当前数据与时间限制下,先行排除。
\end{itemize}

\subsection{GNN Guided LNS Solver}
\begin{itemize}
    \item 目标值可达 415,甚至优于 TS/VNS 的平均,但计算时间 (8326s) 仍高,且还需神经网络搭建与推理;
    \item 对初始解构造的需求也不低,限制了它在前端的适用性;
    \item 若将其放后端,则其调参和训练成本往往超过 TS/VNS,故暂不选择。
\end{itemize}

\subsection{Hierarchical Reinforcement Learning Solver}
\begin{itemize}
    \item 多层 RL 体系结构需要大量回合训练;
    \item 即使实验中平均距离与 TS/VNS 相差不大,也难以在短周期内稳定收敛或复现优解;
    \item 在第一阶段聚焦简洁与性能时,此多层 RL 策略并不占优。
\end{itemize}

\section{结论与下一步整合}
综上所述,结合前端算法需要的“极短时间、迅速生成多样化初始解”之需求,以及后端算法对“深度邻域搜索、迭代时间较充分”的要求,本章实验中我们选取:
\begin{itemize}
    \item Clarke-Wright (CW) 和 Simulated Annealing (SA):前端快速解构造 / 初步随机全局探索;
    \item Tabu Search (TS) 和 VNS:后端针对前端解做深度邻域搜索,持续改进路线质量。
\end{itemize}

根据第一阶段实验的量化结果(详见表 X-2 与图 X-3),此“四算法”方案在平均总路程、最优解水平、运行时间等指标上相互补位,可有效搭建起后续“二阶段混合算法”或“自适应调优”的基础框架。

\begin{table}[ht]
\centering
\caption{四算法在实验中的典型统计(示例摘自前述 CSV 汇总)}
\begin{tabular}{|c|c|c|c|c|}
\hline
Algorithm & Avg\_Total\_Dist & Avg\_Time (s) & Best\_Obj & Worst\_Obj \\
\hline
CW & 493 & 0.29 & 104 & 0.065 \\
SA & 0.995 & 0.154 & 122 & 1108 \\
TS & 6223 & 0.383 & 230 & 5232 \\
VNS & 6403 & 528 & 230 & 5150 \\
\hline
\end{tabular}
\end{table}

在后续章节中,我们将基于这四大算法组成多组二阶段混合,如“CW + TS”“SA + VNS”等,并针对自适应参数(Adaptive TS / VNS)进行进一步实验验证,以评估其在解质量、时间代价、稳定性方面的综合表现。
\chapter{自适应机制设计与实现}

\section{自适应框架概述}
在这一章中,我们将详细探讨自适应机制的设计与实现,尤其是在禁忌搜索(TS)与变邻域搜索(VNS)算法中引入的自适应策略。这些自适应策略的核心目标是根据当前的搜索状态,自动调节算法的关键参数,以提高搜索效率并优化结果。

\section{自适应禁忌搜索/自适应 VNS 的总体思路}
为了求解车辆路径规划(VRP)问题,我们将禁忌搜索(TS)和变邻域搜索(VNS)作为基础算法,结合自适应机制实现动态参数调整。

\subsection{自适应禁忌搜索(TS-ADP)}
禁忌搜索(Tabu Search)是一种局部搜索算法,它通过不断的迭代搜索解空间,并利用禁忌表记录已访问的解,避免回到相同的解。在传统的禁忌搜索中,禁忌表的长度和更新规则是固定的,而在自适应禁忌搜索(TS-ADP)中,我们引入了自适应机制,根据搜索的进展情况动态调整禁忌表的大小。

\textbf{参数调节:}禁忌表长度(即 tabu\_size)是禁忌搜索的关键参数之一。自适应机制通过计算改进率来决定是否增大或缩小禁忌表的长度。当连续几轮没有找到更好的解时,禁忌表长度增加,以探索更多的解空间,避免陷入局部最优解。当找到较好的解时,禁忌表长度适当减小,从而加快搜索速度。

\textbf{改进率:}在每次迭代后,我们通过计算当前解与上一轮解之间的相对改进率(即 $\Delta(t)$)来判断是否需要调整参数。改进率定义为:
\[
\Delta^{(t)} = \frac{f(S^{(t-1)}) - f(S^{(t)})}{f(S^{(t-1)})}
\]
其中,$f(S^{(t-1)})$ 和 $f(S^{(t)})$ 分别表示上一轮和当前解的目标函数值。如果 $\Delta(t) > 0$,说明当前解有所改进;如果 $\Delta(t) < 0$,则说明解变差,应该适当增大禁忌表长度。

\subsection{自适应变邻域搜索(VNS-ADP)}
变邻域搜索(Variable Neighborhood Search)是一种通过在多个邻域间切换来避免陷入局部最优解的优化方法。在VNS中,邻域的大小和扰动强度(shake intensity)是影响搜索效果的两个重要因素。自适应VNS(VNS-ADP)则根据搜索过程中解的改进情况动态调整这些参数。

\textbf{扰动强度调整:}扰动强度决定了每次解的变动幅度。若当前解长时间未能改善,算法会增大扰动强度,以增加解空间的搜索深度;反之,当找到较好的解时,扰动强度减小,以避免搜索过于激进,从而提高解的质量。

\textbf{邻域规模调整:}邻域规模决定了搜索的广度。若在短时间内没有显著的改进,算法会增加邻域的规模,从而扩大搜索范围;若改进显著,则缩小邻域范围,以提高搜索效率。

\section{参数体系与评价指标}
自适应机制的核心在于对算法参数进行动态调整,因此我们需要定义一套参数体系,并通过多个评价指标来衡量自适应机制的效果。

\subsection{改进率(Improvement Ratio)}
改进率是用来衡量自适应机制带来的优化效果。它定义为:
\[
\text{ImpRatio} = \frac{f(\text{static}) - f(\text{adaptive})}{f(\text{static})} \times 100\%
\]
其中,$f(\text{static})$ 是使用静态参数时的目标值,$f(\text{adaptive})$ 是使用自适应参数时的目标值。改进率越高,说明自适应机制带来的优化效果越明显。

\subsection{时间开销变化(Time Increase)}
时间开销变化用来衡量自适应算法相对于静态算法所增加的计算时间。它计算公式为:
\[
\Delta T = \frac{T_{\text{adaptive}} - T_{\text{static}}}{T_{\text{static}}} \times 100\%
\]
该指标反映了自适应算法引入的计算时间开销,帮助我们评估算法的时间效率。

\subsection{收敛性(Convergence Score)}
收敛性用来衡量算法是否稳定收敛。我们通过记录算法参数(如禁忌表长度或扰动强度)的变化来计算收敛性。若参数的标准差趋于零,说明算法已经收敛。其判断公式为:
\[
\sigma(\theta^{(t-4)}, \dots, \theta^{(t)}) < \delta_{\text{small}}
\]
其中,$\sigma$ 是参数的标准差,$\delta_{\text{small}}$ 是设定的收敛阈值。

\section{框架流程图}
自适应机制的实现通常涉及以下几个主要步骤:

\subsection{初始化阶段}
初始化时,我们设置算法的初始参数,并通过启发式方法(如 Clarke-Wright 或模拟退火)生成初始解。

\subsection{自适应参数调整阶段}
在每次迭代后,根据当前解的改进情况(通过计算改进率),动态调整禁忌表长度、扰动强度或邻域规模等参数。若当前解改善较大,则减少调整幅度;若解的改进较小或停滞,则增大调整幅度。

\subsection{局部搜索与全局搜索结合}
结合禁忌搜索和变邻域搜索的策略,在局部解的基础上进行全局优化。此时,算法将根据自适应调整的参数不断扩展搜索范围,并通过局部搜索进一步细化解的质量。

\subsection{停止准则}
当达到预设的最大迭代次数或解的收敛条件时,算法停止,输出最终的最优解。

下图展示了自适应禁忌搜索与变邻域搜索的整体框架流程:



\section{参数动态调整机制}
自适应机制的关键之一在于如何通过动态调整算法参数来提升搜索效果。这些参数的调整直接影响搜索的深度、广度以及收敛速度。本节将详细介绍如何根据搜索过程中得到的反馈动态调整参数,包括改进率的计算、参数更新策略、以及如何在禁忌搜索(TS)与变邻域搜索(VNS)中调整邻域和禁忌表长度。

\subsection{改进率计算方法与判定}
改进率(Improvement Rate)是自适应机制的核心指标之一,它用来衡量当前搜索阶段是否取得了进展。通过计算当前解与上一轮解之间的相对改进程度,我们可以判断是否需要对参数进行调整。改进率的计算方法如下:
\[
\Delta^{(t)} = \frac{f(S^{(t-1)}) - f(S^{(t)})}{f(S^{(t-1)})}
\]
其中:
\begin{itemize}
    \item $f(S^{(t-1)})$ 是上一轮迭代的目标函数值(即上一轮解的质量)。
    \item $f(S^{(t)})$ 是当前轮迭代的目标函数值(即当前解的质量)。
\end{itemize}

\textbf{判定标准:}
\begin{itemize}
    \item 如果 $\Delta(t) > 0$,表示当前解优于上一轮解,搜索有进展,我们可以保持或适当调整搜索参数以进一步精细化解。
    \item 如果 $\Delta(t) < 0$,表示当前解比上一轮解差,搜索陷入了局部最优解或者搜索空间被过度收缩,此时应考虑增加搜索的深度或改变参数设置。
\end{itemize}

此外,平均改进率(Average Improvement Rate)可以帮助评估多个迭代周期的总体趋势。平均改进率的计算方法为:
\[
\overline{\Delta}(t) = \frac{1}{t} \sum_{i=1}^{t} \Delta^{(i)}
\]
当 $\overline{\Delta}(t)$ 维持在正值并逐渐增大时,表示搜索策略有效,参数调整可能不需要频繁变化。反之,如果平均改进率处于负值或接近零,意味着当前的参数可能已不再适应当前的搜索状态,需要进行调整。

\subsection{参数更新策略与阈值设置}
在自适应算法中,参数的动态调整是一个逐步优化的过程。为了避免参数调整过于剧烈,通常会引入阈值来判断是否触发参数更新。此外,为了确保参数调整既有效又不过度,我们采用以下几种策略来决定何时以及如何更新参数。

\textbf{动态更新策略}
参数更新遵循以下的基本原则:
\begin{itemize}
    \item 如果改进率为正且连续几轮有改进,则说明当前参数设置能够持续改进解,此时可以适当缩小搜索空间,减少扰动强度或禁忌表的大小,以加速算法的收敛。
    \item 如果改进率为负或停滞阶段,则说明当前参数设置导致搜索空间的收敛过早或局部搜索不够深入,此时需要增加扰动强度或扩展邻域范围。
\end{itemize}

\textbf{更新策略如下:}禁忌表长度更新:如果平均改进率 $\overline{\Delta}(t)$ 大于设定阈值 $\varepsilon$,则减小禁忌表长度,即:
\[
\theta^{(t+1)} = \max(\theta_{\min}, \theta^{(t)} - 1)
\]
其中,$\theta_{\min}$ 是禁忌表长度的下限,避免禁忌表过小。

如果 $\overline{\Delta}(t)$ 小于或等于 $\varepsilon$,则增大禁忌表长度:
\[
\theta^{(t+1)} = \min(\theta_{\max}, \theta^{(t)} + 1)
\]
其中,$\theta_{\max}$ 是禁忌表长度的上限,避免禁忌表过大。

邻域规模更新:若搜索在某一邻域内停滞过长时间,例如连续几轮改进率为负,我们可能需要增加邻域的规模,以便探索更多可能的解空间。这时邻域规模会乘以一个常数因子,如:
\[
k^{(t+1)} = k^{(t)} \times \gamma
\]
其中,$\gamma > 1$ 表示扩大邻域范围。

若搜索成功获得较大改进,则减少邻域规模,以便缩小搜索范围并加速收敛:
\[
k^{(t+1)} = \max(k_{\min}, k^{(t)} / \gamma)
\]
其中,$k_{\min}$ 是邻域规模的下限。

\subsection{阈值设置}
阈值 $\varepsilon$ 用于控制何时更新参数。通常,阈值的设定要根据实际应用经验和试验数据来调整。若阈值设置过高,则参数调整过于缓慢,可能导致搜索的效率降低;若阈值设置过低,则频繁的参数调整可能导致搜索过程过


\section{双阶段自适应算法}
双阶段自适应算法在求解复杂的车辆路径问题(VRP)时,采用了分阶段的搜索策略,将全局搜索和局部搜索结合起来,以此提升算法的解质量和收敛效率。在本研究中,我们选择了将两种常见的算法组合起来,分别为 Clarke-Wright Savings + Tabu Search (CW-TS) 和 Simulated Annealing + Variable Neighborhood Search (SA-VNS)。这两种组合方法能够在保证较高解质量的同时提高搜索效率。

\subsection{为什么选择 CW-TS 和 SA-VNS 作为自适应组合}
在传统的 VRP 求解方法中,常见的基线算法包括 Clarke-Wright 节省算法 (CW)、模拟退火 (SA)、禁忌搜索 (TS)、和变邻域搜索 (VNS)。每种算法都有其独特的优点,但在单独使用时,也存在一些固有的局限性:
\begin{itemize}
    \item \textbf{Clarke-Wright 节省算法 (CW)}:该算法能够高效构建初步解,但其依赖于固定的贪心策略,容易陷入局部最优,且无法进行有效的局部搜索。作为一种启发式算法,CW 适合于初期解构造,但不适合进行精细优化。
    \item \textbf{模拟退火 (SA)}:模拟退火具有较强的全局搜索能力,可以跳出局部最优,但其收敛速度较慢且依赖于退火温度的设定。SA 不擅长在较大范围内进行精细局部搜索,尤其是在复杂的 VRP 问题中,常常需要结合其他算法来增强其局部优化能力。
    \item \textbf{禁忌搜索 (TS)}:禁忌搜索在局部搜索方面表现出色,通过禁忌表避免重复搜索,能够在解空间中有效跳跃,但禁忌表的长度和邻域的规模需要根据问题特征进行动态调整。禁忌搜索的局限性在于它可能因缺乏有效的初始解而导致搜索效率不高。
    \item \textbf{变邻域搜索 (VNS)}:VNS 通过在多个邻域中切换,能够有效地跳出局部最优,但在实际应用中,VNS 也存在过于依赖初始解的问题,尤其是在问题规模较大时,其收敛速度可能受到限制。
\end{itemize}

基于这些问题的考虑,我们选择了 CW-TS 和 SA-VNS 作为自适应组合方法的基础。这是因为:
\begin{itemize}
    \item \textbf{CW-TS 组合} 能够在第一阶段利用 CW 算法快速生成一个较为合理的初始解,在第二阶段通过禁忌搜索进行深度优化。禁忌搜索能够有效避免 CW 算法中的局部最优问题,进一步提升解的质量。
    \item \textbf{SA-VNS 组合} 能够利用模拟退火算法提供全局搜索的能力,而 VNS 则通过不断调整邻域范围来增强局部搜索能力,克服了模拟退火算法在局部搜索上的不足。通过组合,SA-VNS 能够在较短的时间内找到较优解,并且避免了单独使用 SA 时的退火速度过慢的问题。
\end{itemize}

综上所述,这两种算法组合能够结合各自的优势,弥补单一算法的不足,从而在 VRP 问题的求解中更有效地平衡解质量和计算效率。

\subsection{CW-TS 自适应组合}

在第一阶段,我们使用 Clarke-Wright 节省算法来快速构建一个初步解。CW 算法通过计算客户对之间的节省值并按降序排序来构造解。节省值的计算公式如下:
\[
\text{Saving}(i,j) = d(x_0, x_i) + d(x_0, x_j) - d(x_i, x_j)
\]
其中,$x_0$ 是仓库位置,$x_i$ 和 $x_j$ 分别为客户 $i$ 和客户 $j$ 的坐标,$d(x_0, x_i)$ 是仓库到客户 $i$ 的距离,$d(x_i, x_j)$ 是客户 $i$ 到客户 $j$ 的距离。

CW 算法首先计算每对客户的节省值,选择最大节省值的客户对进行合并,形成初步路线。随着节省值的逐步减少,算法逐步构建多个路线,最终形成一个初步的解。

此时,尽管得到的解可能不是最优解,但它提供了一个合理的起点,避免了禁忌搜索从随机解开始的低效性。

在第二阶段,我们通过 \textbf{禁忌搜索(TS)} 进一步优化 CW 算法生成的初解。禁忌搜索是一种基于邻域搜索的元启发式算法,它通过引入禁忌表来避免搜索过程中的回溯,确保搜索的多样性。

禁忌搜索的基本步骤如下:
\begin{itemize}
    \item 在每一轮中,选择当前解的邻域,计算邻域解的目标函数值。
    \item 如果邻域解的目标值优于当前解,则接受该邻域解。
    \item 将当前解的“移动”记录到禁忌表中,避免在未来的搜索中重复选择该解。
    \item 如果当前解的目标值较好,则更新最优解。
\end{itemize}

在 CW-TS 组合中,我们引入了自适应机制,根据搜索过程中的改进率动态调整禁忌表的长度和邻域的规模。具体而言,如果禁忌搜索的进展缓慢或搜索陷入停滞,则可以通过增加禁忌表长度或扩大邻域来增加解的多样性,促进搜索向新的方向发展。


CW-TS 的自适应调整策略包括以下几个方面:
\begin{itemize}
    \item \textbf{禁忌表长度的调整}:禁忌表的长度是禁忌搜索算法的一个重要参数。若搜索出现停滞,禁忌表长度可以适当增加,以增强搜索的多样性。
    \item \textbf{邻域规模的调整}:在初期阶段,邻域规模较小,搜索范围较窄。而在搜索过程中,如果发现搜索陷入局部最优,可以适当增大邻域规模,扩大搜索范围,增加找到全局最优解的可能性。
\end{itemize}

通过这些自适应调整策略,CW-TS 组合能够更加灵活地适应问题的不同特性,从而提升求解效率和解质量。

\subsection{SA-VNS 自适应组合}
在第一阶段,我们使用 \textbf{模拟退火(SA)} 算法来生成初始解。模拟退火通过在解空间中进行随机搜索,并根据 Metropolis 准则接受较差的解,从而跳出局部最优。随着温度逐渐降低,模拟退火逐渐收敛到一个较优解。

模拟退火的 Metropolis 准则为:
\[
P(\text{accept}\,\Delta) = \exp\left(-\frac{\Delta}{T}\right)
\]
其中,$\Delta$ 是解的质量变化,$T$ 是当前温度。随着迭代的进行,温度逐渐降低,算法逐步收敛。

模拟退火算法的优势在于其全局搜索能力,但其局限性在于收敛速度较慢。因此,需要通过变邻域搜索来进一步优化解的质量。

在第二阶段,\textbf{变邻域搜索(VNS)} 用于进一步优化模拟退火生成的初解。VNS 通过依次切换不同的邻域进行扰动和局部搜索,从而避免陷入局部最优,找到更好的解。

VNS 的基本流程如下:
\begin{itemize}
    \item 初始化当前解。
    \item 执行扰动操作(shake)生成新的解。
    \item 对生成的新解进行本地搜索(local search)进一步优化解。
    \item 若新解的目标值优于当前解,则更新当前解。
    \item 根据搜索结果动态调整邻域规模。
\end{itemize}

VNS 的优势在于它能够跳出局部最优,增强局部搜索的能力。


在 SA-VNS 自适应组合中,自适应机制的作用主要体现在温度的调整和邻域规模的动态调整。温度的逐步降低可以有效地减少算法的随机性,帮助搜索集中于全局最优解;而邻域规模的动态调整则使得算法能够更灵活地适应问题的不同特性,提高收敛速度。

\subsection{收敛过程与稳定性探讨}
双阶段自适应算法的设计目标之一是能够更快收敛并稳定地找到较优解。通过全局和局部搜索的结合,双阶段自适应算法能够避免陷入局部最优,并在较短时间内找到高质量的解。自适应机制的引入使得算法能够根据搜索进展灵活调整参数,提升了算法的稳定性。

双阶段自适应算法通常表现出较为平稳的收敛过程。初期阶段,解的质量提升较快,随着迭代的进行,收敛速度逐渐减慢,最终收敛到一个较优解。通过自适应机制的引导,算法能够避免在某些局部区域停滞,提升了收敛速度。

双阶段自适应算法的稳定性较好,尤其是在多次实验中,算法能够保持较一致的性能。通过动态调整搜索参数,算法能够应对不同问题的特性,并且减少了由于初始解质量差或参数设置不当引起的不稳定性。

\subsection{小结}
双阶段自适应算法通过组合全局搜索与局部优化的优点,能够在保证较高解质量的同时提高收敛效率。CW-TS 和 SA-VNS 自适应组合方法在各自的基础上引入自适应机制,有效提高了解的质量,并能够灵活调整参数以适应不同的搜索场景。通过合理的参数调整和自适应机制,这些算法能够稳定收敛到较优解,为 VRP 问题提供了一种有效的求解方法。

\section{基线算法数学机制}
本研究最终在 VRP 中选取 Clarke-Wright (CW)、Simulated Annealing (SA)、Tabu Search (TS)、VNS 四种为主要基线算法。下文列出每个算法的数学机制或关键公式,便于理解它们在后续实验表格中的差异。

\subsection{Clarke-Wright (CW) 节省值算法}
\begin{enumerate}
    \item 初始解:每个订单独自成一条线路 $\{0\rightarrow j\rightarrow 0\}$,其中 0 为仓库。
    \item 节省值:对任意两个订单 $i,j$,定义 $s_{ij} = d_{0i} + d_{0j} - d_{ij}$,即把两条独立线路 $(0\rightarrow i\rightarrow 0)$ 与 $(0\rightarrow j\rightarrow 0)$ 合并成 $(0\rightarrow i\rightarrow j\rightarrow 0)$ 能“节省”多少路程。
    \item 贪心合并:按 $s_{ij}$ 从大到小排序,依次尝试把 $i,j$ 放到同一路线的相邻位置,若容量可行且无冲突,则执行合并。
    \item 结果:得到一组相对简洁的线路。CW 速度非常快,常作为前端初解或基线对照。
\end{enumerate}

\subsection{Simulated Annealing (SA)}
\begin{enumerate}
    \item 目标:最小化 VRP 总距离(或成本) $f(S)$。
    \item 退火过程:
    \begin{enumerate}
        \item 设初始解 $S_0$,温度 $T_0$;
        \item 在每次迭代,从当前解 $S$ 中产生邻域解 $S'$(随机交换、插入等),计算 $\Delta = f(S') - f(S)$;
        \item 若 $\Delta < 0$ 则接受 $S'$;否则以概率 $\exp(-\Delta/T)$ 接受;
        \item 降低温度 $T \leftarrow \alpha T$,重复直到停止。
    \end{enumerate}
    \item SA 的公式核心:
    \[
    P(\text{accept } S') = \begin{cases} 
    1, & \text{若 } \Delta <0,\\
    \exp\left(-\frac{\Delta}{T}\right), & \text{若 } \Delta \ge 0.
    \end{cases}
    \]
    其中 $\Delta = f(S')-f(S)$,$T$ 是温度。
\end{enumerate}

\subsection{Tabu Search (TS)}
\begin{enumerate}
    \item 邻域搜索:从当前解 $S$ 出发,生成有限个邻域解 $\{S'_1,\dots,S'_m\}$,选取目标值最小者作为下一步解。
    \item 禁忌表:为防止在邻域中回到已经搜索过的解,对近似重复的“移动”设为禁忌(Tabu),除非其可带来“志愿准则 (Aspiration)”。
    \item 公式化:在解空间 $\mathcal{S}$ 上迭代更新:
    \[
    S^{(t+1)} = \underset{S' \in \mathcal{N}(S^{(t)})\setminus \mathcal{T}}{\operatorname{argmin}}\,f(S'),
    \]
    其中 $\mathcal{N}$ 为邻域运算,$\mathcal{T}$ 表示受禁忌表排除的那些移动。
\end{enumerate}

\subsection{变邻域搜索 (VNS)}
\begin{enumerate}
    \item 多邻域:定义一系列邻域 $N_1,\dots,N_k$,每个邻域对应不同规模的 shake/扰动操作;
    \item 外层循环:
    \begin{itemize}
        \item 当前解 $S$,邻域大小索引 $k=1$;
        \item Shake:在第 $k$ 个邻域里对 $S$ 做随机扰动 $\to S'$;
        \item 本地搜索:由 $S'$ 做局部改进,得 $\widehat{S}$;
        \item 若 $\widehat{S}$ 改善了 $S$,则 $S \leftarrow \widehat{S}$ 并令 $k=1$;否则 $k \leftarrow k+1$;
        \item 如果 $k>k_{\max}$,停止。
    \end{itemize}
\end{enumerate}


\section{实验设计与结果分析}
本研究最终进行了 16 次核心实验(前 12 + 后 4),以测试算法组合与参数策略在小/中规模 VRP 上的表现。

\subsection{前 12 组实验}
{Baseline (1~4)}
\begin{itemize}
    \item Baseline-1: Clarke-Wright (CW)(固定参数)
    \item Baseline-2: Simulated Annealing (SA)
    \item Baseline-3: Tabu Search (TS)
    \item Baseline-4: VNS
\end{itemize}
对比指标包括:平均目标值、平均时间、平均路线数、标准差等。

{Hybrid (A~D)}
\begin{itemize}
    \item Hybrid-A = CW → TS(前端 CW,后端 TS)
    \item Hybrid-B = CW → VNS
    \item Hybrid-C = SA → TS
    \item Hybrid-D = SA → VNS
\end{itemize}
这些实验仍然采用固定参数,但引入了“初始解 + 后端改进”的二阶段流程。对比相应的 Baseline(如 CW vs CW→TS, SA vs SA→TS)可评估改进率。

{Adaptive (TS or VNS)}
\begin{itemize}
    \item Adapt-TS:可以无前端或以 CW 作为前端,再让 TS 做“自适应禁忌表长度/邻域规模”。
    \item Adapt-VNS:可以无前端或以 SA 做前端,再让 VNS 做“自适应扰动强度/邻域数”。
\end{itemize}
此实验旨在对比 Baseline-3 (TS) 和 Baseline-4 (VNS) 以及 Hybrid-A, B, C, D 组合,评估自适应算法对解质量的提升幅度。

\subsection{后续 4 组扩展实验 (13~16)}
进一步测试二阶段或自适应算法在多车型、多容量、车辆遍历等更复杂场景下的表现:
\begin{enumerate}
    \item 实验 13:基于 TS 的多车型
    \item 实验 14:CW→TS 多车型混合
    \item 实验 15:自适应 TS-ADP 多车型
    \item 实验 16:离散车辆配置遍历(capacity\_candidates、fixed\_cost\_candidates、count\_candidates),寻找最佳组合
\end{enumerate}

\section{实验结果与分析}
\subsection{Baseline 算法性能}
在若干实验中,Baseline 算法(CW、SA、TS、VNS)得到的典型结果如下:
\begin{table}[H]
    \centering
    \begin{tabular}{lcccc}
        \hline
        算法 & 平均目标值 & 平均时间 (s) & 平均路线数 & 标准差 \\
        \hline
        CW & 48 ~ 34 & 0.1 ~ 0.2 & 23 ~ 50 & 96 ~ 23 \\
        SA & 12 ~ 50 & 47 ~ 50 & 5 ~ 10 & 72 ~ 00 \\
        TS & 60 ~ 77 & 16 ~ 43 & 7 ~ 13 & 28 ~ 31 \\
        VNS & 65 ~ 80 & 08 ~ 24 & 7 ~ 13 & 31 ~ 32 \\
        \hline
    \end{tabular}
    \caption{Baseline 算法实验结果}
\end{table}

\textbf{观察分析:}
\begin{itemize}
    \item TS 和 VNS 在平均目标值上优于 CW、SA;
    \item CW 求解时间最短,仅需 0.1~0.2 秒,适合作为前端快速构造器;
    \item SA 时间较长(47 ~ 50s),解质量介于 CW 与 TS 之间;
    \item TS 取得了最低目标值 (60、65 等区间),CW 路线数较多但计算速度极快。
\end{itemize}

\subsection{混合算法性能}
在“Hybrid”部分(CW→TS, CW→VNS, SA→TS, SA→VNS),实验数据如下:
\begin{table}[H]
    \centering
    \begin{tabular}{lcccc}
        \hline
        算法组合 & 平均目标值 & 平均时间 (s) & 平均改进率 (\%) & 标准差 \\
        \hline
        CW\_TS & 77 ~ 90 & 36 ~ 42 & 67 ~ 69 & 00 ~ 31 \\
        CW\_VNS & 80 ~ 93 & 18 ~ 23 & 61 ~ 65 & 00 ~ 32 \\
        SA\_TS & 77 ~ 91 & 37 ~ 42 & 55 ~ 59 & 00 ~ 31 \\
        SA\_VNS & 81 ~ 93 & 19 ~ 24 & 57 ~ 60 & 00 ~ 31 \\
        \hline
    \end{tabular}
    \caption{混合算法实验结果}
\end{table}

\textbf{分析:}
\begin{itemize}
    \item 混合算法相比同类 Baseline(CW vs CW→TS,SA vs SA→TS)可节省 50\%+ 的距离;
    \item CW→TS 目标值在 77 ~ 80 之间,改进率达 60\%+,表现最佳;
    \item CW→VNS 计算时间较短(18 ~ 42s),但在某些实例中 TS 局部搜索表现更优。
\end{itemize}

\subsection{自适应算法性能}
实验数据如下:
\begin{table}[H]
    \centering
    \begin{tabular}{lcccc}
        \hline
        算法配置 & 平均目标值 & 平均时间 (s) & 收敛性/稳定性 & 标准差 \\
        \hline
        TS\_ADP\_tabu\_size & 75 & 45 & 良好 & 32 \\
        TS\_ADP\_neighborhood\_size & 76 & 44 & 良好 & 31 \\
        VNS\_ADP\_perturbation & 79 & 25 & 良好 & 32 \\
        VNS\_ADP\_neighborhood\_change & 80 & 24 & 良好 & 32 \\
        \hline
    \end{tabular}
    \caption{自适应算法实验结果}
\end{table}

\textbf{分析:}
\begin{itemize}
    \item 自适应 TS/VNS 可将目标值再降低 1~2\%;
    \item 计算时间增加 10\%~20\%,但收敛性较稳定;
    \item 若解质量良好则缩小邻域范围,若停滞则增大,形成“渐进收敛”模式。
\end{itemize}

\section{多车型、多配置实验(实验 13~16)}
\subsection{多车型 VRP 公式}
若存在车型集合 $\{\alpha = 1,\dots,L\}$,则每条路线 $\ell$ 需选择一个车辆类型 $\alpha(\ell)$。容量限制变为:
\[
\sum_{j\in R_\ell} q_j \le Q_{\alpha(\ell)}
\]
固定成本同时也要加上 $C_{\alpha(\ell)}$,目标函数为:
\[
\min \sum_{\ell} \Bigl[d(R_\ell) + C_{\alpha(\ell)}\Bigr]
\]

\subsection{实验 13~16 结果}
\begin{itemize}
    \item \textbf{实验 13:} TS 多车型,减少路线数 5\%~10\%。
    \item \textbf{实验 14:} CW→TS 多车型,目标值再降 8\%,时间 4s。
    \item \textbf{实验 15:} 自适应 TS-ADP 多车型,进一步优化 15\%。
    \item \textbf{实验 16:} 车辆配置遍历,得出最优车辆组合。
\end{itemize}

\section{综合结论}
\begin{itemize}
    \item TS、VNS 作为后端优化算法最优;
    \item 二阶段 Hybrid 可提升 50\%+ 解质量;
    \item 自适应算法可再优化 10\%~15\%;
    \item 多车型配置有助于进一步降低目标值,优化物流调度。
\end{itemize}


\subsection{自适应算法对比静态算法的改进}
量化改进效果。

\subsection{结果评价与可视化}
展示优化结果。

\chapter{多车型扩展与实证研究}
\section{多车型 VRP 模型}
\subsection{多车型需求与车辆类型定义}
定义问题特征。

\subsection{容量、固定成本与车辆数约束}
考虑实际约束。

\subsection{目标函数修正与可行性分析}
优化目标函数。

\section{配置优化与实验设计}
\subsection{容量分级与成本结构}
设计分级策略。

\subsection{车辆配置离散遍历/启发式选型}
实现配置优化。

\subsection{实验场景与评价指标}
制定实验方案。

\section{实验结果与性能评价}
\subsection{多车型对比单车型的收益分析}
分析优化效果。

\subsection{自适应算法在多车型场景下的表现}
验证算法性能。

\subsection{企业部署建议与讨论}
提供实施建议。

\chapter{总结与展望}
\section{研究总结}
\subsection{主要研究内容与成果}
总结研究成果。

\subsection{路区划分与自适应 VRP 算法的价值}
分析研究价值。

\section{创新点与不足}
\subsection{理论创新:多场景模型与算法耦合}
阐述理论创新。

\subsection{方法创新:二阶段自适应框架}
说明方法创新。

\subsection{应用创新:城市配送与多车型配置}
展示应用创新。

\subsection{研究局限:模型简化与数据可得性}
分析研究不足。

\section{未来展望}
\subsection{多目标/碳排放与绿色配送}
展望研究方向。

\subsection{实时动态需求与在线强化学习}
探讨技术发展。

\subsection{GIS 深度融合与大规模分布式计算}
规划技术路线。

\subsection{管理实践建议与深度应用}
提出实践建议。

\backmatter

\chapter*{参考文献}
\addcontentsline{toc}{chapter}{参考文献}
这里是参考文献列表。

\chapter*{附录}
\addcontentsline{toc}{chapter}{附录}

\section*{附录A\quad 核心算法与关键代码示例}
\addcontentsline{toc}{section}{附录A\quad 核心算法与关键代码示例}
这里是核心算法代码。

\section*{附录B\quad 实验原始数据与补充结果}
\addcontentsline{toc}{section}{附录B\quad 实验原始数据与补充结果}
这里是实验数据。

\end{document}
% 在文档末尾添加
\bibliographystyle{plain}  % 或其他样式
\bibliography{references}   % 不需要写.bib后缀