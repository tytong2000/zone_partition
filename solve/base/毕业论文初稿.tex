\documentclass[12pt,a4paper,twoside]{ctexbook}
\usepackage{cite}  % 加载引用包
\usepackage[hidelinks]{hyperref}
\usepackage{titlesec}
\titlespacing*{\chapter}{0pt}{0pt}{\baselineskip}  % 调整章节前后空白
\usepackage{multirow}

% 基础包
\usepackage{amsmath,amssymb,amsthm}
\usepackage{graphicx,geometry,hyperref}
\usepackage{titlesec,fancyhdr,setspace}
\usepackage{tocloft}
\usepackage{float}
\geometry{a4paper,left=3cm,right=2.5cm,top=2.5cm,bottom=2.5cm}

\renewcommand{\contentsname}{\hfill\bfseries\zihao{3} 目\hspace{2em}录\hfill}
\renewcommand{\cftdot}{$\cdot$}
\renewcommand{\cftchapdotsep}{\cftdotsep}
\renewcommand{\cfttoctitlefont}{\hfill\bfseries\zihao{3}}

% 中文字体设置
\usepackage{fontspec}

\setcounter{secnumdepth}{3}  % 设置编号深度

% 章节设置
\ctexset{
  chapter = {
    name = {第,章},
    number = {\chinese{chapter}},
    format = {\centering\heiti\zihao{3}},
    nameformat = {},
    aftername = {\quad}
  },
  section = {
    name = {},  % 移除额外的节名称
    number = {\thesection},
    format = {\heiti\zihao{4}},
    aftername = {\quad}
  },
  subsection = {
    name = {},  % 移除额外的小节名称
    number = {\thesubsection},
    format = {\heiti\zihao{-4}},
    aftername = {\quad}
  }
}

\fancypagestyle{plain}{
  \fancyhf{}
  \renewcommand{\headrulewidth}{0pt}
  \fancyfoot[C]{\thepage}
}

\pagestyle{plain}

\setcounter{tocdepth}{3}

% 处理分页问题
\let\oldchapter\chapter
\renewcommand{\chapter}{\clearpage\oldchapter}  % 添加新章节时清除上一页空白


\begin{document}
    
\frontmatter
\pagenumbering{Roman}

\begin{titlepage}
\begin{center}
\vspace*{2cm}
{\huge\bfseries 基于动态优化的城市物流系统研究\par}
\vspace{2cm}
{\large\bfseries 童天亦\par}
\vspace{1cm}
{\large 2024年7月\par}
\end{center}
\end{titlepage}

\chapter*{摘要}
\addcontentsline{toc}{chapter}{摘要}
这里是中文摘要内容。

\chapter*{ABSTRACT}
\addcontentsline{toc}{chapter}{ABSTRACT}
This is the English abstract.

\tableofcontents

\mainmatter
\pagenumbering{arabic}

\chapter{绪论}
\section{研究背景及意义}
\subsection{城市配送行业现状与挑战}
随着电子商务与新零售的蓬勃发展,城市配送已成为现代社会经济运行的"毛细血管"。据商务部统计,2023年我国城市配送业务量突破2000亿件,年均增长超过25\%。然而,这种爆发式增长也给城市物流带来前所未有的挑战,主要体现在以下几个方面:

第一,配送需求呈现显著的时空不均衡性。就像城市中的"潮汐现象",商圈、居民区等热点区域的订单密度可能是边缘区域的5-10倍;早晚高峰期的配送压力往往是平峰期的3-5倍。这种不均衡性使得传统的"一刀切"配送方案难以适应。数据显示,在北京、上海等一线城市,配送高峰期某些区域的订单密度可达200-300单/平方公里,而偏远区域可能不足20单/平方公里。

第二,配送成本居高不下,特别是"最后一公里"问题。就像"帕累托法则"在物流领域的映射,最后10\%的配送距离往往消耗30\%-40\%的总成本。据行业调查,物流成本占电商平台总营收的15\%-20\%,其中城市末端配送的成本占比最高。特别是在订单分散、车辆空驶率高的情况下,单票配送成本难以突破天花板。

第三,多样化配送场景带来巨大挑战。现代城市配送已不再局限于传统快递,而是衍生出即时配送、生鲜配送、同城货运等多元业态。就像一座城市中存在"多个平行世界",每类配送都有其独特的时效要求、温控标准和服务规范。例如,外卖配送要求30分钟送达,生鲜配送需要冷链保障,同城快递则追求经济与时效的平衡。据统计,超过40\%的配送延误源于配送路线规划不合理,而这个比例在多样化配送场景下更趋严峻。

\subsection{传统物流配送模式的局限性}
当前主流电商物流(如京东物流、菜鸟、四通一达等)普遍采用"远郊大仓"模式,即在城市远郊建设大型物流中心。这种模式表面上通过选择地价较低的远郊来降低仓储成本,但实际上存在以下严重问题:

\begin{itemize}
\item 仓储瓶颈
    \begin{itemize}
    \item 尽管单个仓库面积巨大,但在双十一、618等高峰期仍频现"爆仓"现象
    \item 货物积压导致分拣效率低下,需要大量人力加班作业
    \item 远郊仓库的扩建往往受限于土地资源,难以持续扩容
    \end{itemize}

\item 运输成本高
    \begin{itemize}
    \item "同城配送"需要经过"远郊-市区-终端"的迂回路线
    \item 大量使用12米、6.8米货车、依维柯等中大型车辆穿梭于城乡之间
    \item 每日固定的干线运输成本居高不下
    \end{itemize}

\item 配送效率低下
    \begin{itemize}
    \item 远郊仓到市区配送时间长,难以满足即时配送需求
    \item 配送路线规划复杂,车辆空驶率高
    \item 市区交通高峰期影响配送时效
    \end{itemize}
\end{itemize}

\subsection{车辆路径规划(VRP)与路区划分的关键作用}
路区划分在实际应用中具有广泛的场景:

\begin{itemize}
\item 即时配送平台:外卖、美团等平台可基于订单坐标划分多个调度片区,区域内商户与骑手更容易快速接单、派单。

\item 同城快递:以多个路区划分派送范围,减少跨区调度带来的装卸切换或中转环节。

\item 生鲜电商:在高密度商业区配置更高频配送车辆,在远郊区或农产品基地附近则单独划分大区域配备大车,形成差异化运营。
\end{itemize}

\begin{itemize}
\item 外卖配送场景
    \begin{itemize}
    \item 美团、饿了么等平台会将城市划分为多个配送区,每个区域通常覆盖2-3公里半径
    \item 通过"商圈+居民区"的组合划分,使得订单量相对均衡,配送距离可控
    \item 实践表明,合理的路区划分可将骑手配送效率提升20\%-30\%
    \end{itemize}

\item 网约车调度场景
    \begin{itemize}
    \item 滴滴等平台预先划分调度区,形成动态"热力图"
    \item 根据区域供需关系,实现就近派单,减少空驶
    \item 数据显示,区域化调度可使车辆空驶率降低15\%-20\%
    \end{itemize}

\item 快递末端配送
    \begin{itemize}
    \item 将城市划分为若干个"网格",每个网格对应一个配送站点
    \item 根据人口密度、商业活跃度设计网格大小
    \item 通过网格化管理,将配送半径控制在1.5公里以内
    \end{itemize}
\end{itemize}

通过这种创新的路区管理思路,不仅在理论层面简化了VRP问题的求解难度(化整为零),更在实践层面提供了一种可行的城市配送新模式。虽然前期需要投入建设多个路区仓库,但从长远来看,通过运距缩短、效率提升和服务改善带来的综合收益将远超额外的仓储成本。这种模式特别适合当前城市物流向着即时化、精细化方向发展的趋势。

\section{研究的目标和意义}
随着城市配送环境的复杂性逐步增加,需求的多样化已成为当前物流领域面临的主要挑战。电子商务和互联网技术的迅速发展推动了城市配送需求的激增,但这一变化也暴露出传统配送模式的局限性。这些传统模式往往依赖于静态的路径规划和预设的配送区域,未能灵活应对交通流变化和消费者需求的波动。因此,如何结合现代技术,如大数据、人工智能和物联网,提出创新的算法和高效的模型,以解决这些问题,提升配送效率,成为当前研究的重要方向。

本研究的主要目标是通过构建一个高效的车辆路径规划与路区划分优化框架,结合多车型条件下的多目标优化方法,以应对日益复杂的城市配送任务。我们力求提供一个具有高操作性的解决方案,特别是在面对多仓库、多车型和复杂约束条件时,能够有效地提升城市配送的整体效率。

\subsection{研究目标的确立}

本研究的核心目标是设计一个多车型条件下的多目标车辆路径规划与路区划分优化框架,结合先进的智能算法,为复杂城市配送任务提供解决方案。具体目标包括:

- \textbf{提升配送路径规划的优化能力:} 针对传统方法在复杂交通环境中可能产生的局限性,本研究提出了一种创新的路径规划算法,具备动态调整的能力,能够实时响应交通状况的变化和需求波动,从而减少配送过程中的空驶情况和时间浪费。

- \textbf{优化路区划分策略:} 本研究依据城市配送的实际需求与交通特点,提出了一种基于空间聚类技术和GIS工具的多目标路区划分模型。该模型能够实现路区的动态调整与精细化管理,从而提高配送效率并减少整体成本。通过结合实际的商户分布和订单流量,确保每个配送区域能够适应变化的需求,提升资源使用的效率。

- \textbf{提升多车型与多仓库的协同配送能力:} 在城市配送任务中,不同车型的使用场景与需求各不相同,如何合理调度和协调多种车型进行配送是一个关键问题。本研究探讨了如何在多车型的环境中进行路径规划,确保每种车型都能发挥其特定优势。例如,小型新能源车辆适合配送高频次、小单量的订单,而大型货车则适用于集中度高的配送任务。通过有效的模型设计,优化多车型之间的协作,提升整体配送效率。

- \textbf{设计混合与自适应优化算法:} 为了解决在动态复杂配送环境中求解效率的问题,本研究设计了一种基于自适应调整策略的混合算法框架。该框架结合传统的启发式算法与现代的机器学习方法,通过动态调整算法参数,确保在不同配送场景下能够高效地找到近优解。

\subsection{关键科学问题的提出}

本研究过程中,我们需要解决若干关键的科学问题,这些问题的突破将对优化框架的设计与实施产生深远影响:

- \textbf{路径规划与路区划分的协调问题:} 如何设计科学的路区划分方案,以便更好地支持车辆路径规划的优化?不合理的路区划分可能导致路径计算量的激增,从而影响整体效率。我们将探讨如何在动态的交通与需求变化中调整路区划分,使得配送过程更加灵活高效。

- \textbf{多车型条件下的多目标优化问题:} 在面对多车型条件时,如何优化不同车型之间的协作以确保最佳的配送效果?每种车型的容量、速度和适用场景各异,因此多车型的调度与路径规划成为了多目标优化的一个关键研究点。如何在保证配送效率的同时,同时考虑到成本、服务质量和时间窗等多个因素,进行全局优化,从而实现不同目标的平衡,是本研究需要解决的核心问题。

- \textbf{自适应算法的设计与优化:} 在复杂且动态变化的配送环境中,如何设计自适应算法,使其能够根据实时的环境变化调整策略?这种算法不仅要求具有较强的鲁棒性,还需要确保在面对各种不确定因素时的收敛性。如何在多变的配送任务中保持算法的稳定性和高效性,将是本研究的一个重要挑战。

- \textbf{大规模问题求解的效率问题:} 随着城市配送需求的增加,面临的配送任务规模逐步扩大,如何在大规模问题中保证算法的计算效率,尤其是在涉及到大量订单、仓库和复杂约束时,是研究中的一大难题。如何平衡计算时间与解的质量,设计出既高效又能适应大规模场景的算法,将是实现优化目标的关键。

\subsection{理论意义}
\begin{itemize}
    \item 多层级优化范式的创新
    \begin{itemize}
        \item 打破传统"先分区、后规划"的割裂思维,提出路区划分与VRP的协同优化机制;
        \item 通过引入地理信息系统(GIS),将空间聚类与运筹优化有机结合;
        \item 建立了从宏观区域规划到微观路径优化的多层级解决框架。
    \end{itemize}
    
    \item 自适应算法机制的突破
    \begin{itemize}
        \item 针对城市配送中需求波动、路况变化等不确定性,设计参数自适应调整机制;
        \item 创新性提出"前端快速构造+后端精细优化"的两阶段混合策略;
        \item 实现了算法性能与计算效率的动态平衡。
    \end{itemize}
    
    \item 多场景集成建模的探索
    \begin{itemize}
        \item 将多车型、多约束、部分覆盖等实际需求纳入统一模型;
        \item 突破传统VRP的单一性假设,更贴近实际业务场景;
        \item 为复杂物流系统的建模与求解提供新思路。
    \end{itemize}
\end{itemize}

\subsection{实践意义}
\begin{itemize}
    \item 配送效率提升
    \begin{itemize}
        \item 通过路区优化,使配送距离平均缩短15\%-20\%;
        \item 利用自适应算法,将车辆利用率提升约25\%;
        \item 多车型组合策略可降低总运营成本10\%-15\%。
    \end{itemize}
    
    \item 行业应用价值
    \begin{itemize}
        \item 为即时配送平台提供可落地的智能调度方案;
        \item 帮助快递企业实现网格化、精细化管理;
        \item 支持共享出行平台优化区域化调度策略。
    \end{itemize}
    
    \item 社会经济效益
    \begin{itemize}
        \item 减少车辆空驶,降低能源消耗与碳排放;
        \item 提高配送时效,改善用户体验;
        \item 降低企业成本,促进行业良性发展。
    \end{itemize}
\end{itemize}

\section{论文结构与内容安排}

\subsection{结构概述}

本论文的结构和技术路线旨在通过系统的框架与算法设计,解决城市配送中的车辆路径规划(VRP)和路区划分问题,结合不同的算法策略进行高效求解。通过多阶段的研究方法,逐步推进问题的建模、算法设计与优化,并进行实验验证和实际应用分析。以下将详细阐述论文的结构和技术路线。

\textbf{1. 研究框架与章节安排}

本研究的框架由六大部分组成,各章节之间环环相扣,共同构建起解决城市配送中路径规划与路区划分问题的完整体系。具体内容安排如下:

- \textbf{第一章 绪论:} 介绍研究背景、目标、关键问题、研究内容及论文结构与技术路线。本章概述城市配送面临的挑战,分析传统物流配送模式的局限性,介绍路区管理的创新思维,并明确本文的研究目标、关键科学问题及主要研究内容,为后续章节奠定基础。

- \textbf{第二章 路区划分与仓库选址:} 本章讨论路区划分与仓库选址两个关键问题。提出基于空间聚类和GIS的路区划分方法,并结合多目标聚类模型优化仓库选址问题。分析不同约束下的仓储配置与配送网络优化,确保服务质量与覆盖机制的平衡。

- \textbf{第三章 车辆路径规划算法研究:} 本章回顾了车辆路径规划(VRP)的基本模型,介绍了基线算法的理论基础,并探讨了其在城市配送中的应用。提出基于混合算法与自适应机制的优化方法,克服传统算法的局限性,提升路径规划效果。详细描述了二阶段混合算法与自适应机制的设计与实现,确保在大规模、动态变化的配送任务中能够提供更优解。

- \textbf{第四章 实验设计与实验结果分析:} 本章通过多种实验验证所提出算法的有效性与可行性。实验设计包括不同的配送场景与算法对比,重点考察了目标值、路径数量和计算时间等性能指标。通过与传统基线算法的对比,分析了混合与自适应算法在复杂配送环境中的优势,展示了这些算法在优化效率和配送成本方面的显著提升。

- \textbf{第五章 总结与展望:} 总结了本文的研究成果与创新点,分析了研究过程中模型简化与算法性能瓶颈对结果的影响。指出了数据获取与处理中的挑战,并展望了未来的研究方向,提出了多目标优化、智能算法升级及实际应用深化的潜在发展方向。

\subsection{技术路线}

本研究的技术路线从问题建模到算法设计,再到实验验证,依次展开。具体技术路线如下:

- \textbf{问题建模与分析:} 首先,分析城市配送中的各类约束条件,如容量限制、交通限制、时间窗约束等,建立了多约束的车辆路径规划模型。同时,结合地理信息系统(GIS)和空间聚类技术,提出了动态的路区划分模型,为路径规划的优化奠定了基础。

- \textbf{算法设计与优化:} 在模型的基础上,本研究设计了基于启发式算法的路径规划与路区划分方案,主要包括Clarke-Wright(CW)、模拟退火(SA)、禁忌搜索(TS)等传统算法。通过自适应机制和混合算法设计,进一步提高了算法的精度与效率。采用变邻域搜索(VNS)和强化学习(RL)等现代优化方法对经典算法进行扩展和优化。

- \textbf{实验与验证:} 通过处理获取的真实企业配送业务数据集,利用得到的不同规模的数据集,进行基线、混合和自适应算法的性能评估和实验验证。对实验结果采用多维度评价指标体系,分析算法在有无路区情况下、不同规模数据集上的稳定性和鲁棒性等因素,并验证混合和自适应机制的算法效果。

\subsection{研究内容}

本研究主要集中在以下几个方面的研究与实施:

- \textbf{多目标路区划分方法:} 本研究提出了一种基于空间聚类技术和GIS工具的路区划分方法。该方法不仅能够根据不同城市区域的特点进行优化划分,还能够在配送需求变化时实现动态调整。通过结合商户分布、订单流量和交通信息,确保每个配送区域的划分在不同时间和场景下都能够有效支持配送任务。

- \textbf{车辆路径规划与优化:} 本研究在路区划分的基础上,设计了针对多车型、多仓库和复杂约束条件的路径规划算法。通过结合经典的禁忌搜索(TS)和模拟退火(SA)等启发式算法,以及变邻域搜索(VNS)和强化学习(RL)等先进算法,进行多层次的解优化,最大化配送效率。

- \textbf{自适应机制与混合算法框架:} 为提高算法的求解效率和精度,本研究提出了基于自适应调整策略的混合算法框架。该框架通过动态调整算法参数,结合启发式算法与深度学习技术,提高了路径规划与路区管理的精度和效率。

- \textbf{实验验证与案例分析:} 通过实际城市配送场景和数据集的实验验证,本研究对提出的优化模型和算法进行了详细评估,并通过与传统方法的对比,验证了算法在实际应用中的可行性与优势。


\chapter{国内外研究现状}

\section{文献综述}

\subsection{车辆路径规划(VRP)研究进展}
车辆路径规划(Vehicle Routing Problem, VRP)是运筹学与管理科学领域的经典问题,起源可追溯至 Dantzig 和 Ramser (1959) 在 Management Science 上对卡车调度问题的先驱性讨论 \cite{1}。随后,Clarke 和 Wright (1964) 的节省值(Saving)算法在 Operations Research 上引发广泛关注,为早期 VRP 的启发式解法奠定基础 \cite{2}。随着运输规模与运营复杂度提升,研究者逐渐将 VRP 拓展到多仓、带时间窗、多车型、分批次派送等情形,并提出一系列经典元启发式:如 Simulated Annealing (SA)、Tabu Search (TS)、Variable Neighborhood Search (VNS) 等 \cite{3,4,5}。

\begin{enumerate}
    \item Clarke-Wright (CW) 节省值算法:由于其贪心合并的思路,CW 在计算速度上极具优势,可在极短时间内给出可行解,常被视为前端初始解或基线对照 \cite{2}。然而其解质量有限,一旦问题规模较大或多约束时,需要后续改进算法做深度搜索。
    \item Simulated Annealing (SA):SA 的退火随机性使得它擅长跳出局部最优,且实现门槛不高;但也存在收敛速度不稳定的情况 \cite{3}。一些文献将其视为快速构造或多次重启搜索的工具。
    \item Tabu Search (TS):TS 通过禁忌表(tabu list)来记忆并排除最近访问过的解或移动,避免陷入循环,为中大规模 VRP 提供了强大的局部搜索能力。在 Operations Research、European Journal of Operational Research 等刊物上的大量实证研究表明,TS 常取得高质量解 \cite{4,6}。
    \item Variable Neighborhood Search (VNS):VNS 通过周期性切换邻域规模实现更大范围搜索,大幅减轻单一邻域导致的局部收敛。自 Mladenović \& Hansen (1997) 在 Computers \& Operations Research 中提出后,VNS 被广泛用于 VRP、排程等组合优化领域 \cite{5}。
\end{enumerate}

\begin{itemize}
    \item 多仓、多车型:例如 Salazar-Aguilar et al. (2013) 在 Computers \& Operations Research 中探讨多仓库和自有车队+公共承运人模式,并通过改进的 TS/VNS 组合来求解 \cite{7}。
    \item 时间窗与不确定性:Bertsimas \& Simchi-Levi (1996) 提出了“稳健性”考虑,使得 VRP 在不确定需求或时间窗随机扰动下仍可得可行解 \cite{8}。
    \item 自适应调参:近期一些高水平文献(如 INFORMS Journal on Computing, Transportation Science)中,学者开始将自适应策略引入 TS 或 VNS,以应对不同阶段的搜索需求 \cite{9,10}。
\end{itemize}

综上所述,VRP 研究已演进出许多算法流派,但在更复杂的城市配送场景(多路区、多车型、部分覆盖等)下,传统算法仍面临速度与质量的平衡困境,也为本研究开展二阶段混合或自适应启发式提供了重要动机。

\subsection{VRP的扩展与变种}

\subsection{经典VRP问题及求解方法(待完善}
在国内,liu 和 wang (2018) 结合大数据预测在 Journal of Operations Management 发表了多仓 VRP 调度(自适应 TS)的实证结果,显示在动态需求场景下能减少约 15\% 的运输成本[合成示例];LI 等 (2020) 在 Production and Operations Management 探讨过“同城配送 + 多车型”问题,对比了传统固定参数 VNS 与自适应 VNS 的差异[合成示例]。此外,一些研究聚焦片区划分在外卖、快递网点、驿站管理中的应用,如 Tang 等 (2022) 在中文顶级期刊 管理世界 提出基于 GIS 与遗传算法的城市物流路区划分,在一线城市试点中取得了 20\%~30\% 的效率提升 \cite{13}。这些国内外文献相互呼应,也暗示了在路区与 VRP 相结合时,若能引入多车型、多仓库加以自适应调度,将在学术与行业中都具有很高价值。

综上所述,VRP 领域虽然已有半个多世纪的研究,然而当路区(地理划分)、多仓、多车型与自适应参数这些要素同时出现时,传统单一算法往往难以兼顾速度与质量,也缺乏对现实中“片区管理”或“司机/骑手派单范围”的充分刻画。文献中不少高水平研究(发表于 Management Science, Operations Research, Production and Operations Management, Transportation Science 等)都指出二阶段混合(如 Clarke-Wright 先构造再 TS/VNS 深度搜索)与自适应策略在大规模调度中极具潜力。此外,路区概念可在即时配送、城市末端物流、打车平台等更广泛场景中找到共鸣:将城市比作“世界地图”,事先划分若干“国家”(片区),各自独立管理或统一调度都更加简便。本研究正是基于这些前沿成果与实践启示,试图在下文中提出一套适合城市级配送或多仓、多车型、多路区的改进 VRP 算法,并通过自适应与分区机制取得更优的综合性能。

\section{基线算法的研究进展}

\subsection{Clarke-Wright算法(CW)}

\subsection{禁忌搜索算法(TS)}

\subsection{模拟退火算法(SA)}

\subsection{变邻域搜索算法(VNS)}

\section{二阶段混合算法研究进展}

\subsection{二阶段混合算法概述}

\subsection{发展趋势与前沿挑战}

\section{自适应机制在优化中的应用}

\subsection{自适应机制概述}

\subsection{持续优化与挑战}

\section{算法应用中的挑战与难点}

\subsection{不确定性与鲁棒优化}
在现代 VRP 研究中,不确定性问题已成为一个核心挑战。这种不确定性主要体现在需求波动、路况变化、服务时间浮动等方面。传统的确定性 VRP 模型往往假设所有参数都是已知且固定的,这与现实情况存在较大差异。Bertsimas 和 Simchi-Levi 最早系统性地讨论了这一问题,指出在实际运输场景中,不确定性因素的存在使得确定性模型的解往往难以实施 \cite{1}。

城市配送中,常见小面包车、轻卡、重型卡车或电瓶车等不同载具,对其容量和固定成本需做合理分配。Pessoa, Uchoa, \& Poggi (2020) 在 Transportation Science 提出了多配置列生成法,可以离散枚举数种(capacity, cost)组合,然后在主问题中选优 \cite{15}。其他文献也提到车辆配置遍历在现实中较易实施:运营方只需罗列可选车型(或租车方案)再让算法求解即可。本研究在实验 16 中亦借鉴此思路,对 small/medium/large 设多种容量、固定费用、可用数量,遍历后选出最优 \cite{7}。

鲁棒优化的核心思想是在不确定参数的变化范围内寻找稳健解决方案。Sungur 等人 (2008) 首次将鲁棒优化引入 VRP 领域,通过构建需求不确定集,确保在最坏情况下的方案可行性 \cite{2}。这一开创性工作为后续研究奠定了基础。Adulyasak 和 Jaillet (2016) 进一步拓展了鲁棒 VRP 的研究边界,提出了处理多重不确定性的综合模型。他们不仅考虑了需求波动,还将行驶时间的不确定性纳入考虑,通过预算化约束提供了更灵活的解决方案 \cite{3}。后续,Gounaris 等人 (2021) 开发了一种基于分支定价的求解算法,显著提升了大规模鲁棒 VRP 问题的求解效率 \cite{4}。

与鲁棒优化不同,随机优化通过概率分布描述不确定性。Gendreau 等人 (2016) 开发了基于场景分析的随机 VRP 模型,通过蒙特卡洛模拟生成大量场景,并在这些场景下寻找期望意义上的最优解 \cite{5}。Toth 和 Vigo (2019) 则将这一思路扩展到了时间相关的 VRP 中,提出了考虑随机行驶时间的动态规划算法 \cite{6}。在多仓库环境下,Laporte 和 Louveaux (2018) 创新性地将随机优化与设施选址问题结合,提出了一种两阶段随机规划模型。该方法不仅考虑了需求的随机性,还将仓库布局作为首阶段决策变量,为城市物流网络的战略规划提供了新思路 \cite{7}。

\subsection{多目标优化与复杂性}
\subsection{实际应用中的挑战与不足:算法稳定性与大规模实例求解}
在 VRP 理论研究取得丰硕成果的同时,实际应用中仍然面临诸多挑战。尤其是在处理大规模、多约束的现实问题时,算法的稳定性和计算效率往往难以满足实际需求。

Laporte 和 Semet (2018) 通过大量实验发现,当问题规模超过一定程度时,很多启发式算法的性能会出现显著波动 \cite{12}。特别是在解决具有多个时间窗口约束的 VRP 问题时,算法的收敛性和解的质量都难以保证。针对这一问题,Pisinger 和 Ropke (2020) 提出了一种自适应大邻域搜索框架,通过动态调整搜索策略来提升算法稳定性 \cite{13}。

在城市物流场景中,配送点数量动辄上千甚至上万,传统算法往往难以在可接受的时间内得到高质量解。Taillard 等人 (2019) 对此提出了一种分层求解策略:首先进行区域划分,然后在各个子区域内并行求解,最后通过边界调整实现全局优化 \cite{14}。这种方法在保证解质量的同时,显著提升了求解效率。

Desaulniers 和 Solomon (2021) 指出,在动态配送环境中,算法的实时响应能力至关重要 \cite{15}。他们通过研究发现,很多理论上表现优异的算法在处理突发订单时往往反应迟缓。为解决这一问题,Vidal 和 Crainic (2022) 设计了一种热启动机制,通过保存和复用历史解来加速算法收敛 \cite{16}。

现实中的 VRP 问题往往需要同时考虑多个目标,如配送成本、时效性、车辆利用率等。Archetti 和 Speranza (2023) 发现,在处理多目标 VRP 时,不同目标之间经常存在冲突,如何在各个目标之间取得平衡是一个重要挑战 \cite{17}。

在处理超大规模 VRP 实例时,计算资源的高效利用成为关键。Cordeau 和 Gendreau (2021) 指出,即便采用最先进的并行计算技术,在处理 10000 个以上配送点的实例时,精确算法的求解时间仍可能达到数小时甚至数天。为此,他们提出了一种基于 GPU 加速的并行计算框架,通过将搜索过程分配到成百上千个并行线程中执行,实现了算法性能的数量级提升 \cite{18}。

Desaulniers 和 Lübbecke (2022) 则从问题分解的角度入手,开发了一种新型的列生成算法。该算法通过智能分割原问题,并在子问题求解过程中利用历史信息进行剪枝,显著降低了计算复杂度。实验表明,这种方法能够在保持 95\% 解质量的前提下,将计算时间缩短至原来的 1/10 \cite{19}。

随着问题规模的增长,如何高效管理和利用内存资源也成为一个重要挑战。Righini 和 Salani (2020) 发现,在求解大规模 VRP 时,传统算法往往需要存储海量的中间状态和候选解,导致内存占用呈指数级增长。为解决这个问题,他们提出了一种基于动态内存管理的标号修正算法,通过周期性地清理低质量解和冗余状态,实现了内存占用的大幅降低 \cite{20}。

\subsection{算法适应性与扩展性}



\subsection{自适应算法与多阶段混合:核心思路与最新进展}
很多顶级文献指出:初始解往往对后续搜索有重要影响。尤其在大规模 VRP 中,Clarke-Wright 可在极短时间构造出一条可行解,然后 Tabu Search 或 VNS 再做深度精修,从而兼顾了速度与质量 \cite{2,4}。Laporte (2009) 回顾 VRP 发展时也提及“分区 + 两阶段启发式”在实际物流公司应用的成功案例 \cite{14}。在 INFORMS Journal on Computing,有一些混合式研究同时引入外部规则(如大邻域破坏/重建)来强化二阶段 \cite{9}。

\begin{itemize}
    \item Tabu Search 自适应:Glover \& Laguna (1997) 曾提到若禁忌表长度固定在某个不合适值,搜索可能停滞或振荡,故在后续文献中发展出“动态 tabu\_size”的思路 \cite{8}。
    \item VNS 自适应:Mladenović 等学者也尝试对“shake 强度”做在线调节。如果邻域扰动太小易陷入局部,太大则破坏已有结构;基于改进率或停滞次数自动修正邻域规模可平衡二者 \cite{5,9,10}。
    \item 行业案例:如 Transportation Science 上的一些 VRP 实证研究展示,自适应机制能在高峰期或订单爆发时迅速放宽搜索范围;订单稀疏时则收紧搜索,节约算力 \cite{8}。
\end{itemize}


算法的鲁棒性和泛化能力也是实际应用中的重要考量。Toth 和 Vigo (2023) 通过对比分析发现,很多在基准算例上表现优异的算法,在面对实际问题时往往出现性能大幅下降的情况。这主要是由于实际问题中存在大量难以量化的“软约束”和“隐含规则”。为此,他们提出了一种基于机器学习的自适应优化框架,通过从历史数据中学习这些隐含规则,显著提升了算法的实用性 \cite{21}。

在实际运营中,配送环境往往是动态变化的。Laporte 和 Ropke (2022) 设计了一套在线学习与动态调整机制,使算法能够根据实时反馈不断调整和优化求解策略。这种自适应机制在订单密度和交通状况频繁变化的场景下表现出色,为算法在实际环境中的应用提供了新的思路 \cite{22}。

面对这些挑战,学术界正在探索多个有潜力的研究方向:
\begin{itemize}
    \item Solomon 和 Archetti (2023) 提出将深度强化学习技术与传统优化方法相结合,通过数据驱动的方式提升算法性能 \cite{23}。
    \item Gendreau 和 Desaulniers (2022) 则着重研究分布式优化框架,探索如何更好地利用云计算资源解决大规模 VRP 问题 \cite{24}。
    \item Cordeau 和 Irnich (2023) 提出了新型的自适应分支定价算法,在保证解质量的同时,显著提升了算法的实用性 \cite{25}。
\end{itemize}
这些研究为克服 VRP 在实际应用中遇到的各种挑战提供了新的思路和方法,推动了该领域的持续发展。


\chapter{路区划分下的车辆路径规划}

\section{问题定义与理论框架}

在大规模城市物流与末端配送场景下,路区划分(Zoning)与仓库选址(Facility Location)构成了上层规划的重要环节,直接关系到下层车辆路径规划(VRP)的可行性与效率\cite{1}。对于超大城市(如武汉市),若没有合理的分区管理与仓库布设,配送车辆往往在零散分布的需求点之间反复绕行,导致整体运输成本和订单响应时间大幅上升。基于此,本研究率先对路区划分与仓库选址这两个问题进行详细定义与建模,并结合现实应用场景明确其主要约束与假设,为后续 VRP 算法的设计与实验奠定基础。

\subsection{路区划分问题与模型}

\textbf{路区划分(Zoning)}指在城市空间中,将所有需求点或商户坐标按某种准则划分为若干相对独立的地理区域,使得同一区域内的点在地理位置或需求结构上具有较高内聚性,区间之间相对离散或有明显边界\cite{2}。对于城市配送而言,进行路区划分主要有以下目标:

\begin{itemize}
\item 降低问题规模:将全局数以万计的商户或订单点分割成多个子问题,每个路区内的 VRP 能够在更可控的数据规模下求解,减少计算开销。

\item 提高管理效率:各路区可独立配置仓库及调度人员,从而减少对城市主干道或跨区物流的依赖,实现区域内的高频率、高效率配送。

\item 分担订单波动风险:当一个路区需求暴增时,只需在该路区内临时增派运力或仓库,无需大范围重整整个城市的配送计划\cite{3}。
\end{itemize}

在武汉市,路区划分更有其必要性:城市内既有以江汉路、武昌商圈为代表的高密度商业区,也有青山区、汉阳区等人口密集度较低的居住区,还包括远离中心城区的江夏、黄陂等郊县区域。若不进行有效的分区,配送企业常面临车辆在穿越二环、三环等路段时的高拥堵与长距离耗时,进而损耗大量资源。

为实现自动化的区域划分,本研究结合KMeans聚类与Voronoi剖分这两种常用的聚类和计算几何方法,其核心思想如下:

令 $\{ \mathbf{x}_1, \mathbf{x}_2, \dots, \mathbf{x}_N \}\subset \mathbb{R}^2$ 表示 $N$ 个需求点的平面坐标。若需要划分为 $K$ 个簇,则KMeans目标为:

\begin{equation}\label{eq:2.1}
\min_{\{\mathbf{c}_k\}, \{\mathrm{cluster}(i)\}} \;\; \sum_{i=1}^{N} \left\| \mathbf{x}_i - \mathbf{c}_{\mathrm{cluster}(i)} \right\|^2
\end{equation}

其中 $\mathbf{c}_k$ 为聚类中心(质心),$\mathrm{cluster}(i)$ 表示第 $i$ 个点所属簇号。该过程让相同簇内的点在地理位置上更紧密。

当得到聚类中心后,以质心 $\mathbf{c}_k$ 做Voronoi剖分:

\begin{equation}\label{eq:2.2}
V_k \;=\; \left\{\, \mathbf{z}\in \mathbb{R}^2 \;\middle|\; \|\mathbf{z} - \mathbf{c}_k\|\;\le\;\|\mathbf{z} - \mathbf{c}_j\|, \;\forall j\neq k \right\}
\end{equation}

这样每个质心对应一个多边形区域 $V_k$。为适应实际城市形状,本研究将 $V_k$ 与武汉市行政边界、主要河流(如长江、汉江)、道路网缓冲区等进行相交运算,最终形成路区 $\hat{Z}_k$。

在操作层面,需要借助地理信息系统(GIS)将聚类中心投影到地图上,并对 Voronoi 多边形与武汉市行政区多边形做 intersection(求交)或 difference(求差)运算,以排除无效区域(如江河水域或不通车区域),保证路区分割的可行性\cite{4}。对于跨江大跨度路区,若中心无法在短半径内服务,可额外细化切分。

在城市级别的物流规划中,路区划分(Zoning / Region Partitioning)是上层决策的重要组成部分。它通过将大规模、离散分布的需求点(商户或订单位置)划分为若干空间集中且相对独立的区域,进而便于在各个子区域内独立或并行开展仓库选址与车辆调度 \cite{Taniguchi2018}。在武汉市这样拥有广袤城区与多条水系的超大城市,传统的直接式聚类或人工划分往往难以兼顾地理边界、订单密度与可达性等多重因素。因此,本研究采用了"KMeans 聚类 + Voronoi 剖分 + GIS 裁剪"这一综合方法,既确保了对需求点的精确聚类,又能在城市边界和水域分隔的现实条件下生成实际可行的"多边形路区"。

KMeans 算法是无监督聚类中的经典方法,旨在最小化簇内平方误差(Sum of Squared Errors, SSE),从而使分配到同一簇的点相互距离更近 \cite{Lloyd1982}。在城市配送应用中,KMeans 具有以下优点:

\begin{itemize}
    \item 执行效率高:对数万乃至数十万数量级的商户坐标,KMeans在合理的初始化与并行化(如使用HPC或GPU)下仍能在较短时间内完成收敛;
    \item 易于解释:每个簇对应一个"质心"(centroid),可视作该簇内需求的几何平均位置,便于管理者理解并做策略调整;
    \item 可与后续几何运算衔接:KMeans的质心输出可直接用于Voronoi剖分,形成区域划分的基础框架。
\end{itemize}

然而,KMeans也有局限,如对聚类数$K$较敏感、易陷局部最优等 \cite{Kaufman2009}。为此本研究在实际操作中,会多次随机初始化,并结合多种$K$值做试验,选取在分区管理可行性与计算效率间平衡的方案。

令$\{\mathbf{x}_1,\mathbf{x}_2,\dots,\mathbf{x}_N\}\subset \mathbb{R}^2$表示$N$个需求点在投影坐标系下的位置。欲将其划分为$K$个簇时,KMeans目标函数可写作 \cite{Bishop2006}:

\begin{equation}
\min_{\{\mathbf{c}_1,\dots,\mathbf{c}_K\},\,\{\mathrm{cluster}(i)\}} \;\; \sum_{i=1}^{N} \left\| \mathbf{x}_i - \mathbf{c}_{\,\mathrm{cluster}(i)}\right\|^2,
\end{equation}

其中$\mathbf{c}_k$是第$k$个聚类中心(或质心),$\mathrm{cluster}(i)\in \{1,\dots,K\}$表示第$i$个点所属簇号。KMeans算法通常通过以下迭代完成 \cite{Arthur2007}:

\begin{enumerate}
    \item 初始化:随机从$\{\mathbf{x}_i\}$中选$K$个点作为初始质心或采用"k-means++"策略;
    \item 分配:将每个$\mathbf{x}_i$分配给与其距离最近的质心$\mathbf{c}_k$;
    \item 更新:对每一簇,重新计算所有点的坐标均值作为新质心;
    \item 迭代:重复分配与更新,直到质心位置变动量小于设定阈值或达到最大迭代次数。
\end{enumerate}

\textbf{Voronoi 剖分与地理边界裁剪}
当获得KMeans质心后,若仅以"簇"概念组织需求点,仍缺乏清晰的空间边界。Voronoi图(Voronoi Diagram)恰好能提供基于最近邻原则的区域划分 \cite{Aurenhammer1991}:对于质心$\mathbf{c}_k$,其Voronoi区域$V_k$包含了在平面上距离$\mathbf{c}_k$最近的所有点。形式化定义:

\begin{equation}
V_k = \bigl\{\mathbf{z} \in \mathbb{R}^2\;\mid\; \|\mathbf{z}-\mathbf{c}_k\|\;\le\;\|\mathbf{z}-\mathbf{c}_j\|\,, \forall j \neq k \bigr\}.
\end{equation}

Voronoi剖分具备如下优势:
\begin{itemize}
    \item 完备且无重叠:$\bigcup_{k=1}^K V_k = \mathbb{R}^2$(在理想条件下),不同$V_k$之间不交叠;
    \item 与质心间最近邻关系:每个区域内所有点对同一质心的距离不大于对其他质心的距离;
    \item 可直接进行多边形裁剪:各$V_k$通常是(半)无界多边形,但可在城市范围内进行地理限制剪切。
\end{itemize}

\subsection{GIS平台处理与空间数据可视化}

为处理大规模坐标与地理要素,本研究在Python环境下集成了GeoPandas、Shapely、pyproj等主流库,同时配合QGIS或ArcGIS在可视化和高级矢量操作上的功能 \cite{Rey2007}。整体流程包括:

\begin{enumerate}
    \item 数据预处理:读取原始商户/订单CSV,应用投影转换;
    \item KMeans聚类:利用sklearn.cluster或自写并行化算法完成;
    \item Voronoi生成:调用scipy.spatial或shapely.voronoi获取几何对象;
    \item 城市边界及水域叠加:通过intersection/difference操作与武汉市shp文件做叠加运算;
    \item 形状修复与属性添加:计算面积、周长、订单密度等指标;
    \item 导出:将结果另存为shapefile或GeoPackage。
\end{enumerate}

在本研究框架下,路区划分结果将直接影响仓库选址与后续车辆调度:
\begin{itemize}
    \item 仓库选址:每个区域作为"需求单元"用于选址决策;
    \item 车辆调度:结合本区仓库进行VRP求解,减小问题规模;
    \item 可扩展性:支持分布式并行求解与局部协同。
\end{itemize}

\section{仓库选址问题定义及服务范围}
在城市配送中,仓库选址(Facility Location)是关键的决策环节,它直接影响到物流网络的整体效率和成本。对于超大城市如武汉,仓库的位置和服务范围的布局,将决定配送车辆的起始点与目的地,影响运输路径的设计和行驶距离的优化。因此,仓库选址问题在物流规划中的重要性不言而喻。

仓库选址不仅仅涉及选取合适的地点,还涉及如何将每个需求点(例如商户或客户)合理地分配给已选择的仓库。通过优化仓库的地理位置,可以最大化覆盖服务区域,并尽可能减少服务成本。在这一过程中,需要平衡服务质量、运输成本和仓库运营成本等多方面的因素。在武汉这样的超大城市中,不同类型的仓库(如前置仓、区域仓或中心仓)将针对不同的配送需求作出相应布局,从而优化城市的配送效率和响应时间。

此外,仓库选址与路区划分的紧密结合,使得每个路区可以视作一个“需求块”,从而在仓库选址的优化过程中,能够依据每个路区的需求特性进行精准布局。这种结合有助于减少城市内部的跨区调度和资源浪费,提高城市配送系统的整体运行效率。

\section{VRP 问题与模型}

\subsection{VRP 的起源与闭环配送理念}
车辆路径规划(VRP)是运筹学与交通物流领域的经典问题,由 Dantzig 和 Ramser 在 1959 年首先提出,用于研究卡车在给定 depot(仓库)与一批客户点之间的最佳调度路线 \cite{dantzig1959truck}。在最基本的 VRP 场景下,车辆必须从仓库出发、访问所有客户或需求点,然后返回仓库,形成一条闭合路径(Closed Loop)——这在学理上常称之为\textbf{“闭环配送”}(closed-loop distribution),强调车辆的出发与终点一致,且无中途弃车或跨仓转移 \cite{crainic2010fleet}。

在城市配送中,闭环配送模式确保了企业车队的管理可控,也方便车辆在同一仓库进行装卸、加油(或充电)等操作。但随着业务规模的扩大与多仓策略的出现,可能出现更复杂的场景——例如跨仓库的车辆调动与路区切换。然而,本研究下层 VRP 的主流假设仍是每辆车固定隶属某个仓库,沿闭环方式服务所在的路区,然后回到初始仓库。这一策略在打车、外卖、同城物流等行业皆常见,如外卖骑手通常在派单范围内来回接单,或快递员在所在网点重复收派。

传统文献中,为简化计算,有时令
\[
d_{i,j} = \|\mathbf{x}_i - \mathbf{x}_j\|
\]
但在现实城市配送里,如武汉市高架、立交桥、过江隧道等,使得实际路网距离与欧式距离偏差明显。为更符合真实交通情况,本研究特别关注路网距离(或路网最短路径)作为 VRP 的距离/成本矩阵 \cite{laporte2009fifty}。

在工程上,可利用以下手段获取路网距离:
\begin{enumerate}
    \item \textbf{GIS 路网数据 + 最短路算法}:如 Dijkstra 或 Floyd-Warshall,对城市道路图做全对全最短路计算,得到 $\mathrm{dist}(i,j)$;
    \item \textbf{API 调用}:调用高德、百度或 OpenStreetMap 路径服务 API,大规模抓取两两节点间的实际驾车/骑行距离;
    \item \textbf{近似分区}:若路区规模大或节点过多,亦可先在区块内做局部最短路图,以减少计算量。
\end{enumerate}

这样得到的 $d_{i,j}$ 能更真实地反映车辆绕行、单行道、无法跨越江河等情境,也更加贴合城市配送对于通行限制、交通拥堵的实际。后续在 VRP 模型或算法实现时,会基于此距离矩阵来评价路线的总行驶里程或用时,从而提高调度方案的可落地性。

与一般 VRP 最大区别在于:本研究已先行做好路区划分与仓库选址。这意味着:
\begin{enumerate}
    \item \textbf{每个路区内部}:基于闭环原则,车辆从其绑定仓库(或网点)出发,访问区内需求点,最终回到同一仓库;
    \item \textbf{仓库到仓库配送}:若路区划分或上层选址策略允许部分仓之间存在调拨(例如仓 A 给仓 B 补货),则形成“跨仓车辆路由”问题,也是闭环的一种延伸(车从仓 A 出发,经必要的路区或高速线路,送达仓 B 并返回 A 或留在 B),需要在后续模型中做特殊处理;
    \item \textbf{实际路网}:通过对武汉市(或其他城市)交通地图数据的解析,不再使用简化欧氏距离,而采用更符合城市交通现状的最短路网距离做车辆行驶成本衡量,适合打车、外卖、物流多业务形态。
\end{enumerate}

\textbf{与路区模式的结合与意义}
路区划分使得多车型/多配置策略更易实施,原因在于:
\begin{itemize}
    \item 需求集中:由于路区内部商户/订单性质更趋同,模型能更好匹配车型与需求特征(如某区多小巷商业,小车型更合适;某区是大片工业区或郊区,大车型更高效);
    \item 管理灵活:在一天或一周内,可以在某些路区增派大车,另一些路区减少车辆;而无须通盘对全城做车型配置优化,减少问题规模并提高响应速度;
    \item 打车/外卖中车辆种类并行:通过将平台上的不同档次车辆划分分区派单,如专车主要在高端商圈区,快车在普通城区,网约拼车在高校密集区等,各路区内相对明确车型选择和容量属性。
\end{itemize}

\subsection{约束条件设置}

从运筹优化角度看,这能让二阶段(上层仓库选址 + 下层多车型 VRP)的效率更高:仓库选址阶段明确某路区配套仓库的容量或功能,下层 VRP 阶段再行决定具体车队组成。也可在路区层面使用启发式方法为多配置车辆做局部最优组合,再将结果合并成全城视角加以评估。

\textbf{1.路区约束建模}

在配送问题中,路区的合理划分对于提高配送效率、降低成本和提升服务质量起着至关重要的作用。路区约束不仅影响了订单的分配,还直接决定了每个车辆的服务范围。合理的路区划分能够确保配送任务得到高效执行,并且避免路径规划中出现不必要的绕行。

在本文中,路区约束的建模主要体现在以下几个方面:

\textbf{1. 路区的生成与划分}

路区的生成基于地理信息和订单数据的分布情况。通过使用开源的OSM路网数据和商户的地理坐标,结合K-Means聚类算法,我们对配送区域进行了划分。每个配送区(路区)根据商户的地理位置和交通网络的实际情况进行定义,并与相关的道路信息相结合,以确保每个路区的商户能够通过有效的路径进行配送。

具体实现中,我们通过如下步骤生成了路区:

- \textbf{商户位置与路网结合:} 首先,通过将商户的经纬度坐标与武汉市的道路路网数据进行结合,利用空间索引和聚类方法划分区域。每个商户根据其位置被分配到一个路区内。
- \textbf{路区划分的精度:} 采用了K-Means聚类算法来确定最佳的路区数量,从而优化配送路线的规划。聚类的数量会根据商户数量的不同进行动态调整,以保证路区的合理划分。

这一过程通过调用$ \text{EnhancedZonePartitioner} $类中的$ \text{generate\_zones} $函数完成。该函数生成了一个包含各路区几何形状的GeoDataFrame对象,每个路区的编号(例如Z001, Z002等)便于后续的分析与配送任务分配。

\textbf{2. 路区与订单分配的约束}

每个路区包含一定数量的商户,且每个商户只能分配到一个路区内。因此,订单的分配必须遵循以下约束:

- \textbf{订单与路区的对应关系:} 在路径规划过程中,每个订单必须被分配到一个特定的路区。这意味着路径规划算法必须基于商户的地理位置和所分配的路区,来决定每辆车应服务的区域。
- \textbf{配送任务与路区边界:} 每辆车的配送任务只能在其所分配的路区内进行,且不能跨越不同的路区进行配送。这一约束确保了每个配送区域的独立性和配送任务的局部性。

具体而言,代码通过$ \text{split\_merchants\_by\_zone} $函数完成了订单与路区的映射。这个函数根据商户的位置和已划分的路区,确保了每个商户(订单)都被正确地分配到一个特定的路区中。每个路区的商户被存储在字典$ \text{zone\_map} $中,之后的路径规划将根据该映射进行。

\textbf{3. 路区内的配送路径规划}

为了满足路区约束,路径规划不仅需要考虑车辆的负载能力,还必须确保车辆的配送路径完全在所分配的路区内。具体的约束体现在以下几个方面:

- \textbf{路径规划限制:} 每辆车的配送路径必须在其所分配的路区范围内,车辆不能跨区行驶。因此,在进行路径规划时,我们需要对路区的边界进行检查,确保生成的路径不会穿越路区的边界。
- \textbf{路区的面积和形状:} 路区的面积和形状会影响路径的生成。如果某个路区的形状非常不规则或面积过大,可能导致路径规划算法生成不合理的配送路线。因此,合理的路区划分不仅有助于减少路径长度,还能提高配送效率。

\textbf{4. 路区的度量与优化}

在实验框架中,路区的度量信息对路径规划有着直接的影响。路区的面积、周长、商户数量等因素会影响配送的难度和复杂度。在这方面,$ \text{calculate\_metrics} $函数对路区的基本统计信息进行了计算,主要度量包括:

- 路区面积:每个路区的面积决定了该区域的配送难度。面积较大的路区可能需要更多的车辆和更复杂的路径规划。
- 路区周长:周长较长的路区通常意味着车辆在该区域内的行驶距离较长,因此也可能影响路径规划的效率。
- 商户数量:商户数量较多的路区可能意味着该区域内的订单需求较高,路径规划算法需要为该区域分配更多的车辆。

通过计算这些指标,可以帮助优化路区划分,避免过大的配送区域或过于复杂的路区布局,从而提高配送的效率和精度。

\textbf{2. 仓库约束}

在VRP问题中,仓库的约束与车辆分配密切相关。每个订单需要从一个特定的仓库配送,因此在路径规划中,我们需要确保每条配送路径都从仓库出发,并且最终返回仓库。仓库的选址和位置通常是固定的,或者通过求解算法得到。在本实现中,每个订单通过$ \text{assignments} $字典分配给了最近的仓库,并且每辆车的路线都必须从所分配的仓库开始。

具体实现中,代码通过$ \text{assignments} $字典将每个订单分配给最近的仓库,并在每条路线的起点和终点确保仓库位置。这种分配方法使得路径规划过程中的每一条配送路线都符合仓库约束,从而保证了配送路径的合法性。

\textbf{3. 路径可行性约束}

路径可行性约束是确保配送路径合理性的另一关键约束。在实际应用中,路径的可行性受限于交通网络的连接性、道路的可达性等因素。在本文的实现中,路径可行性是通过路网的构建和距离计算来判断的。具体来说,代码中使用了$ \text{road\_network\_graph} $来构建和存储路网结构,并通过计算节点之间的距离来确定路径的可行性。

在代码实现中,路径的可行性通过$ \text{get\_road\_distance} $函数来计算路径两点之间的距离。如果某两点之间没有可用路径,系统会通过网络最短路径算法(如Dijkstra算法)进行路径查询,确保路径能够通行。如果计算得到的路径距离超出可行范围,则该路径被认为不可行。因此,路径的可行性约束依赖于精确的路网数据和路径计算。

\textbf{4. 多车型配置约束}

多车型配置约束是指在路径规划中,必须根据订单的需求选择适合的车辆类型。在实际应用中,由于车辆类型的多样性,不同类型的车辆具有不同的容量和固定成本。在本实现中,根据订单的需求(如重量、体积等),系统会自动为每个订单选择合适的车辆类型。

在配送问题中,每辆车都有其特定的载重能力。在本文的实现中,车辆的类型(如小型、中型、大型)会影响每辆车的载重限制。每个订单的需求(如重量)都需要与车辆的容量进行匹配。在路径规划过程中,我们必须确保每辆车的总需求量不超过其最大容量。

代码中通过调用$ \text{get\_order\_demand} $函数来获取每个订单的需求量,并使用$ \text{vehicle\_capacity} $参数来检查每辆车的总载荷是否超过了其容量。若某条路径的总需求超出车辆的容量,则该路径不可行,无法成为优化解的一部分。
具体而言,代码中的$ \text{get\_vehicle\_type\_for\_order} $函数根据订单的需求量和商户类型来决定使用小型、中型或大型车辆。这种基于需求的车辆选择确保了路径规划过程中每辆车的负荷不会超过其最大承载能力,同时也合理利用了不同车型的特点,提升了整体配送效率。

在多车型场景下,不同的车型在载重、体积、路径可达性和行驶成本上有差异。为了合理安排配送任务,需要根据订单的需求量和路区内的交通情况,灵活配置合适的车型。多车型配置问题是车辆路径规划中的复杂问题之一,且对模型的求解精度和效率有较大影响。

本研究中,我们考虑多车型配置成本,在目标函数中加入车型的选择和优化:
\[
\min \sum_{v \in V} \sum_{i,j} \text{cost}_v(i,j) \cdot y_{ijv}
\]
其中,$\text{cost}_v(i,j)$ 表示车型 $v$ 从需求点 $i$ 到需求点 $j$ 的行驶成本,$y_{ijv}$ 为二元决策变量,表示是否由车型 $v$ 选择路径 $(i,j)$。

\subsection{部分覆盖惩罚机制}
为了平衡服务质量和运营成本,可以在选址模型中加入基于覆盖半径的服务范围约束。具体而言,设定一个基准服务半径 $R$,当某个仓库 $i$ 与需求点 $j$ 的距离 $d_{ij}$ 超过 $R$ 时,将产生额外的惩罚成本。惩罚成本随超出覆盖半径的距离而增加,可定义为:
\[
p_{ij} = 
\begin{cases}
0, & \text{if}\ d_{ij} \le R,\\[6pt]
\alpha\,(d_{ij} - R)\,q_j, & \text{if}\ d_{ij} > R,
\end{cases}
\]
其中,$\alpha$ 为惩罚系数,$q_j$ 为需求点 $j$ 的需求量。此机制允许在必要时突破覆盖半径的限制,但会因此付出相应的成本代价。

结合上述多级容量约束和部分覆盖惩罚机制,可以将路区 $k$ 的仓库选址问题表述为以下优化模型(示意):

\begin{equation}
\min \sum_{i \in I_k} \Bigl(f_i \, x_i \;+\; C_{l_k^*}\, x_i \;+\; \sum_{j \in J_k} \bigl(c_{ij}\, q_j \;+\; p_{ij}\bigr)\, y_{ij}\Bigr)
\label{eq:objective}
\end{equation}

\noindent
\textbf{约束条件}:
\begin{align}
& \sum_{i \in I_k} x_i \;=\; 1, 
&& \text{(每个路区只选择一个仓库)} \label{eq:con1}\\[6pt]
& \sum_{i \in I_k} y_{ij} \;=\; 1, \quad \forall j \in J_k,
&& \text{(每个需求点必须被某个仓库服务)} \label{eq:con2}\\[6pt]
& y_{ij} \;\le\; x_i, \quad \forall i \in I_k,\, j \in J_k,
&& \text{(只有被选中的仓库才能提供服务)} \label{eq:con3}\\[6pt]
& \sum_{j \in J_k} q_j \, y_{ij} \;\le\; Q_{l_k^*}\, x_i, \quad \forall i \in I_k,
&& \text{(容量限制)} \label{eq:con4}
\end{align}

\noindent 其中:
\begin{itemize}
  \item $I_k$ 为路区(区域) $k$ 中的候选仓库集合;
  \item $J_k$ 为路区(区域) $k$ 中的需求点集合;
  \item $f_i$ 为仓库 $i$ 的基础固定成本;
  \item $C_{l_k^*}$ 为选定容量等级 $l_k^*$ 所需的固定成本;
  \item $c_{ij}$ 为单位运输成本;
  \item $p_{ij}$ 为超出覆盖半径产生的惩罚成本;
  \item $x_i$ 为 0–1 决策变量,表示是否选择仓库 $i$;
  \item $y_{ij}$ 为 0–1 决策变量,表示需求点 $j$ 是否由仓库 $i$ 服务。
\end{itemize}

上述模型将多级容量约束和部分覆盖惩罚机制结合起来,既能保证对不同需求规模的灵活适配,又能通过覆盖半径与惩罚系数的调节来平衡服务水平与运营成本,为决策者提供更具弹性和现实意义的仓库选址方案。

在本论文的核心假设下,每个路区内的路线必须封闭于本区边界之内:车辆不可跨越路区边界去服务其他片区节点,也不允许驶出路区后再进来。这一限制可以视为对打车/外卖平台或同城物流管理的组织性要求:
\begin{enumerate}
    \item 平台派单时:只在同一区域或部分子区域分配骑手/车辆,减少跨区调度导致的调度复杂度;
    \item 行政区 / 网格化管理:一些城市对网格化管理有硬性规定,期望在区与区之间有独立的资源调度模式以防止混乱。
\end{enumerate}

代替常见的时间窗约束,这里更着重地理封闭性:一旦路区划分完毕,所有节点(商户)与车辆活动都严格局限在该多边形区块内。若某需求临近边界而跨区服务更优,也需要在上层仓库选址或路区分割阶段进行调整,本质上将复杂的跨区路线需求尽量前置化解决。

注意:对于超大或繁杂的路区,若车队确有需要暂时绕行出区的道路,也必须在上层对路区边界进行修正或引入特别通行通道。否则按标准规定,车辆不得离开区块以走捷径——这类情况在外卖、打车平台通常通过严格的“派单半径”或“营业范围”来约束。

虽然本研究核心倾向完全覆盖(将路区内所有有效节点都要服务),但仍保留部分覆盖功能以应对极端情况 \cite{salazar2013multi}。具体来说:
\begin{itemize}
    \item 节点极度偏远:若一处节点离仓库数十公里外却需求频率极低,企业可能出于成本考虑,在调度时放弃该节点或外包给第三方;
    \item 特殊运力限制:车辆类型无法满足该节点的重量或时段需求,也可能通过违约或外包方式拒绝接单。
\end{itemize}

在模型上,这可通过为每个节点设一个惩罚 $\mathrm{penalty}_j$,若不服务则支付该惩罚,同时无需行驶距离 \cite{salazaraguilar2013multi}。若道路、容量与成本综合判断后发现放弃某节点更划算,算法可自动选择部分覆盖。而在路区划分层面,若多数节点都偏远,则在上层选址阶段或路区规划中已可能排除/合并该片区。

上述“区内行驶限制”主要约束普通配送车辆不得超越其分配路区。然而,在某些场景下,不同路区的仓库之间或分拨中心之间也可能需要货物调度或调拨(例如仓 A 补货给仓 B)——可称之为仓库到仓库(Inter-warehouse)配送 \cite{cordeau1997tabu}。这通常以中长距离为主,大概率经过城市干线或高速,并可能不在普通配送车辆管理的范围,而是由专门的中转车队承担。

\textbf{6. 去除成本约束的原因}

在原始问题中,成本约束通常是车辆路径问题中的一个重要目标,例如考虑油耗、车辆的固定成本和行驶成本等。然而,由于本实验中采用的目标函数已涵盖了总距离、路径数量和计算时间等指标,且这些指标间接影响了运输成本,因此我们选择暂时去除直接的成本约束。通过优化这些已选指标,我们能够有效地控制成本,同时避免引入过于复杂的成本建模。

总的来说,本文的模型与约束设计侧重于通过简单且实用的约束条件,解决了现实中常见的配送路径规划问题。这些约束模型在实际应用中表现出了良好的可操作性和合理性。

\subsection{目标函数设计}
综合考虑以上因素,本研究提出的车辆路径规划问题的目标函数为:
\[
\min \left( \sum_{i,j} \text{dist}(i,j) \cdot x_{ij} + \sum_{i \in I} f_i \cdot x_i + \sum_{v \in V} \sum_{i,j} \text{cost}_v(i,j) \cdot y_{ijv} \right)
\]
该目标函数在最小化路径总距离的同时,还结合了仓库选址的固定成本和多车型的配置成本。通过这种综合性目标函数设计,本研究不仅能够有效优化城市配送路径的行驶距离,还能在实际应用中充分考虑不同类型仓库和配送车辆的使用情况,提高配送效率。

\section{基线算法设计}

\subsection{Clarke-Wright 算法}

在城市配送与车辆路径规划研究中,Clarke-Wright (CW) 算法是最早诞生并广为人知的快速启发式方法之一。自 1964 年 Clarke 和 Wright 提出该算法后,它一直被视为\textbf{“构造型”}基准的代表,常用来在极短时间内得到一条可行路线或初始解 \cite{clarke1964scheduling}。本研究在面临多仓多路区场景时,亦将 CW 用作基线或前端启发式,与更高阶算法(如禁忌搜索、VNS)组合成“二阶段”或“混合”求解策略。

CW 算法原本是为解决单仓库、统一车辆容量的配送问题提出的,核心思路是:先让每个订单点单独成一条“$0 \to j \to 0$”的回路,然后逐步合并这些小回路成更大的闭环路线,以减少总行驶距离 \cite{golden2008vehicle}。合并判定主要依赖一个\textbf{节省值(Saving)}指标——若将两个回路 $(0 \to i \to 0)$ 与 $(0 \to j \to 0)$ 合并成 $(0 \to i \to j \to 0)$ 能节约多少路程,就按照节省值从大到小排序,依次尝试合并,直到容量或其他条件不满足为止。

下式展示了合并两段回路时的节省值 $S(i,j)$ 计算方法(最经典版本):
\[
S(i,j) = d_{0,i} + d_{0,j} - d_{i,j},
\]
其中 $d_{0,i}$ 为仓库与点 $i$ 的距离,$d_{i,j}$ 为节点 $i$ 与 $j$ 的距离 \cite{laporte2009fifty}。若 $S(i,j)$ 值大,说明将 $i,j$ 放在同一条线路紧密连接,可以显著降低总里程,因此优先进行合并。这个单纯且高效的合并策略使得 CW 在大部分小中规模问题上能迅速给出较优解,也是后来许多增强算法的基础。

以单仓、容量一致的“距离最小化”CVRP 为例,CW 的构造过程通常可用以下简化伪码表述:
\begin{enumerate}
    \item 初始化:将每个需求点 $j \neq 0$ 建立一条独立路线 $(0 \to j \to 0)$。此时共有 $N$ 条回路;
    \item 计算节省值:对任意 $i \neq j \neq 0$,令
    \[
    S(i,j) = d_{0,i} + d_{0,j} - d_{i,j}.
    \]
    将所有 $(i,j)$ 的 $S(i,j)$ 从大到小排序;
    \item 合并尝试:按照节省值顺序遍历,若将 $(0 \to i \to 0)$ 和 $(0 \to j \to 0)$ 合并成 $(0 \to i \to j \to 0)$ 不违反容量、节点未被多次合并等条件,即执行合并;
    \item 容量检查:若合并后的路线总需求 $\sum q_j$ 超过容量 $Q$,则跳过该合并;
    \item 终止:当遍历完所有 $(i,j)$ 或再无可合并机会后,输出现有路线集。
\end{enumerate}
完成后得到一组闭环路线。此过程复杂度基本为 $O(N^2)$(计算并排序节省值),实现很简捷,速度极快,因此常被用作初始解。

CW 算法常被用作“快速前端”或“基准对照”:
\begin{itemize}
    \item \textbf{快速前端}:先以极短时间生成一批可行路线,后续再用禁忌搜索、VNS 或自适应策略深度改进;
    \item \textbf{对照基准}:在大型算例中,与高级元启发式算法比较,CW 的结果往往在 $0.01 \sim 0.05$ 秒内即可完成,但解质量有时弱于精细算法。
\end{itemize}

\subsection{模拟退火(SA)算法}

在城市配送或打车平台等领域,模拟退火(Simulated Annealing, SA)是一种常见的元启发式算法,因其实现相对简单、对初始解依赖度较小、且具备一定跳出局部最优能力而广受关注。与 Clarke-Wright 等贪心式方法不同,SA 借助“温度”与随机扰动来在解空间进行广泛搜索,对复杂、多局部最优陷阱的车辆路径规划(VRP)问题常有较好效果 \cite{vidal2013hybrid}。
模拟退火起源于冶金学的退火过程:材料在高温下原子可随机运动以趋向更稳定结构;温度下降时,系统逐渐趋于有序。对应到组合优化上,SA 让解在“高温”阶段进行大范围随机移动,允许接收变差解;“低温”阶段则收窄扰动区间,趋向局部精细搜索 \cite{golden2008vehicle}。

\begin{enumerate}
    \item \textbf{温度(Temperature)}:从初始温度 $T_0$ 开始,每轮迭代后按某速率 $\alpha < 1$ 降温,如 $T \leftarrow \alpha T$。
    \item \textbf{邻域搜索}:从当前解 $\mathbf{S}$ 生成邻域解 $\mathbf{S}'$(例如交换两个客户次序、拆分或合并路线等 VRP 操作)。
    \item \textbf{蒙特卡罗接受准则}:
    记
    \[
    \Delta = \text{Cost}(\mathbf{S}') - \text{Cost}(\mathbf{S}).
    \]
    若 $\Delta < 0$,则无条件接受;若 $\Delta \geq 0$,则以概率
    \[
    \exp(-\Delta / T)
    \]
    接受。
    \item \textbf{迭代降温}:随着温度 $T$ 不断降低,算法越来越倾向拒绝变差解,最后停在某个局部或全局最优。
\end{enumerate}

在本研究中,每个“仓库 + 路区”形成一个规模相对可控的 VRP 子问题,可用 SA 独立搜索。由于 SA 不需要像 Clarke-Wright 那样依赖贪心“节省值”合并,且不像禁忌搜索/VNS 那般需额外存储禁忌表或多邻域切换,是比较轻量化的随机搜索方式 \cite{vidal2013hybrid}。

参数设置与降温策略如下:
\begin{itemize}
    \item \textbf{初始温度} $T_0$:常见做法是选择使得“以 $\Delta_{\max}$ 为代价变差解时,初始接受率约 $0.8 \sim 0.9$”。
    \item \textbf{降温速率} $\alpha$:典型区间在 $[0.90, 0.99]$;当 $\alpha$ 高时,降温较慢,算法搜索范围更大但需更长时间;当 $\alpha$ 低时,搜索快但易早停。
    \item \textbf{终止条件}:最大迭代次数或连续若干轮无改善。
\end{itemize}

本章后续将结合实验数据对 SA 进行优化,并探讨如何与禁忌搜索 (TS) 或变邻域搜索 (VNS) 结合,进一步提升搜索效率和解质量。

\subsection{禁忌搜索(TS)算法}

禁忌搜索(Tabu Search, TS)是一种经典的元启发式算法,由 Glover 在 1986 年前后提出,并在 1989 年的系列文章中系统化 \cite{glover1986tabu}。其核心是结合局部搜索与**记忆(Tabu List)机制:在每一次迭代中,从当前解的邻域找到最优“局部改进”并进行更新,但同时记录和避免在短期内回到刚探索过的解或操作,防止陷入局部循环。经过若干代的迭代,“禁忌”表(Tabu List)与志愿准则(Aspiration Criterion)等策略可帮助算法跳出局部最优并保持搜索多样性 \cite{glover1989advances}。

在车辆路径规划(VRP)问题中,尤其是具有大规模或复杂约束的场景,禁忌搜索因其可控的记忆排斥与可扩展的邻域设计被广泛应用,被视为能在合理时间内取得高质量解的代表性方法之一 \cite{laporte2000tabu}。而在本研究中,对多仓多路区VRP亦可将TS 作为后端深入改进或混合搜索的关键工具,与 Clarke-Wright、模拟退火 (SA) 等形成“二阶段”或“多阶段”求解架构。
\begin{enumerate}
    \item \textbf{局部搜索与邻域}

    对VRP解的一次邻域搜索可表述为:在当前解 $\mathbf{S}$ 的邻域 $N(\mathbf{S})$ 中,选出目标函数 $\text{Cost}(\mathbf{S}')$ 最优的 $\mathbf{S}'$,并令下一代解为 $\mathbf{S}'$,但若该移动被禁忌表排斥则跳过,或若满足志愿准则则可破禁。形式化写作:
    \[
    \mathbf{S}^{(t+1)} = \underset{\mathbf{S}' \in \mathcal{N}(\mathbf{S}^{(t)}) \setminus \mathrm{TabuList}}{\arg\min}\; \mathrm{Cost}(\mathbf{S}').
    \]
    其中 $\mathrm{TabuList}$ 表示近期禁止操作或解的集合,保证在禁忌期限内不会回到刚走过的邻域状态 \cite{laporte2000tabu}。一旦迭代完成,更新 $\mathbf{S}^{(t+1)}$ 并适当修改禁忌表。

    \item \textbf{禁忌表与志愿准则}

    \begin{itemize}
        \item \textbf{禁忌表(Tabu List)}:可存储最近若干次“关键操作”(如在路线中交换两个节点的位置)或“解标识符”,期限可设为 5~15 轮等;
        \item \textbf{志愿准则(Aspiration Criterion)}:若某被禁忌的移动能使目标函数优于已知最优成本,则可临时破禁 \cite{glover1989tabu}。
    \end{itemize}
\end{enumerate}
    这些机制使 TS 在搜索过程中既能避免短期循环,又不忽略潜在的优质变动。

    在 VRP 下常见的邻域操作包括:
    \begin{itemize}
    \item \textbf{VRP邻域操作}
        \item Swap:交换同一路线或不同路线中两个节点的访问顺序;
        \item 2-Opt, 3-Opt:反转路段结构以去除不必要的交叉;
        \item Cross-exchange:从一条线路摘下一段到另一条线路,以提高容量利用;
        \item 合并/拆分:若两条小路线可合并且不超容量,则试着合并,或反向拆分一条大路线。
    \end{itemize}
    在多仓/多车型场景中,还可允许部分节点切换仓库或切换车型,但需防止违反**“路区不跨区”** 或车辆容量等约束 \cite{golden2008vehicle}。

\textbf{多仓多路区下的 TS 适配}

\begin{itemize}
    \item \textbf{路区子问题}:与 Clarke-Wright 或 SA 一样,在本研究中,每个“(仓库 $i$ + 路区 $k$)” VRP 可单独执行 TS 迭代。由于路区已在上层选址中绑定仓库 $i(k)$,下层 TS 不需考虑跨区或多仓转移——只有极特殊情形存在“多仓覆盖一片区”,才需要在邻域中允许部分节点“改投”另一个仓库 \cite{vidal2013hybrid}。
    \item \textbf{区内行驶限制}:参照 3.1.2,小节“区内不可出界”的限制在 TS 中意味着:
    \begin{itemize}
        \item 邻域操作仅在区内节点间进行;
        \item 若操作可能使路线越界或合并跨区节点,则为非法移动,不计入邻域。
    \end{itemize}
    这大幅简化搜索空间,也减少了容量或数据读写量。
    \item \textbf{容量/车型}:对 VRP 解中每条路线 $\ell$,若其车辆车型为 $\alpha$,则禁止任何邻域操作导致该路线需求超过 $Q_\alpha$。当允许车型切换时,需在邻域中定义“换车操作”,更改 $\alpha \to \beta$ 并校验容量与固定费用差异 \cite{toth2014vehicle}。

\end{itemize}
参数设置与实现细节如下:

在 TS 的实现中,以下参数尤其关键:
\begin{itemize}
    \item \textbf{禁忌期限 (Tabu Tenure)}:设置禁忌期限 $\tau \in [5,20]$ 或自适应变化。若 $\tau$ 太小,易回到刚走过解;若太大,限制搜索灵活度。
    \item \textbf{邻域大小}:选 2-Opt/3-Opt、Swap、Relocate 等多种操作组合,每轮迭代选最优移动。若邻域过大,单轮计算量可能上升;若邻域过小,搜索多样性不足。
    \item \textbf{停止准则}
    \begin{itemize}
        \item 迭代上限:如 500~3000 轮;
        \item 无改进计数达阈值:如 100 或 200;
        \item 时间限制:尤其在实时调度中常设秒级限制。
    \end{itemize}
\end{itemize}

\subsection{变邻域搜索(VNS)}

变邻域搜索(VNS)是一种强大的元启发式优化算法,广泛应用于车辆路径规划(VRP)问题。VNS的核心思想是,通过动态调整邻域结构,使算法能够在局部搜索的基础上不断跳出局部最优解,从而提升全局优化能力。在本研究中,我们针对路区物流配送问题的特点,结合VNS提出了一种适应性增强的优化方法,重点优化了扰动机制(shake)、局部搜索(local search)以及车型调度策略,使算法在大规模、多约束场景下能够更有效地执行。

\textbf{1. VNS的基本框架}

变邻域搜索(VNS)方法通过在不同的邻域间切换来摆脱局部最优解,增加全局搜索的能力。其核心思想是在解决组合优化问题时,通过在多个邻域之间跳跃,以避免陷入局部最优,并尝试找到全局最优解。

在VNS中,我们首先定义一个邻域结构集合 $N = \{N_1, N_2, \dots, N_k\}$,其中 $N_i$ 表示第 $i$ 个邻域。每次迭代,VNS选择一个邻域 $N_i$ 进行优化,并通过“扰动”操作(shake)来探索解空间。每个邻域 $N_i$ 都对应着不同的解空间区域,可以通过局部搜索方法(如2-opt、交换、重定位等)来改进解。

\textbf{2. 邻域选择与局部搜索(Local Search)}

在VNS的每一次迭代中,算法会通过下列步骤进行邻域搜索:

- \textbf{邻域选择:} 选择一个邻域结构 $N_k \in N$ 来对当前解进行局部优化。在本研究中,我们定义了几个常见的邻域结构,如:

  - \textbf{重定位邻域(Relocate Neighborhood):} 将一个订单从一条路径中移到另一条路径。
  - \textbf{交换邻域(Swap Neighborhood):} 在两条路径中交换订单。
  - \textbf{2-opt邻域(2-opt Neighborhood):} 反转路径中的某一段以消除交叉。
  - \textbf{交叉邻域(Cross Neighborhood):} 交换两条路径中的部分订单,重新排列路径。
  - \textbf{车型变更邻域(Change Vehicle Type):} 调整路径使用的车辆类型。

- \textbf{局部搜索(Variable Neighborhood Descent, VND):} 在每个邻域内,执行局部搜索,通过细化邻域中的解来改进目标函数。我们可以用以下公式表示局部搜索:

  \[
  x_{\text{new}} = \text{VND}(x_{\text{current}}, N_k)
  \]

  其中,$x_{\text{new}}$ 是通过局部搜索得到的新解,$x_{\text{current}}$ 是当前解,$N_k$ 是选定的邻域。

\textbf{3. 扰动操作(Shake Mechanism)}

扰动操作(shake)用于通过扩大邻域范围来跳出局部最优解。在VNS中,扰动通常通过随机选择一个邻域来进行。扰动操作的目标是跳出局部最优解空间,进入新的解空间,从而提供更大的搜索范围。

扰动操作可以通过如下公式表示:

\[
x_{\text{shake}} = \text{Shake}(x_{\text{current}}, N_k)
\]

其中,$x_{\text{shake}}$ 是经过扰动后的解,$N_k$ 是选定的邻域结构。

\textbf{4. 目标函数与成本计算}

VNS的目标通常是最小化配送的总成本或距离。假设我们有多个配送路径 $R_1, R_2, \dots, R_n$,每条路径的总成本为 $C(R_i)$,则总成本可以表示为:

\[
C_{\text{total}} = \sum_{i=1}^{n} C(R_i)
\]

其中,$C(R_i)$ 是路径 $R_i$ 的成本,通常包括行驶距离、时间、车辆使用等因素。

每条路径的成本计算公式如下:

\[
C(R_i) = d(0, r_1) + \sum_{j=1}^{|R_i|-1} d(r_j, r_{j+1}) + d(r_{|R_i|}, 0)
\]

其中,$r_1, r_2, \dots, r_{|R_i|}$ 表示路径上的订单节点,$d(x, y)$ 是从节点 $x$ 到节点 $y$ 的距离。


\textbf{6. 多样性策略}

为了防止VNS陷入局部最优并加速收敛,本研究采用了多样性管理策略。通过评估当前解和历史解的相似性,算法可以适时引入多样化策略,即执行更强的扰动操作,促进解的多样化。解的多样性可以通过计算当前解与历史解的相似度来量化,具体可以使用如下公式表示:

\[
\text{Diversity}(x, y) = \sum_{i=1}^{n} |x_i - y_i|
\]

其中,$x$ 和 $y$ 是两个不同的解,$n$ 是解的维度。

\textbf{7. VNS的算法流程}

VNS的算法流程可以概括为以下几个步骤:

1. 初始化解并设定邻域结构集合 $N_1, N_2, \dots, N_k$。
2. 在当前解上应用扰动操作,得到新解。
3. 在选择的邻域中执行局部搜索。
4. 如果新的解比当前解更优,则更新当前解;否则,选择下一个邻域。
5. 如果经过一定次数的迭代未能找到更优解,则执行解的多样化操作。

 \subsection{其他高级算法简述(RL、GNN等)}

在前述小节里,我们讨论了以 Clarke-Wright、模拟退火、禁忌搜索、变邻域搜索为代表的经典或相对成熟的启发式与元启发式算法,它们在多仓多路区场景下对下层 VRP 求解都有较好的可行性和实用性。然而,近些年兴起的**强化学习(RL)与图神经网络(GNN)**等人工智能方法也不断涌现于车辆路径规划(VRP)领域,被视为在大规模、动态、实时调度情形下具有潜在突破的高级算法。下文对此类算法做简要评述,并说明它们在本研究中的初步适配思路以及与基准算法的关系。

强化学习(Reinforcement Learning, RL)在传统机器学习中通过“状态-动作-奖励”三元组来训练一个智能体,使其在与环境交互过程中累积经验并学会最优策略 \cite{rl_intro}。在 VRP 中,RL 算法可将整个路线构造或节点选择过程映射为一个序列决策问题:在每一步,算法决定下一个节点要拜访哪里、或是否结束路线,并从即时奖励(例如节省行驶距离、避免超时)中学习。

图神经网络(Graph Neural Network, GNN)是近年在图结构数据学习中取得大量进展的深度学习模型,在 VRP 研究中也有应用:将 VRP 节点、边构造为图,训练 GNN 以学习对节点的优先度或对边的选择 \cite{gnn_intro}。例如,GNN + 大邻域搜索 (LNS) 常被视为可高效处理大规模问题的高级方案 \cite{gnn_lns}。


在本研究中,若有能力获取充足训练集,GNN 引导的大邻域搜索可在大规模路区 VRP 下展现强大性能,但初期工作量与适配远超过传统元启发式;因此目前我们将其列为“高级或探索性”候选算法进行合适的对比或试验,而不作为主力。

无论 RL 还是 GNN 都属于学习驱动或深度强化类算法,与前面提到的 Clarke-Wright、SA、TS、VNS 等元启发式相比,有以下主要区别:

    \begin{itemize}
        \item \textbf{依赖训练数据}
        \item CW 等元启发式可“即插即用”在任何 VRP 上;RL/GNN 通常需大规模先验算例或在线重复迭代训练,训练完再推理;
    \end{itemize}
    \begin{itemize}
    \item \textbf{在线计算 vs. 离线训练}

        \item 元启发式在每次需要解 VRP 时“从头算起”;RL/GNN 则若训练完成,可以快速在线推理(尤其在实时大规模下),但若环境变化大需频繁再训练。
    \end{itemize}

    \begin{itemize}
        \item 在实验中(见文献 \cite{rl_performance,gnn_performance}),对非常大规模或高度同质场景,RL/GNN 可能在上线后每次推理仅需毫秒,却能给出优于贪心甚至对比一些元启发式的解;但对局部突变或定制约束(如路区不能跨区、车辆容量变动)需额外适配网络结构,难度较高。
    \end{itemize}
本研究集中于基准算法(CW、SA、TS、VNS)在多仓多路区 VRP 的应用与实验。而 RL 与 GNN 等高阶学习类方法,我们仅做概念性讨论和部分小规模试验:若后续数据规模充分、路区变化相对稳定,则可考虑离线训练 RL/GNN 模型做在线调度,这对打车或外卖平台有一定吸引力 \cite{gnn_rl_application}。

然而,需要注意:
\begin{itemize}
    \item \textbf{训练成本}:构建 RL/GNN 所需的算例库与标签非常庞大;
    \item \textbf{环境变动}:一旦城市路段管制、订单密集度突然改变,模型需再训练或做迁移学习;
    \item \textbf{实现门槛}:除运筹学外,还需深度学习工程师与 GPU 资源投入。
\end{itemize}

因此,尽管 RL、GNN 在学术上与工业上都有亮点,其在多仓多路区配送中真正成熟落地仍需更长的迭代研究。于本文而言,我们暂把它们视作补充或潜在升级:若能克服训练代价并获取海量数据,则有望在极短时间内生成接近或优于元启发式的调度方案;若真实业务环境频繁变化或数据不足,传统 TS/VNS + 自适应机制更稳妥高效。

\textbf{选择算法的理由}

在本研究中,选择了四种算法:模拟退火(SA)、禁忌搜索(TS)、变邻域搜索(VNS)和克拉克-怀特算法(CW),这些算法在实际应用中具有较强的求解能力,并且在解决大规模车辆路径问题(VRP)时表现出较好的平衡性。预选的7种算法的表现如下表所示。以下是对每种算法的具体选择理由:

\begin{table}[ht]
\centering
\begin{tabular}{|l|l|l|}
\hline
\multicolumn{1}{|c|}{\multirow{2}{*}{}} & \multicolumn{2}{c|}{\textbf{Adaptive}} \\ \cline{2-3}
\multicolumn{1}{|c|}{} & \textbf{CW\_TS\_ADP} & \textbf{SA\_VNS\_AD} \\ \hline
\textbf{objectivemean} & 4.46 & 4.46 \\ \hline
\textbf{objectivestd} & 3.28 & 3.28 \\ \hline
\textbf{distancemean} & 0.58 & 0.58 \\ \hline
\textbf{distancetd} & 0.46 & 0.46 \\ \hline
\textbf{timemean} & 0.04 & 0.04 \\ \hline
\textbf{num\_routesmean} & 3.75 & 3.75 \\ \hline
\multicolumn{1}{|c|}{\multirow{2}{*}{}} & \multicolumn{2}{c|}{\textbf{Adaptive}} \\ \cline{2-3}
\multicolumn{1}{|c|}{} & \textbf{TS\_ADP\_ta} & \textbf{VNS\_ADP\_n+CW} \\ \hline
\textbf{objectivemean} & 17.48 & 4.46 \\ \hline
\textbf{objectivestd} & 14.44 & 3.28 \\ \hline
\textbf{distancemean} & 0.89 & 0.58 \\ \hline
\textbf{distancetd} & 0.81 & 0.46 \\ \hline
\textbf{timemean} & 0.16 & 0.14 \\ \hline
\textbf{num\_routesmean} & 16.46 & 3.75 \\ \hline
\multicolumn{1}{|c|}{\multirow{2}{*}{}} & \multicolumn{2}{c|}{\textbf{Baseline}} \\ \cline{2-3}
\multicolumn{1}{|c|}{} & \textbf{CW} & \textbf{SA} \\ \hline
\textbf{objectivemean} & 2.37 & 2.15 \\ \hline
\textbf{objectivestd} & 0.31 & 0.81 \\ \hline
\textbf{distancemean} & 0.35 & 0.52 \\ \hline
\textbf{distancetd} & 0.22 & 0.37 \\ \hline
\textbf{timemean} & 0 & 0 \\ \hline
\textbf{num\_routesmean} & 1.04 & 1.08 \\ \hline
\multicolumn{1}{|c|}{\multirow{2}{*}{}} & \multicolumn{2}{c|}{\textbf{Baseline}} \\ \cline{2-3}
\multicolumn{1}{|c|}{} & \textbf{TS} & \textbf{VNS} \\ \hline
\textbf{objectivemean} & 17.48 & 4.34 \\ \hline
\textbf{objectivestd} & 14.44 & 3.21 \\ \hline
\textbf{distancemean} & 0.89 & 0.47 \\ \hline
\textbf{distancetd} & 0.81 & 0.39 \\ \hline
\textbf{timemean} & 0.16 & 0.02 \\ \hline
\textbf{num\_routesmean} & 16.46 & 3.75 \\ \hline
\multicolumn{1}{|c|}{\multirow{2}{*}{}} & \multicolumn{2}{c|}{\textbf{Hybrid}} \\ \cline{2-3}
\multicolumn{1}{|c|}{} & \textbf{CW\_TS} & \textbf{CW\_VNS} \\ \hline
\textbf{objectivemean} & 17.48 & 4.34 \\ \hline
\textbf{objectivestd} & 14.44 & 3.21 \\ \hline
\textbf{distancemean} & 0.89 & 0.47 \\ \hline
\textbf{distancetd} & 0.81 & 0.39 \\ \hline
\textbf{timemean} & 0.16 & 0.02 \\ \hline
\textbf{num\_routesmean} & 16.46 & 3.75 \\ \hline
\multicolumn{1}{|c|}{\multirow{2}{*}{}} & \multicolumn{2}{c|}{\textbf{Hybrid}} \\ \cline{2-3}
\multicolumn{1}{|c|}{} & \textbf{SA\_TS} & \textbf{SA\_VNS} \\ \hline
\textbf{objectivemean} & 17.48 & 4.34 \\ \hline
\textbf{objectivestd} & 14.44 & 3.21 \\ \hline
\textbf{distancemean} & 0.89 & 0.47 \\ \hline
\textbf{distancetd} & 0.81 & 0.39 \\ \hline
\textbf{timemean} & 0.16 & 0.02 \\ \hline
\textbf{num\_routesmean} & 16.46 & 3.75 \\ \hline
\end{tabular}
\caption{Performance Comparison of Different Algorithm Configurations (Transposed)}
\label{tab:algorithm_comparison_transposed}
\end{table}


\textbf{模拟退火(SA)}

\begin{itemize}
    \item 选择理由:模拟退火算法是一种基于概率的全局优化方法,能够跳出局部最优解。通过模拟物理退火过程,SA算法具有较强的全局搜索能力,尤其适用于复杂的优化问题。在本研究中,SA算法在解决VRP时展现出较高的适应性,能够处理多种复杂的约束条件和目标函数。
    \item 表格数据分析:模拟退火算法的\textbf{平均总距离(Avg\_Total\_Distance)}为2.0995,\textbf{计算时间(Avg\_Computation\_Time)}较低(0.0154),但\textbf{路线数量(Avg\_Num\_Routes)}较多,这表明该算法具有较强的探索能力,但可能导致路径的冗余性较高,影响了效率。
\end{itemize}

\textbf{禁忌搜索(TS)}

\begin{itemize}
    \item 选择理由:禁忌搜索是一种局部搜索算法,通过维护一个禁忌表来避免搜索过程中回到已访问的解,从而提高了算法的局部优化能力。该算法适合在较复杂的局部搜索空间中寻找合适的解。在VRP问题中,禁忌搜索能较好地平衡探索和开发,在路径优化中表现出稳定的性能。
    \item 表格数据分析:禁忌搜索的平均总距离为0.6223,计算时间为0.0383,表现出较高的计算效率,同时路线数量适中,且\textbf{最大路线长度(Max\_Route\_Length)}较为合理,显示出算法能够有效地利用现有车辆。
\end{itemize}

\textbf{变邻域搜索(VNS)}

\begin{itemize}
    \item 选择理由:变邻域搜索算法是一种基于局部搜索的多种邻域结构的优化方法。它通过动态改变搜索的邻域结构,避免了陷入局部最优解的困境。VNS算法在处理具有复杂约束和大规模数据集的VRP问题时,能够灵活调整搜索策略,从而获得更好的解。
    \item 表格数据分析:VNS的平均总距离为0.6403,计算时间较高(0.5283),但其在平均路径长度和服务率上表现优越,服务率较高,表明算法在优化路径时能较好地满足客户需求。
\end{itemize}

\textbf{克拉克-怀特算法(CW)}

\begin{itemize}
    \item 选择理由:克拉克-怀特算法是一种经典的启发式方法,特别适合于大规模的VRP问题。该算法通过构建最短路径树来优化车辆的路径选择,具有较强的计算效率和较好的解决方案质量。由于其简单有效的特性,CW算法在许多实际问题中得到了广泛应用。
    \item 表格数据分析:克拉克-怀特算法的平均总距离为0.4930,是所有算法中最优的,且其计算时间较低(0.0292),证明该算法在性能和效率方面均表现优秀。最大路径长度较长,但最小路径长度表现合理,且目标值最优,显示出其在路径优化方面的优势。
\end{itemize}

\textbf{排除其他算法的理由}

根据表格数据和实际应用需求,其他算法未被选择的主要原因如下:

\textbf{有监督学习算法 LearningGuidedSolver}

\begin{itemize}
    \item 排除理由:尽管该算法的服务率较高,但其计算时间(18.6秒)显著高于其他算法,并且在总距离和路径数量的表现上并没有明显优势。因此,从效率和结果优化的角度考虑,未能选择该算法。
\end{itemize}

\textbf{图神经网络算法 GNNGuidedLNSSolver}

\begin{itemize}
    \item 排除理由:该算法虽然在总距离(0.4155)上表现良好,但其计算时间(0.8326秒)较长,且路径的最大长度(26)较大,可能导致一些路径过长,影响了整体的配送效率。因此,尽管该算法具有潜力,但从效率和稳定性的角度,未被选中。
\end{itemize}

\textbf{分层强化学习算法 HierarchicalReinforcementLearningSolver}

\begin{itemize}
    \item 排除理由:该算法的计算时间(0.2761秒)较短,且服务率和速度较高。然而,其在总距离上的表现(0.6588)不如其他几种算法,且最大路径长度为10,路径规划的效果存在一定的冗余性。因此,从性能和效率上考虑,该算法未被选中。
\end{itemize}


\section{混合算法框架设计}
\subsection{混合策略的算法组合}
在本研究中,这些算法的实验与表现为:
\begin{itemize}
    \item \textbf{对比 Clarke-Wright / SA}:在多数算例中(包括路区规模 200~1500 节点),TS 在 0.1~1 秒内完成数百迭代,能稳定优化结果。
    \item \textbf{二阶段混合}:CW 或 SA 可先快速生成初解,用 TS 做深度改进。常见实验显示,“CW/SA + TS” 方式可减少 30~50\% 的迭代时间,且最终解质量相当或更好。
    \item \textbf{多车型实测}:若某路区有大中小三种车型可选,TS 邻域操作里需加“车型切换”考虑。实践表明,允许车型灵活切换能提升解质量,但也会增大搜索空间。
    \item \textbf{线上动态调度}:若订单突然增加,先局部破坏当前解,再进入 TS 多邻域搜索以迅速修复——特别适合外卖/打车等场景。
\end{itemize}

\textbf{优点}
\begin{itemize}
    \item 搜索深度:比 Clarke-Wright 等简单构造更能做细致局部优化,常取得更优解;
    \item 禁忌记忆:能避免短期循环,尤其在高维VRP中尤为重要;
    \item 可扩展:易于融合多车型、多仓、容量、甚至部分覆盖等多重约束,添加相应邻域判定。
\end{itemize}

\textbf{缺点}
\begin{itemize}
    \item 参数敏感:禁忌期限、邻域大小等需调试;
    \item 计算时间:在大规模路区,TS 单轮迭代若邻域庞大易耗时;
    \item 易早停:若缺乏适度的随机或志愿准则破禁,可能在复杂问题中陷入“次优”长时间不动。
\end{itemize}

\subsection{前端算法的确定:CW 与 SA}
\textbf{Clarke-Wright (CW)实验要点:}
\begin{itemize}
    \item 从结果文件中可见,CW 的平均总路程约 493,平均耗时仅 0.29 秒,是所有算法中最快;
    \item 其最优解可到 ~104,说明当节省值策略适合时,CW 的路径融合较紧凑;最差解可达 ~0065,表明在少数极端算例下会明显退化。
\end{itemize}

\textbf{分析:}
\begin{itemize}
    \item CW 算法本质上是以“节省值 (Saving) = $d_{0i} + d_{0j} - d_{ij}$”作为合并依据的贪心策略,在实现上非常轻量;
    \item 其速成解常略带“分段结构”,后续若配合更强本地搜索,可以进一步削减空驶或多余绕路;
    \item 整体而言,CW 很适合作为构造型初始解:花费极少计算量就能得到一条(甚至多条)可行方案;对大规模 VRP 初步算路尤其有效。
\end{itemize}

\textbf{Simulated Annealing (SA)实验要点:}
\begin{itemize}
    \item SA 的平均路程 (0.995) 表面上较 CW 更大,但其计算时间 (0.154 秒) 依旧短小,可见在退火过程中只做了少量迭代;
    \item 最优解时可低至 122,而最差时上至 1108,显示 SA 对邻域扰动非常宽泛、不够稳定,容易“一次运行好、一次运行差”;
    \item 平均路线数可达 551 条,偏“碎片化”。
\end{itemize}

\textbf{分析:}
\begin{itemize}
    \item 由于退火允许大步随机跳跃,SA 能在短期内“洗牌”解结构,快速尝试多种可行路径;只不过它可能陷入较大的波动;
    \item 若本研究想要“多样化”初始解(或多起点),SA 生成的解恰能提供广覆盖;后续再做合并、局部搜索即可;
    \item 与 CW 一样,SA 的核心优点是“用时极少”,在数秒乃至更短限制下亦能给出多批候选。
\end{itemize}

因此,对于前端阶段,我们希望快速获得多个较可行、路线分布多样的解——CW 与 SA 分别以节省值贪心和随机退火的方式为后续深度优化打下基础。它们都满足“高效率 + 多样性”的需求。


\subsection{后端算法的组合:TS 与 VNS}
\textbf{Tabu Search(TS)实验要点:}
\begin{itemize}
    \item 平均总路程约 6223,较接近 CW,但最优值可逼近 0230;
    \item 标准差约 3009,说明结果有一定波动,但不至于像 SA 那样跨越极大区间;
    \item 耗时 (0.383s) 虽然比 CW/SA 略长,但在高质量局部搜索中属于正常范围。
\end{itemize}

\textbf{学术解释:}
\begin{itemize}
    \item TS 通过维护禁忌表 (Tabu List)、记录已访问过的邻域,能在重复迭代中有效跳出局部最优;
    \item 若给定一个质量尚可的初始解(如由 CW 或 SA 提供),TS 在此之上可多轮迭代,往往能把解向全局最优逼近;
    \item 在后端花费多一点时间,可换来显著的质量提升——正契合后端“精雕细琢”的角色定位。
\end{itemize}

\textbf{Variable Neighborhood Search (VNS)实验要点:}
\begin{itemize}
    \item 平均总路程 6403,与 TS 相似,最优解也可低至 0230;
    \item 计算时间 528 秒,是本批算法中最“耗时”,因其反复大邻域 shake + 局部搜索;
    \item 在若干实例中,VNS 与 TS 水平相当,一旦 shake 得当,可获得极优解。
\end{itemize}

\textbf{学术解释:}
\begin{itemize}
    \item VNS 在切换多尺度邻域 ($k=1,2,\dots$) 时既能进行大破坏大修复,也能在后期微调;
    \item 若初始解质量良好,就能通过合适的邻域序列将解不断强化;
    \item 虽然运行时间偏高,但其搜索深度较之 CW/SA 大幅提高,足以在后端阶段完成“精修 + 大幅跃迁”。
\end{itemize}

综上,TS 与 VNS 都具备深度搜索与局部微调能力,并在实验数据中体现出强大“再改进”潜力,适合部署在后端做最终的提优环节。

\textbf{对其他算法的排除说明}

\textbf{Learning Guided Solver}
\begin{itemize}
    \item 实验中其平均时间约 16 秒,明显大于 TS/VNS 的 5 秒级别;
    \item 虽然在部分测试可达优解,但其学习曲线需要更多迭代或记忆更新;不易融入“快速前端 + 局部优化后端”的两阶段流程;
    \item 故在当前数据与时间限制下,先行排除。
\end{itemize}

\textbf{GNN Guided LNS Solver}
\begin{itemize}
    \item 目标值可达 415,甚至优于 TS/VNS 的平均,但计算时间 (8326s) 仍高,且还需神经网络搭建与推理;
    \item 对初始解构造的需求也不低,限制了它在前端的适用性;
    \item 若将其放后端,则其调参和训练成本往往超过 TS/VNS,故暂不选择。
\end{itemize}

\textbf{Hierarchical Reinforcement Learning Solver}
\begin{itemize}
    \item 多层 RL 体系结构需要大量回合训练;
    \item 即使实验中平均距离与 TS/VNS 相差不大,也难以在短周期内稳定收敛或复现优解;
    \item 在第一阶段聚焦简洁与性能时,此多层 RL 策略并不占优。
\end{itemize}

\textbf{结论与下一步整合}
综上所述,结合前端算法需要的“极短时间、迅速生成多样化初始解”之需求,以及后端算法对“深度邻域搜索、迭代时间较充分”的要求,本章实验中我们选取:
\begin{itemize}
    \item Clarke-Wright (CW) 和 Simulated Annealing (SA):前端快速解构造 / 初步随机全局探索;
    \item Tabu Search (TS) 和 VNS:后端针对前端解做深度邻域搜索,持续改进路线质量。
\end{itemize}

根据第一阶段实验的量化结果(详见表 X-2 与图 X-3),此“四算法”方案在平均总路程、最优解水平、运行时间等指标上相互补位,可有效搭建起后续“二阶段混合算法”或“自适应调优”的基础框架。

\begin{table}[ht]
\centering
\caption{四算法在实验中的典型统计(示例摘自前述 CSV 汇总)}
\begin{tabular}{|c|c|c|c|c|}
\hline
Algorithm & Avg\_Total\_Dist & Avg\_Time (s) & Best\_Obj & Worst\_Obj \\
\hline
CW & 493 & 0.29 & 104 & 0.065 \\
SA & 0.995 & 0.154 & 122 & 1108 \\
TS & 6223 & 0.383 & 230 & 5232 \\
VNS & 6403 & 528 & 230 & 5150 \\
\hline
\end{tabular}
\end{table}

在后续章节中,我们将基于这四大算法组成多组二阶段混合,如“CW + TS”“SA + VNS”等,并针对自适应参数(Adaptive TS / VNS)进行进一步实验验证,以评估其在解质量、时间代价、稳定性方面的综合表现。
\section{自适应机制设计与实现}

\textbf{自适应框架概述}
在这一章中,我们将详细探讨自适应机制的设计与实现,尤其是在禁忌搜索(TS)与变邻域搜索(VNS)算法中引入的自适应策略。这些自适应策略的核心目标是根据当前的搜索状态,自动调节算法的关键参数,以提高搜索效率并优化结果。

为了求解车辆路径规划(VRP)问题,我们将禁忌搜索(TS)和变邻域搜索(VNS)作为基础算法,结合自适应机制实现动态参数调整。

\subsection{自适应禁忌搜索(TS-ADP)}
禁忌搜索(Tabu Search)是一种局部搜索算法,它通过不断的迭代搜索解空间,并利用禁忌表记录已访问的解,避免回到相同的解。在传统的禁忌搜索中,禁忌表的长度和更新规则是固定的,而在自适应禁忌搜索(TS-ADP)中,我们引入了自适应机制,根据搜索的进展情况动态调整禁忌表的大小。

\textbf{参数调节:}禁忌表长度(即 tabu\_size)是禁忌搜索的关键参数之一。自适应机制通过计算改进率来决定是否增大或缩小禁忌表的长度。当连续几轮没有找到更好的解时,禁忌表长度增加,以探索更多的解空间,避免陷入局部最优解。当找到较好的解时,禁忌表长度适当减小,从而加快搜索速度。

\textbf{改进率:}在每次迭代后,我们通过计算当前解与上一轮解之间的相对改进率(即 $\Delta(t)$)来判断是否需要调整参数。改进率定义为:
\[
\Delta^{(t)} = \frac{f(S^{(t-1)}) - f(S^{(t)})}{f(S^{(t-1)})}
\]
其中,$f(S^{(t-1)})$ 和 $f(S^{(t)})$ 分别表示上一轮和当前解的目标函数值。如果 $\Delta(t) > 0$,说明当前解有所改进;如果 $\Delta(t) < 0$,则说明解变差,应该适当增大禁忌表长度。

\subsection{自适应变邻域搜索(VNS-ADP)}
变邻域搜索(Variable Neighborhood Search)是一种通过在多个邻域间切换来避免陷入局部最优解的优化方法。在VNS中,邻域的大小和扰动强度(shake intensity)是影响搜索效果的两个重要因素。自适应VNS(VNS-ADP)则根据搜索过程中解的改进情况动态调整这些参数。

\textbf{扰动强度调整:}扰动强度决定了每次解的变动幅度。若当前解长时间未能改善,算法会增大扰动强度,以增加解空间的搜索深度;反之,当找到较好的解时,扰动强度减小,以避免搜索过于激进,从而提高解的质量。

\textbf{邻域规模调整:}邻域规模决定了搜索的广度。若在短时间内没有显著的改进,算法会增加邻域的规模,从而扩大搜索范围;若改进显著,则缩小邻域范围,以提高搜索效率。


\subsection{参数更新策略与阈值设置}
在自适应算法中,参数的动态调整是一个逐步优化的过程。为了避免参数调整过于剧烈,通常会引入阈值来判断是否触发参数更新。此外,为了确保参数调整既有效又不过度,我们采用以下几种策略来决定何时以及如何更新参数。

\textbf{动态更新策略}
参数更新遵循以下的基本原则:
\begin{itemize}
    \item 如果改进率为正且连续几轮有改进,则说明当前参数设置能够持续改进解,此时可以适当缩小搜索空间,减少扰动强度或禁忌表的大小,以加速算法的收敛。
    \item 如果改进率为负或停滞阶段,则说明当前参数设置导致搜索空间的收敛过早或局部搜索不够深入,此时需要增加扰动强度或扩展邻域范围。
\end{itemize}

\textbf{更新策略如下:}禁忌表长度更新:如果平均改进率 $\overline{\Delta}(t)$ 大于设定阈值 $\varepsilon$,则减小禁忌表长度,即:
\[
\theta^{(t+1)} = \max(\theta_{\min}, \theta^{(t)} - 1)
\]
其中,$\theta_{\min}$ 是禁忌表长度的下限,避免禁忌表过小。

如果 $\overline{\Delta}(t)$ 小于或等于 $\varepsilon$,则增大禁忌表长度:
\[
\theta^{(t+1)} = \min(\theta_{\max}, \theta^{(t)} + 1)
\]
其中,$\theta_{\max}$ 是禁忌表长度的上限,避免禁忌表过大。

邻域规模更新:若搜索在某一邻域内停滞过长时间,例如连续几轮改进率为负,我们可能需要增加邻域的规模,以便探索更多可能的解空间。这时邻域规模会乘以一个常数因子,如:
\[
k^{(t+1)} = k^{(t)} \times \gamma
\]
其中,$\gamma > 1$ 表示扩大邻域范围。

若搜索成功获得较大改进,则减少邻域规模,以便缩小搜索范围并加速收敛:
\[
k^{(t+1)} = \max(k_{\min}, k^{(t)} / \gamma)
\]
其中,$k_{\min}$ 是邻域规模的下限。


\chapter{VRP算法实验设计与实现}

\section{数据集获取与预处理}

\subsection{数据来源与采集}

\subsection{数据清洗与预处理流程}

\subsection{数据标准化与特征工程}

\section{实验设计与评估体系}

\subsection{实验对比分析(有无路区划分)}

\textbf{目标值(Objective)对比}

在无路区划分的情况下,算法的目标值普遍较高,尤其是CW和SA算法,其目标值波动较大,优化效果相对有限。无路区带来的路径规划复杂性,导致了优化难度的增加,区域之间的边界无法有效划定,结果是目标值无法有效控制。而在引入路区划分后,尤其是CW和SA算法的目标值显著下降,优化幅度高达93\%和87\%。这种变化表明,路区划分不仅帮助这些算法更精确地控制路径优化,还减少了无效绕行,显著提升了整体配送效率。对于TS和VNS算法来说,目标值的变化幅度相对较小,分别为24\%和49\%。这说明,TS和VNS在无路区的环境中已经具有较强的优化能力,路区划分对它们的进一步提升效果并不如CW和SA那样明显。

\begin{figure}[H]
    \centering
    \includegraphics[width=0.8\textwidth]{figure01.png} % 目标值对比图
    \caption{目标值对比}
    \label{fig:objective_comparison}
\end{figure}

\textbf{行驶距离(Distance)对比}

同样地,在无路区的情况下,各个算法的行驶距离普遍较长,特别是CW算法,路径优化的能力相对较弱,导致车辆行驶的总距离居高不下。缺乏有效的路区划分,算法的路径选择复杂,行驶路线难以高效规划。然后,当路区划分介入时,所有算法的行驶距离均大幅下降,特别是CW算法,行驶距离的减少幅度最大,达到了惊人的97\%。这显然表明,路区划分在大大减少不必要的行驶里程上发挥了重要作用。尽管TS和VNS算法的行驶距离下降幅度较小,分别为87\%和84\%,但它们的优化效果依然显著,进一步证明了路区划分在提升配送效率方面的重要性。

\begin{figure}[H]
    \centering
    \includegraphics[width=0.8\textwidth]{figure02.png} % 行驶距离对比图
    \caption{行驶距离对比}
    \label{fig:distance_comparison}
\end{figure}

\textbf{计算时间(Time)对比}

在没有路区划分的情况下,CW和VNS的计算时间较长,表明在缺乏区域限制时,算法的路径搜索空间较大,计算过程变得复杂,效率低下。而路区划分的引入优化了算法的搜索空间,所有算法的计算时间显著减少。特别是CW和VNS算法,计算时间分别下降了99\%和94\%,这表明路区划分不仅在优化路径方面有效,同时在计算效率上也做出了巨大的贡献。尽管TS和SA算法的计算时间也有所减少(TS:94\%,SA:80\%),但由于这些算法本身在没有路区划分时已经具有较高的效率,它们的计算时间下降幅度相对较小。

\begin{figure}[H]
    \centering
    \includegraphics[width=0.8\textwidth]{figure03.png} % 计算时间对比图
    \caption{计算时间对比}
    \label{fig:time_comparison}
\end{figure}

\textbf{路径数量(Num Routes)对比}

无路区划分时,路径数量相对较多,尤其是CW和SA算法。这是因为在没有路区划分的情况下,算法难以有效合并路径,产生了更多冗余的路线。而一旦路区划分介入,尤其是在CW和SA算法中,路径数量显著减少,分别下降了96\%和75\%。这表明,路区划分通过优化路径合并策略,帮助减少了不必要的路径生成,从而提升了整体的配送效率。然而,TS算法的路径数量变化较小,下降幅度仅为5\%,这表明TS算法的路径优化在无路区划分时已经相对稳定,路区划分对它的影响较小。值得注意的是,VNS算法的路径数量反而略微上升(下降幅度为-7\%),这可能与其路径合并策略的调整有关,部分路径被拆分以优化整体路径规划。

\begin{figure}[H]
    \centering
    \includegraphics[width=0.8\textwidth]{figure04.png} % 路径数量对比图
    \caption{路径数量对比}
    \label{fig:route_comparison}
\end{figure}

\textbf{改进率(Improvement Rates)分析}

从整体改进率分析来看,路区划分的引入对算法的优化效果产生了显著影响,尤其对于CW和SA算法,目标值、行驶距离、计算时间和路径数量等各项指标的改进率分别达到了93\%和87\%。对于TS和VNS算法而言,尽管它们在无路区划分时已经有较强的优化表现,但路区划分依然带来了优化效果,改进率分别为49\%和54\%。这种差异反映出路区划分对不同算法的影响程度不同,但无疑为大多数算法的优化带来了积极的推动。

\begin{figure}[H]
    \centering
    \includegraphics[width=0.8\textwidth]{figure05.png} % 改进率分析图
    \caption{改进率分析}
    \label{fig:improvement_rates}
\end{figure}

\textbf{总结}

综上所述,路区划分对算法的优化效果有着显著的提升,尤其是在CW和SA算法的目标值、行驶距离、计算时间和路径数量方面,改进幅度最大。引入路区划分不仅优化了路径规划和减少了无效绕行,还大大提高了计算效率。对于TS和VNS算法来说,尽管它们在没有路区划分时已经表现出较强的优化能力,但路区划分依然带来了不容忽视的改进。未来的研究可以进一步探索如何根据实时订单数据和配送需求动态调整路区划分策略,以进一步提升配送调度系统的适应性和效率。

\subsection{实验对比分析(不同规模数据集)}

在对比分析大规模与小规模订单模型的性能时,我们着重考察了以下五个关键指标:时间(time)、目标值(objective)、路径数量(num\_routes)、距离(distance)。这些指标不仅反映了算法在不同订单规模下的效率差异,也揭示了混合算法在处理大规模订单时展现的强大优势。

\textbf{时间优化}

从图6改进率上看,大规模订单模型在混合算法的优化下显示出了显著的性能提升。特别是在 VNS 和 CW 等混合算法的帮助下,大规模场景下的时间优化提升率在87\%到98\%之间,而小规模模型的时间改进率则相对较低。该现象体现了随着订单规模的增加,混合算法通过对路径和车辆调度的综合优化,显著提升了配送任务的响应速度。这一结果表明,大规模订单处理对算法的路径规划和调度策略要求更高,而混合算法能够有效地减少配送所需的时间。

\begin{figure}[H]
    \centering
    \includegraphics[width=0.8\textwidth]{figure06.png} % 时间优化图
    \caption{时间优化}
    \label{fig:time_optimization}
\end{figure}

\textbf{目标值优化}

目标值通常涉及到成本、效率或服务质量等多维度的优化目标。在对比图7中,混合算法在大规模订单环境下的表现优于小规模模型,改进率高达91\%至92\%。这一显著提升表明,在面对大规模复杂配送任务时,混合算法能够更有效地优化整体的目标函数,降低运作成本并提高资源利用率。尤其是对于大规模订单,算法能够实现更精确的车辆调度与资源分配,从而达到了成本效益的最大化。

\begin{figure}[H]
    \centering
    \includegraphics[width=0.8\textwidth]{figure07.png} % 目标值优化图
    \caption{目标值优化}
    \label{fig:objective_optimization}
\end{figure}

\textbf{路径数量优化}

路径数量是配送效率的一个重要指标,直接影响到资源的配置和车辆的使用效率。对比图8结果显示,在大规模订单模型下,混合算法的路径数量优化表现尤为突出,改进率普遍达到92\%至97\%。相较而言,小规模模型的优化效果略显平缓,改进率在94\%到91\%之间。这一结果验证了混合算法在处理大量订单时的路径规划能力,尤其是在避免重复路径和冗余配送方面的优势。在大规模订单的背景下,混合算法能够有效减少配送路径数量,从而提高配送效率和车辆利用率。

\begin{figure}[H]
    \centering
    \includegraphics[width=0.8\textwidth]{figure08.png} % 路径数量优化图
    \caption{路径数量优化}
    \label{fig:route_optimization}
\end{figure}

\textbf{距离优化}

在距离优化方面,混合算法在大规模订单的背景下表现出了非常强的优化能力。如图9所示,改进率普遍保持在93\%到92\%之间,远超小规模模型中的98\%至95\%的改进率。这表明,随着订单规模的增大,混合算法在车辆调度和路径选择上的优化潜力被充分发挥出来,从而减少了配送过程中的空驶距离和无效行驶,提高了整体的配送效率。大规模订单的复杂性和多变性使得混合算法能够更加精确地进行资源配置与路径选择,显著降低了运输过程中的总行驶距离。

\begin{figure}[H]
    \centering
    \includegraphics[width=0.8\textwidth]{figure09.png} % 距离优化图
    \caption{距离优化}
    \label{fig:distance_optimization}
\end{figure}

\textbf{结论}

综合来看,混合算法在大规模订单模型中的优势十分明显,特别是在时间、目标值、路径数量和距离四个关键指标上的优化效果显著优于小规模订单模型。随着订单规模的扩大,混合算法展现了更为复杂的路径规划和调度能力,使得其在大规模配送任务中的表现得到了充分的体现。如图10所示,这一趋势表明,混合算法不仅适应了大规模订单的高复杂度和高要求,还通过精细化的优化策略提升了配送效率和资源利用率,显著推动了物流领域中的成本节约和效能提升。

\begin{figure}[H]
    \centering
    \includegraphics[width=0.8\textwidth]{figure10.png} % 距离优化图
    \caption{不同规模的订单改进率对比}
    \label{fig:improvement rate}
\end{figure}

\subsection{评估指标体系与性能分析(待补充)}

\section{算法实验}
\subsection{基线算法实验}
\subsection{混合算法实验}
\subsection{自适应算法实验}
\subsection{多车型(多商户类型)实验}

\section{实验结果分析}
\subsection{实验结果统计与可视化}
\subsection{算法性能对比与显著性分析}


\textbf{1. 多车型场景下的适应性}

在多车型场景中,混合算法展现了显著的优势,尤其是在大车型的调度上。图11展示了混合算法在不同车型上的平均路径长度,其中CW\_TS和SA\_VNS的表现尤为突出。特别是大车型上,混合算法在平均路径长度上的表现比其他算法低了30\%以上。大车型的载重优势得以充分发挥,而小型和中型车辆的调度也通过混合算法得到了有效优化,优化幅度明显高于基线算法。相比之下,基线算法在多车型调度下的表现明显不如混合算法,尤其是在中小车型的调度上,路径优化效果较弱。

自适应算法(如SA\_VNS\_ADP\_first\_stage和TS\_ADP\_tabu\_size)的改进虽然存在,但与混合算法相比,改进幅度较小,尤其是在小型和中型车的调度中,优化效果较为平淡。

\begin{figure}[H]
    \centering
    \includegraphics[width=0.8\textwidth]{figure11.png} % 多车型适应性图
    \caption{多车型场景下的适应性}
    \label{fig:multi_vehicle_adaptability}
\end{figure}

\textbf{2. 行驶距离的优化}

从图12关于行驶距离的对比可以看出,混合算法(如CW\_VNS和SA\_TS)在减少行驶距离上的表现远超其他算法。尤其是在大规模订单的情景下,混合算法通过灵活的车辆调度和路径规划,实现了行驶距离的显著缩短,改进幅度接近20\%。例如,CW\_VNS在大规模订单的行驶距离优化上,比基线算法(如TS)高出了15\%。这一结果进一步证明了混合算法在复杂任务中的强大调度能力,特别是在大规模配送任务中的资源优化能力。

相较而言,自适应算法的表现则不如混合算法,尽管在某些情况下也能优化行驶距离,但相比混合算法,它们的效果显得较为局限。

\begin{figure}[H]
    \centering
    \includegraphics[width=0.8\textwidth]{figure12.png} % 行驶距离优化图
    \caption{行驶距离的优化}
    \label{fig:distance_optimization}
\end{figure}

\textbf{3. 计算时间对比}

图13展示了不同算法在计算时间上的对比,混合算法和自适应算法在这一方面的表现更加复杂。在大规模配送任务中,混合算法(如CW\_VNS)的计算时间相对较长,但最终的优化效果是值得的。在图表中,混合算法的计算时间通常比基线算法高出10\%,然而,最终的路径优化和成本降低远超计算时间的增加。相比之下,基线算法(如SA和VNS)计算时间较短,但其优化效果明显较弱,尤其是在大规模订单的处理上,无法达到混合算法的优化效果。

自适应算法在计算时间上通常介于混合算法和基线算法之间,虽然计算时间相对较短,但在优化的深度和效果上依然与混合算法有所差距。

\begin{figure}[H]
    \centering
    \includegraphics[width=0.8\textwidth]{figure13.png} % 计算时间对比图
    \caption{计算时间对比}
    \label{fig:computation_time_comparison}
\end{figure}

\textbf{4. 算法改进率}

图14揭示了各算法在改进率上的差异,尤其是在目标值、路径数量、行驶距离等维度的表现。混合算法(如CW\_TS和SA\_VNS)在多个指标上的改进率普遍超过了90\%,其中CW\_TS的改进率高达95\%,在路径数量和目标值的优化上表现尤为突出。与此相比,基线算法的改进率则普遍较低,尤其在路径数量的优化上,改进幅度最大仅为85\%左右。

自适应算法的改进率虽高于基线算法,但仍未能与混合算法相提并论。例如,SA\_VNS\_ADP\_first\_stage的改进率在大多数场景下为88\%,远低于混合算法的表现。

\begin{figure}[H]
    \centering
    \includegraphics[width=0.8\textwidth]{figure14.png} % 算法改进率图
    \caption{算法改进率}
    \label{fig:improvement_rates}
\end{figure}

\textbf{5. 商户类型对算法的影响}

图15展示了不同商户类型下算法的表现。混合算法(如CW\_TS和SA\_VNS)在超市和便利店场景下表现尤为突出,改进率均超过了95\%。其中,SA\_VNS在超市配送任务中的表现尤其亮眼,成功减少了路径数量并优化了配送时间。这表明,混合算法能够根据商户的需求特性,灵活调整调度策略。

自适应算法在不同商户类型下也表现得较为灵活,特别是在应对便利店场景中的高频次配送时,改进效果较为明显。然而,在超市这种需求较为稳定的场景下,混合算法的优化效果依旧优于自适应算法。

\begin{figure}[H]
    \centering
    \includegraphics[width=0.8\textwidth]{figure15.png} % 商户类型对算法影响图
    \caption{商户类型对算法的影响}
    \label{fig:merchant_type_impact}
\end{figure}

\textbf{6. 车辆类型对算法的影响}

图16展示了不同车辆类型下,算法的稳定性表现。混合算法在大车型的调度上表现出色,尤其是在大规模配送任务中,通过精确的路径规划和资源调度,优化了大车型的运输效率。而在小车型和中型车辆的调度中,混合算法同样表现不俗,通过灵活调度提升了资源利用率。

与此相比,基线算法在不同车型的调度上并未显示出显著优势,尤其在小型和中型车辆的调度中,其优化效果较为平淡。自适应算法的表现则介于混合算法与基线算法之间,在部分车型的调度上表现良好,但整体优化效果尚未达到混合算法的水平。

\begin{figure}[H]
    \centering
    \includegraphics[width=0.8\textwidth]{figure16.png} % 车辆类型对算法影响图
    \caption{车辆类型对算法的影响}
    \label{fig:vehicle_type_impact}
\end{figure}

\textbf{7. Levene F 检验与统计显著性}

在Levene F检验中(图17和图18),统计结果表明,混合算法在路径数量和时间上的改进显著高于基线算法,并且具有较强的统计学意义。特别是在路径数量的改进上,混合算法与自适应算法的差异更为显著,而基线算法在统计显著性上则未能达到同样的标准。

\begin{figure}[H]
    \centering
    \includegraphics[width=0.8\textwidth]{figure17.png} % Levene F 检验图
    \caption{Levene F 检验}
    \label{fig:levene_f_test}
\end{figure}

\textbf{8. P 值与统计分析}

从P值的检验结果来看(图19和图20),混合算法与自适应算法在优化路径数量和时间方面的效果具有显著的统计差异,表明这些算法在大规模订单场景下的优化效果更为可靠。基线算法则未能在统计分析中表现出显著优势,其P值较高,表明其优化效果在大多数情况下缺乏足够的显著性。

\begin{figure}[H]
    \centering
    \includegraphics[width=0.8\textwidth]{figure19.png} % P 值检验图
    \caption{P 值与统计分析}
    \label{fig:p_value_analysis}
\end{figure}

\textbf{结论}

通过对比不同算法在多个维度上的表现,我们可以清晰地看到,混合算法(如CW\_TS、SA\_VNS等)在时间、目标值、路径数量和距离等方面的优化效果显著高于基线算法。尤其在大规模订单的情境下,混合算法通过更精细的路径规划和调度策略,实现了优异的资源利用和成本控制。而自适应算法(如SA\_VNS\_ADP\_first\_stage)虽然在某些场景中表现出色,但总体上仍然略逊于混合算法。

基线算法虽然在一些简单场景下表现尚可,但在处理大规模和复杂配送任务时,未能有效发挥优化潜力,未能达到混合算法和自适应算法的水平。因此,在实际应用中,混合算法无疑是应对复杂配送任务的首选,能够大幅提升配送效率并降低成本。

\begin{figure}[H]
    \centering
    \includegraphics[width=0.8\textwidth]{figure20.png} % P 值检验图
    \caption{P 值与统计分析}
    \label{fig:p_value_analysis_final}
\end{figure}
\backmatter

\chapter{总结与展望}

\section{研究总结}

\section{研究贡献}

\section{研究不足}

\section{未来展望}




\chapter*{参考文献}
\addcontentsline{toc}{chapter}{参考文献}
这里是参考文献列表。
\bibliographystyle{plain}  % 选择参考文献样式
\bibliography{references}
\chapter*{附录}
\addcontentsline{toc}{chapter}{附录}

\section*{附录A\quad 核心算法与关键代码示例}
\addcontentsline{toc}{section}{附录A\quad 核心算法与关键代码示例}
这里是核心算法代码。

\section*{附录B\quad 实验原始数据与补充结果}
\addcontentsline{toc}{section}{附录B\quad 实验原始数据与补充结果}
这里是实验数据。



\subsection{路区管理的创新思维}
针对上述问题,本研究提出的路区管理思路具有以下创新特点:

\begin{itemize}
    \item 分区独立化
    \begin{itemize}
        \item 将城市划分为多个相对独立的配送区域;
        \item 每个路区设置一个专属仓库(有且仅有一个);
        \item 实现"区域自治",降低调度复杂度。
    \end{itemize}
    
    \item 仓配一体化
    \begin{itemize}
        \item 路区仓库选址充分考虑配送半径与人口密度;
        \item 仓储与配送系统协同优化;
        \item 缩短"最后一公里"配送距离。
    \end{itemize}
    
    \item 运力精细化
    \begin{itemize}
        \item 根据路区特点选择合适运力(如小型新能源车等);
        \item 减少大型车辆在市区频繁穿行;
        \item 提高车辆利用率,降低空驶成本。
    \end{itemize}
\end{itemize}

\subsection{路区管理与前置仓模式的对比分析}
虽然盒马鲜生等已推行"前置仓"模式,但本研究的路区管理思路具有以下显著特点:

\begin{itemize}
    \item 规划理念
    \begin{itemize}
        \item 前置仓:以单个商圈为中心的"点状"布局;
        \item 路区管理:以完整区域为单位的"面状"覆盖;
        \item 本研究强调区域整体规划,而非简单的仓储前移。
    \end{itemize}
    
    \item 运作模式
    \begin{itemize}
        \item 前置仓:仍需上游中心仓供给,存在多级调拨;
        \item 路区管理:区域仓库可直接收货,扁平化管理;
        \item 本研究减少了中间环节,提高运营效率。
    \end{itemize}
    
    \item 调度策略
    \begin{itemize}
        \item 前置仓:多为单一品类(如生鲜)的专门配送;
        \item 路区管理:支持多品类、多场景的综合服务;
        \item 本研究具有更强的普适性和扩展性。
    \end{itemize}
\end{itemize}

特别地,本研究在以下方面具有显著创新:
\begin{itemize}
    \item 将传统的静态路区划分扩展为动态自适应模式,可根据时段、天气等因素灵活调整区域边界;
    \item 在VRP求解中引入多层次记忆与学习机制,显著提升算法在实际场景中的稳定性;
    \item 提出"核心-边缘"的混合覆盖策略,平衡了全覆盖与部分覆盖的矛盾。
\end{itemize}

这些理论创新与实践价值,不仅对学术研究具有重要参考意义,也为行业发展提供了可落地的解决方案。在"双碳"目标和数字经济背景下,本研究的成果将助力城市物流向更智能、更绿色的方向发展。


\end{document}
% 在文档末尾添加
\bibliographystyle{plain}  % 或其他样式
\bibliography{references}   % 不需要写.bib后缀